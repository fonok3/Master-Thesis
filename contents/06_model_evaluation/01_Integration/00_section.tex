\section{Integration von Erklärungen anhand des Modells}

\subsection{Exkurs: NUNAV}

\subsection{Ermittlung des Erklärungsbedarfs}

\cite{golledge1999wayfinding}

\cite{bovy2012route}

\cite{kohl_explainability_2019} gives a good overview to the requirement analysis for Explainability as an NFR

Formulierung der Anforderungen nach \cite{rajnish2010quality, wiegers1999writing, alexander2002writing} formuliert.

\subsection{Anforderungen der Erklärungen}

\subsection{Design der Erklärungen}

\subsubsection{Kollaboratives Routing}
\label{sec:user_count_definition}

\subsubsection{Einflüsse auf die Routenberechnung}
\label{sec:route_explanation_definition}

\subsubsection{Verkehrsaufkommen}
\label{sec:traffic_volume_definition}

\subsubsection{GPS-Qualität}
\label{sec:gps_accuracy_definition}

Auch wenn besser \cite{riveiro_thats_2021}, haben wir uns gegen Interaktionen entschieden, da es während der Navigation nicht gut ist

Wie in \cite{chazette_end-users_nodate} und \cite{wang_integration_2020} vorgeschlagen, wollten wir wenn Interaktion (außershalb der Navigation) möglich, dafür sorgen, dass auf Erklärungen zugegriffen werden kann.

which explanandum X must be explained \cite{kohl_explainability_2019}

Grafiken von eingehendem Datenstrom und ausgehenden Datenstrom.

idea: Frage am Ende der Route verändern: Es hat sich herausgestellt, dass die User diese nicht lesen.

\subsubsection{Umsetzung}