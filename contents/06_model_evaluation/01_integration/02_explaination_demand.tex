\subsection{Ermittlung des Erklärungsbedarfs}

Die Anwendung des Leitfadens zur Integration von Erklärungen ist in mehrere Teile gegliedert. In einem ersten Schritt wurden die existierenden Verständnisprobleme in NUNAV analysiert. Diese wurden aus verschiedenen Datenquellen der Graphmasters GmbH zusammengefasst. Dazu zählen unter anderem mehrere Support-Kanäle.

Um diese Analyse anzureichern wurde darauf aufbauend ein Workshop mit mehreren Mitarbeitern der Graphmasters GmbH durchgeführt. Ergebnis diesen Workshops waren dann eine Aufstellung der Probleme, Ziele und Rohanforderungen für Erklärungen und Ideen zur Umsetzung. Auf Basis dieser Ergebnisse wurden dann im Rahmen dieser Arbeit die Anforderungen konkretisiert und in Designs für Erklärungen umgesetzt. Abschließend ist die technische Realisierung in \textit{NUNAV Navigation} erfolgt.

\subsubsection{User-Feedback-Analyse}

Für die Initiale Analyse, welche Verständnisprobleme in \textit{NUNAV Navigation} bestehen, wurde zunächst manuell das Feedback, welches im Google Play Store und Apple App Store von Nutzern in Form von App-Reviews gegeben wurde, analysiert (ab Oktober 2020). Daraus sind Themengebiete abgeleitet worden, und die Anzahl der Reviews, die ein Themengebiet ansprechen, wurden gezählt. Insgesamt konnten aus 46 Reviews klare Themen abgeleitet werden, wovon 33 Reviews Verständnisprobleme erwähnen. Die Themen, die häufig vorkamen und für Erklärbarkeit relevant sind, sind in \nameref{ch:appendix_1} aufgelistet. Dabei wurde als Hauptproblem ein zu geringes Verständnis für den kollaborativen Routingalgorithmus identifiziert. Allein fünf der 16 Reviews, aus denen sich dies ableiten lässt, bemängeln, dass NUNAV keine Alternativrouten vorschlägt. Dies weist darauf hin, dass nicht klar ist, dass das System des Verteilens der Nutzer auf die vorhandene Verkehrsinfrastruktur nur möglich ist, wenn die Nutzer den vorgeschlagenen Routen folgen und keine Alternativen nehmen.

Außerdem gibt es Hilfe-Artikel, die über einen \textit{Hilfe-Center} der Graphmasters-Webseite erreichbar sind\footnote{\url{https://support.graphmasters.net/}, besucht: 01.10.21}. Diese behandeln zum Teil schon die in \nameref{ch:appendix_1} erwähnten Themen (z.B. kollaboratives Routing). Dabei wurden die Klickzahlen mit den Review-Anzahlen verglichen, welche vom her Verhältnis ähnlich der Anzahlen an Reviews ist.

Aus den Reviews und Hilfe-Center-Artikeln wurden dann erste Themen beziehungsweise Nutzerfragen abgeleitet und mit dem Team \glqq Solution Experts\grqq{} von Graphmasters diskutiert. Das Team ist für die Betreuung von Kunden und die Bearbeitung von Feedback zuständig. Die Ergebnisse sind in \autoref{tab:06_model_evaluation_explicit_questions} zu sehen.

\begin{table}[htb!]
    \begin{tabular}{p{.3\textwidth}p{.66\textwidth}}
        \hline
        Thema         & Fragen \\
        \toprule
        Kollaboratives Routing  & Was ist / wie funktioniert kollaboratives Routing? \\
        &  Warum brauche ich eine ständige Internetverbindung? \\
        &  Wieso werden in NUNAV keine Standardrouten angezeigt?\\
        \tablerowspacing
        Navigation              & Warum wird diese Route / der Umweg genommen? \\
        & Warum sind die Routen bei NUNAV länger? \\
        & Warum ist die Position ungenau? \\
        \tablerowspacing
        Vertrauen in das System & Woher kommen die Verkehrs-/ Sperrungsdaten? \\
        & Was passiert mit meinen Daten? \\
        & Wofür benötige ich Ortungsdienste beim App-Start? \\
        & Kann die Güte der Route nicht bewerten? \\
        \tablerowspacing
        Funktionen   & Was zeigt die Rainbow-Route an? \\
        & Wie kann ich eine Sperrung melden? \\
        & Wie kann ich mein Ziel in NUNAV eingeben? \\
        & Wie kann ich Favoriten anlegen? \\
        \toprule
    \end{tabular}
\caption{Anzahl der Reviews im Google Play Store und Apple App Store mit mehrfach vorkommenden Themen}
\label{tab:06_model_evaluation_explicit_questions}
\end{table}

\bigskip

Auf Basis dieser ersten Ergebnisse wurde im Anschluss ein Workshop mit mehreren Graphmasters-Mitarbeitern durchgeführt.

\subsubsection{Workshop zur Integration von Erklärungen}

Zum Festlegen der Ziele und Anforderungen an die Erklärungen sowie zum Sammeln erster Umsetzungsideen, wurde ein interdisziplinärer Workshop mit Frontend-Entwicklern, \glqq Solution Experts\grqq{} und einem Produktmanager bei Graphmasters durchgeführt. Außerdem sollten Metriken gefunden werden, welche zur Überprüfung der Ziele verwendet werden können. Durch die direkte Anwendung des entwickelten Leitfadens im Workshop konnten außerdem Beobachtungen darüber, wie dieser eingesetzt wird, gesammelt werden. Dies war allerdings nicht Kern des Workshops. 

Nach der initialen Vorstellung von Erklärbarkeit, dem Leitfaden für die Integration von Erklärungen sowie den Ergebnissen aus der User-Feedback-Analyse hat sich der Workshop an dem Aufbau des Modells für Erklärungen orientiert. Folglich wurden zuerst Informationen zu möglichen Nutzern (\textit{Context}), dann Ziele (\textit{Objectives}) und deren Umsetzungsmöglichkeiten (\textit{Characteristics}) und schließlich die Evaluationsmöglichkeiten besprochen (\textit{Evaluation}). Auch wenn der Workshop so geplant war, dass die einzelnen Aspekte nacheinander abgearbeitet werden, hat sich ergeben, dass die Ziele und deren Umsetzung gleichzeitig behandelt wurden. Grund dafür war, dass beim Aufstellen der Ziele bzw. Rohanforderungen bereits die Umsetzbarkeit im Rahmen dieser Arbeit diskutiert wurde. Einige Fragen der Machbarkeit sind allerdings ungeklärt geblieben, da hierfür Wissen von den Entwicklern des Routingalgorithmus benötigt wurde. Explizit ging es um die Frage, welche Daten für bestimmte Erklärungen überhaupt zur Verfügung gestellt werden können. Ein detailliertes Ergebnisprotokoll des Workshops ist in \nameref{ch:appendix_1} zu finden.

Aus den Ergebnissen des Workshops wurden dann iterativ und in Rücksprache mit verschiedenen Teilnehmern des Workshops Personas, die konkreten Anforderungen sowie die finalen Erklärungen, die in \textit{NUNAV Navigation} integriert werden sollten, entwickelt.