\subsection{Ermittlung des Erklärungsbedarfs}

Die Anwendung des Leitfadens zur Integration von Erklärungen ist in mehrere Teile gegliedert. Die Graphmasters GmbH hat mehrere Datenquellen, in denen Nutzer-Feedback gesammelt wird. Daher war es in einem ersten Schritt möglich über die verschiedenen Support-Kanäle eine Zusammenfassung der möglichen Verständnisprobleme in \textit{NUNAV Navigation} zu erstellen. Um diese Analyse anzureichern wurde darauf aufbauend ein Workshop mit mehreren Mitarbeitern der Graphmasters GmbH durchgeführt. Ergebnis diesen Workshops waren dann eine Aufstellung der Probleme, Ziele für Erklärungen und Ideen zur Umsetzung. Auf Basis dieser Ergebnisse wurden dann im Rahmen dieser Arbeit die Anforderungen konkretisiert und in Designs für Erklärungen umgesetzt.

\subsubsection{User-Feedback-Analyse}

Für die Initiale Analyse, welche Verständnis-Probleme in \textit{NUNAV Navigation} bestehen, wurde zunächst manuell das Feedback, welches im Google Play Store und Apple App Store von Nutzern in Form von App-Reviews gegeben wurde, analysiert (Ab Oktober 2020). Daraus wurden Themengebiete abgeleitet, und die Anzahl der Reviews, die ein Themengebiet ansprechen, gezählt. Insgesamt konnten aus 46 Reviews klare Themen abgeleitet werden. In 33 Reviews werden Verständnisprobleme erwähnt. Welche Themen häufig vorkamen und für Erklärbarkeit relevant sind, ist in \autoref{tab:06_model_evaluation_integration_app_store_reviews_overview} zu sehen. Dabei sieht man, dass ein zu geringes Verständnis für den Routingalgorithmus mit Schwarmintelligenz besteht. Allein fünf der 16 Reviews, aus denen sich dies ableiten lässt, bemängeln, dass NUNAV keine Alternativrouten vorschlägt. Dies weist darauf hin, dass nicht klar ist, dass das System des Verteilens der Nutzer auf die vorhandene Verkehrsinfrastruktur nur möglich ist, wenn die Nutzer den vorgeschlagenen Routen folgen und keine Alternativen nehmen.

\begin{table}[bht!]
    \begin{tabular}{p{.25\textwidth}p{.56\textwidth}p{.1\textwidth}}
        \hline
        Thema & Beschreibung        & Anzahl \\
        \toprule
        Kollaboratives Routing      & Das Review deutet auf ein fehlendes Verständnis, was der Grundgedanke des
                                        kollaborativen Routings ist, hin. & 16 \\
        \tablerowspacing
        Schlechte Route             & Das Review enthält eine Beschwerde über eine bestimmte Routenführung. & 9 \\
        \tablerowspacing
        GPS-Verständnis             & Das Review deutet auf ein fehlendes Verständnis für schlechten GPS-Empfang hin. & 3 \\
        \tablerowspacing
        Offline-Modus-Verständnis   & Das Review deutet auf ein fehlendes Verständnis der Offline-Karten-Funktion 
                                        hin. & 3 \\
        \tablerowspacing
        Verständnis Routenfarbe     & Im Review werden Nachfragen zur Einfärbung der Route gestellt. & 2 \\
        \toprule
    \end{tabular}
\caption{Anzahl der Reviews im Google Play Store und Apple App Store mit mehrfach vorkommenden Themen}
\label{tab:06_model_evaluation_integration_app_store_reviews_overview}
\end{table}

Außerdem gibt es Hilfe-Artikel, die über einen \textit{Hilfe-Center} der Graphmasters-Webseite erreichbar sind\footnote{\url{https://support.graphmasters.net/}, besucht: 01.10.21}. Diese decken zum Teil schon die in \autoref{tab:06_model_evaluation_integration_app_store_reviews_overview} erwähnten Themen ab (z.B. kollaboratives Routing). Dabei wurden die Klickzahlen mit den Review-Anzahlen verglichen, welche vom her Verhältnis ähnlich der Anzahlen an Reviews ist.

% \cite{golledge1999wayfinding} \cite{bovy2012route} schreiben, welche Verständnisprobleme es bei Navigationsanwendungen im Allgemeinen gibt.

Aus den Reviews und Hilfe-Center-Artikeln wurden dann erste Themen beziehungsweise Nutzerfragen abgeleitet und mit dem Team \glqq Solution Experts\grqq{} von Graphmasters diskutiert. Das Team ist für die Betreuung von Kunden und die Bearbeitung von Feedback zuständig. Die Ergebnisse sind in \autoref{tab:06_model_evaluation_explicit_questions} zu sehen:

\begin{table}[bht!]
    \begin{tabular}{p{.3\textwidth}p{.66\textwidth}}
        \hline
        Thema         & Fragen \\
        \toprule
        Kollaboratives Routing  & Was ist / wie funktioniert kollaboratives Routing? \\
        &  Warum brauche ich eine ständige Internetverbindung? \\
        &  Wieso werden in NUNAV keine Standardrouten angezeigt?\\
        \tablerowspacing
        Navigation              & Warum wird diese Route / der Umweg genommen? \\
        & Warum sind die Routen bei NUNAV länger? \\
        & Warum ist die Position ungenau? \\
        \tablerowspacing
        Vertrauen in das System & Woher kommen die Verkehrs-/ Sperungs-Daten? \\
        & Was passiert mit meinen Daten? \\
        & Wofür benötige ich Ortungsdienste beim App-Start? \\
        & Kann die Güte der Route nicht bewerten? \\
        \tablerowspacing
        Funktionen   & Was zeigt die Rainbow-Route an? \\
        & Wie kann ich eine Sperrung melden? \\
        & Wie kann ich mein Ziel in NUNAV eingeben? \\
        & Wie kann ich Favoriten anlegen? \\
        \toprule
    \end{tabular}
\caption{Anzahl der Reviews im Google Play Store und Apple App Store mit mehrfach vorkommenden Themen}
\label{tab:06_model_evaluation_explicit_questions}
\end{table}

% \begin{itemize}
%     \item Kollaboratives Routing
%         \begin{itemize}
%             \item Was ist / wie funktioniert kollaboratives Routing?
%             \item Warum brauche ich eine ständige Internetverbindung?
%             \item Wieso werden in NUNAV keine Standardrouten angezeigt?
%         \end{itemize}
%     \item Navigation
%         \begin{itemize}
%             \item Warum wird diese Route / der Umweg genommen?
%             \item Warum sind die Routen bei NUNAV länger?
%             \item Warum ist die Position ungenau?
%         \end{itemize}
%     \item Vertrauen in das System
%         \begin{itemize}
%             \item Woher kommen die Verkehrs-/ Sperungs-Daten?
%             \item Was passiert mit meinen Daten?
%             \item Wofür benötige ich Ortungsdienste beim App-Start?
%             \item Kann die Güte der Route nicht bewerten?
%         \end{itemize}
%     \item Funktionen
%         \begin{itemize}
%             \item Was zeigt die Rainbow-Route an?
%             \item Wie kann ich eine Sperrung melden?
%             \item Wie kann ich mein Ziel in NUNAV eingeben?
%             \item Wie kann ich Favoriten anlegen?
%         \end{itemize}
% \end{itemize}

Außerdem gab es an die \glqq Solution Experts\grqq{} weitere weniger häufig auftretende Fragen bzw. Verständnisprobleme von \textit{NUNAV Navigation}, die in dieser Arbeit aufgrund der Einzelfälle nicht aufgeführt sind.

\bigskip

Auf Basis dieser ersten Ergebnisse wurde im Anschluss ein Workshop mit mehreren Graphmasters-Mitarbeitern durchgeführt.

\subsubsection{Workshop zur Integration von Erklärungen}

% Display context informaiton (Einfach zu integrieren) \cite{wiegand_id_2020}

Zur konkreten Anwendung des Leitfadens für Erklärungen und dabei insbesondere das Klären der externen Abhängigkeiten für zu integrierende Erklärungen und erste Ideen wurde ein interdisziplinärer Workshop mit Frontend-Entwicklern, \glqq Solution Experts\grqq{} und einem Produktmanager bei Graphmasters durchgeführt.

Ziel des Workshops war es Rohanforderungen bzw. Ziele sowie Ideen für die Umsetzung von Erklärungen zusammen zu entwickeln. Auch sollten Metriken gefunden werden, welche zur Überprüfung der Ziele verwendet werden können.

Nach der initialen Vorstellung des Leitfadens für die Integration von Erklärungen sowie den Ergebnissen aus der Uesr-Feedback-Analyse hat sich der Workshop an dem Aufbau des Modells für Erklärungen orientiert. Folglich wurden zuerst Informationen zu möglichen Nutzern, dann Ziele und deren Umsetzungsmöglichkeiten und schließlich die möglichen Metriken besprochen. Dass Umsetzungsmöglichkeiten und Ziele statt nacheinander gleichzeitig bearbeitet wurden, hat sich während des Workshops ergeben, da für die verschiednen Ziele und Erklärungen auch die Machbarkeit für diese Arbeit diskutiert wurde. Einige Fragen der Machbarkeit sind offen geblieben, da hierfür Wissen von den Entwicklern des Routingalgorithmus benötigt wurde. Explizit ging es um die Frage, welche Daten für bestimmte Erklärungen überhaupt zur Verfügung gestellt werden können. Ein detailliertes Ergebnisprotokoll des Workshops ist in \autoref{sec:appendix_workshop_protocol} zu finden.

Aus den Ergebnissen des Workshops wurden dann iterativ und in Rücksprache mit verschiedenen Teilnehmern des Workshops Personas, die konkreten Anforderungen sowie die Erklärungen, die in \textit{NUNAV Navigation} integriert werden sollten, entwickelt. Diese werden im Folgenden vorgestellt.