\subsection{Umsetzung}

Alle zuvor erläuterten und im Rahmen dieser Arbeit entwickelten Erklärungen wurden entweder im \textit{BFF} oder innerhalb der \textit{NUNAV Navigation} Android App umgesetzt. Die Integration ist nur auf Android erfolgt, da die Anzahl der Nutzer auf Android etwa 92\% ausmacht und die Integration in die iOS-App den Aufwand auf Frontend-Seite verdoppelt hätte. Außerdem sind die Apps insgesamt auf verschiedenen Design-Ständen und können daher in einer Evaluation nicht direkt verglichen werden. Die Erklärungen wurden innerhalb des im Mobile-Team bei Graphmasters üblichen Entwicklungsprozesses in die App integriert. Grundsätzlich basiert dieser auf Scrum mit einwöchigen Sprints und \textit{Github-Issues} als Tool zum Tracken von Aufgaben. Für jede abgeschlossene \textit{Issue} erfolgt darüber hinaus ein Review durch ein anderes Team-Mitglied.

Die beschriebenen Erklärungen sind in allen Live-Versionen der Apps potenziell verfügbar. Allerdings wurde in dieser Arbeit ein \textit{Feature-Flag}-System entwickelt, mithilfe dessen es möglich ist über den \textit{BFF} einzelne Features für bestimmte \textit{End User} zu aktivieren oder deaktivieren. Daher sehen nicht alle \textit{End User} die Erklärungen.

Für die Integration der Erklärungen wurden Code-Änderungen am \textit{BFF}, dem \textit{Shared Code} der Mobilplatformen, sowie am nativen Code der Android-App im Rahmen dieser Arbeit vorgenommen. Der Großteil des Codes ist in den Zusatzmaterialien dieser Arbeit einsehbar. Wie zuvor erwähnt wurden zum Teil von anderen Teams bei Graphmasters zuätzliche Daten für die Erklärungen zur Verfügung gestellt. Diese Änderungen sind in dieser Arbeit nicht enthalten.

\newpage