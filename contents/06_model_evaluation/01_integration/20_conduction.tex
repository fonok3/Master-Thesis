\subsection{Umsetzung}

Alle zuvor erläuterten und im Rahmen dieser Arbeit entwickelten Erklärungen wurden entweder im \textit{BFF} oder innerhalt der \textit{NUNAV Navigation} Android App umgesetzt. Die Integration ist nur auf Android erfolgt, da die Anzahl der Nutzer auf Android etwa x\% TODO!!!!!! ausmacht und die Integration in die iOS-App den Aufwand auf Fronted-Seite verdoppelt hätte. Außerdem sind die Apps insgesamt auf verschiedenen Design-Ständen und können daher in einer Evaluation nicht direkt verglichen werden. Die Integration der Erklärungen wurde innerhalb des im Mobile-Team bei Graphmasters üblichen Entwicklungsprozesses in die Apps integriert. Grundsätzlich basiert dieser auf Scrum mit 1-wöchigen Sprints mit \textit{Github-Issues} als Tool zum tracken der Aufgabe. Jede abgeschlossene \textit{Issue} muss darüber hinaus ein Review erfolgen.

Die beschrieben Erklärungen sind prinzipiell in allen Live-Versionen der Apps potentiell verfügbar. Allerdings wurde in dieser Arbeit ein \textit{Feature-Flag}-System entwickelt, mithilfe dessen es möglich ist über den \textit{BFF} einzelne Features für bestimmte \textit{End User} zu aktivieren oder deaktivieren.

Für die Integration der Erklärungen wurden Code-Änderungen am \textit{BFF}, dem \textit{Shared Code} zwischen den Mobilplatformen, sowie am nativen Code der Android-App im Rahmen dieser Arbeit vorgenommen. Der Großteil des Codes ist in den Zusatzmaterialien dieser Arbeit einsehbar. Wie zuvor erwähnt wurden zum Teil von anderen Teams bei Graphmasters zuätzliche Daten für die Erklärungen zur Verfügung gestellt. Diese Änderungen sind in dieser Arbeit nicht enthalten.

\newpage