\subsubsection{Einflüsse auf die Routenberechnung}

\paragraph{[FR2]} \textit{NUNAV Navigation} muss den \textit{End Usern} die Möglichkeit bieten, abzurufen, welche Informationen zu Verkehrsereignissen (z.B. Verkehrsfluss, Sperrungen, Baustellen, Staus) in den Algorithmus einfließen.

\bigskip

Auf der Karte, welche das Hauptinteraktionselement von \textit{NUNAV Navigation} darstellt, sind Sperrungen, Baustellen und Staus bereits eingezeichnet. Wie aus dem Nutzerfeedback hervorgeht, reichen diese \textit{Context}-Informationen für das Verständnis der \textit{End User} nicht aus. Daher wurde ein neuer Hilfe-Center-Artikel im Rahmen dieser Arbeit zu diesem Thema angelegt (siehe \autoref{sec:help_center_routing_data}). Dieser Artikel ist für Nutzer geeignet, welche NUNAV zum ersten Mal wie Ayla oder noch nicht lange wie Michael verwenden. So können sie sich anhand der Daten auf der Karte im Anschluss an das Lesen des Artikels die verschiedenen Routen verständlicher erklären.

Dieser Beitrag wurde ebenfalls vor dem Start der Navigation erreichbar gemacht. Um die Standardkartenansicht nicht durch mehrere Möglichkeiten zum Aufrufen von Hilfe-Center-Artikeln zu überladen, wurde die Möglichkeit, zu dieser Erklärung zu gelangen, in die Routen-Vorschau integriert. Auch für diese wurden mehrere Prototypen erstellt, wobei bei der finalen Entscheidung darauf geachtet wurde, dass die Aufrufmöglichkeit nicht zu viel Platz einnimmt und der Text neugierig macht. Die verschiedenen Prototypen sind in \autoref{fig:prototype_routing_explanation} zu sehen.

\begin{figure}[b!]
    \centering
    \subfloat[1. Prototyp zur Positionierung des Aufrufs der Erklärung]
    {
        \includegraphics[width=.27\textwidth]{contents/06_model_evaluation/01_integration/res/02_routing_algorithm/prototype_1.png}
    }
    \hspace{.055\textwidth}
    \subfloat[Alternativer Prototyp zur Positionierung des Aufrufs der Erklärung]
    {
        \includegraphics[width=.27\textwidth]{contents/06_model_evaluation/01_integration/res/02_routing_algorithm/prototype_2.png}
    }
    \hspace{.055\textwidth}
    \subfloat[Finales Design zur Positionierung des Aufrufs der Erklärung]
    {
        \includegraphics[width=.27\textwidth]{contents/06_model_evaluation/01_integration/res/02_routing_algorithm/final_1.png}
    }
    \caption{Prototyp und finale Designs für den Aufruf der Erklärung zu Einflüssen auf die Routenberechnung}
    \label{fig:prototype_routing_explanation}
\end{figure}

\newpage