\subsection{Studienaufbau}

\subsubsection{Ziel der Studie}

Das Ziel der Studie ist wie auch das Ziel der gesamten Arbeit (Siehe \autoref{sec:goal_definition}) auf Basis der Vorlage von \citeauthor{wohlin2012experimentation} formuliert \cite{wohlin2012experimentation}.

\noindent\fbox{
    \parbox{\textwidth}{
        \smallskip
        \textbf{Analysiere} <Object(s) of study> \textbf{in Bezug auf} <Quality Focus> \textbf{zur} <Purpose> \textbf{aus der Sicht} von NUNAV-Nutzern \textbf{im Kontext} <Context>.
        \smallskip
    }
}

\subsubsection*{Studiengruppen}

\begin{itemize}
    \item \textbf{Gruppe 1}: Nutzer, die keine Erklärungen erhalten
    \item \textbf{Gruppe 2}: Nutzer, die Erklärungen zum Routing-Algorithmus erhalten
    \item \textbf{Gruppe 3}: Nutzer, die Erklärungen basierend auf der aktuellen Navigation erhalten
    \item \textbf{Gruppe 4}: Nutzer, die die Erklärungen von Gruppe 2 und 3 erhalten
\end{itemize}