\subsection{Validität der Evaluation von Erklärungen}

Im Folgenden wird untersucht, inwiefern die Ergebnisse der durchgeführten \textit{Case Study} und des Quasi-Experiments eingeschränkt sein könnten. Für die beiden Evaluation werden die von \citeauthor{wohlin2012experimentation} vorgestellten \textit{Threads to Validity} überprüft \cite{wohlin2012experimentation}.

Die \textit{Conclusion Validity} bezieht sich auf die statistische Analyse der Ergebnisse und die Zusammensetzung der Probanden. Unter \textit{Internal Validity} werden Einflüsse zusammengefasst, die die unabhängige Variable in Bezug auf die Kausalität ohne die Kenntnis der analysierenden Personen beeinflussen können.

In Bezug auf die statistische Analyse der \textit{Case Study} konnte eine große Datenmenge analysiert werden ($\approx$ 4000 Teilnehmer). Außerdem waren die \textit{Teilnehmer} der Studie zufällig auf die verschiedenen Bedingungen verteilt. Folglich wird abgeleitet, dass eine hohe \textit{Conclusion Validity} vorliegt. Die Daten wurden mithilfe von Hypothesentests und einem Signifikanzniveau von 5\% analysiert.

Da bei der \textit{Case-Study} die meisten äußeren Einflüsse nicht überprüfbar oder unbekannt sind, gibt es Einschränkungen für die \textit{Internal Validity}. Allerdings konnten auf Basis von Einflusshypothesen außerhalb der Evaluation von Erklärungen bestimmte Effekte ausgeschlossen oder nicht verwendbare Daten herausgefiltert werden.

\smallskip

Die \textit{Conclusion Validity} des \textit{Quasi-Experiments} ist sehr gering, weder eine zufällige Verteilung der Teilnehmer auf mehrere Bedingungen oder eine verschiedene Zusammensetzung der Bedingungen durchgeführt wurde, noch bei vier Teilnehmern statistisch signifikante Aussagen getroffen werden können.

Für die \textit{Internal Validity} wurden für alle Teilnehmer die gleichen Umstände geschaffen. Das verschiedene Vorwissen zu \textit{NUNAV Navigation} der Probanden kann die Ergebnisse beeinflussen. Da das Vorwissen allerdings ebenfalls gemessen wurde, ist dies in die Interpretation der Ergebnisse mit eingeflossen.

\smallskip

Die zwei verbleibenden \textit{Threads to Validity} werden für die beiden Studien zusammen betrachtet. Unter \textit{External Validity} wird die Verallgemeinerbarkeit von Ergebnissen für andere Kontexte bewertet. Die konkrete Qualität der Erklärungen, die aus den beiden Evaluationen hervorgeht, kann nur für den konkreten Anwendungsfall der Navigation, verallgemeinert werden. Allerdings muss bei der Verallgemeinerung der Erklärungen zum Routing-Algorithmus einschränkend erwähnt werden, dass diese sehr spezifisch für den \textit{NUNAV}-Routing-Algorithmus sind.

\textit{Construct Validity} behandelt die Möglichkeit der Verallgemeinerung von Versuchsergebnissen auf ein Konzept oder die Theorie hinter dem Versuch. Da beide Evaluationen zu dem Zweck durchgeführt wurden, um die Anwendbarkeit des entwickelten Leitfadens zu prüfen, wird die Auswirkung auf den Leitfaden betrachtet. Hier kann anhand der Studien nur die Anwendbarkeit für die \textit{Graphmasters GmbH} eindeutig bewertet werden. 

\smallskip

Zusammenfassend kann gesagt werden, dass die \textit{Conclusion Validity} und \textit{Internal Validity} der beiden Evaluationen gegensätzlich bewertet wurden und somit verschiedene Ergebnisse liefern, welche zusammengefügt werden können. Die Einschränkung der \textit{External Validity} ist für das Ziele dieser Arbeit nicht relevant, da die Ergebnisse der Evaluation nicht auf andere Kontexte übertragen werden soll. Die Bewertung des Leitfadens muss allerdings aufgrund der Limitierung der \textit{Construct Validity} mit Vorsicht betrachtet werden.