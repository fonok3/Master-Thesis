\subsection{Validität der Evaluation von Erklärungen}

Im Folgenden wird untersucht, inwiefern die Ergebnisse der durchgeführten \textit{Case Study} und des Quasi-Experiments eingeschränkt oder gefährdet sein könnten. Für die beiden Evaluation werden jeweils die von \citeauthor{wohlin2012experimentation} vorgestellten \textit{Threads to Validity} überprüft \cite{wohlin2012experimentation}.

Die \textit{Conclusion Validity} bezieht sich auf die statistische Analyse der Ergebnisse und die Zusammensetzung der Probanden. Unter \textit{Internal Validity} werde Einflüsse, die die unabhängige Variable in Bezug auf die Kausalität beeinflussen können, ohne die Kenntnis der analysierenden Personen.

\subsubsection{Case Study}

In Bezug auf die statistische Analyse der \textit{Case Study} konnte eine große Datenmenge analysiert werden. Außerdem waren die \textit{Teilnehmer} der Studie zufällig auf die verschiedenen Bedingungen verteilt. Durch die hohe Zahl der Teilnehmer ($\approx$ 4000) kann außerdem davon ausgegangen werden, dass in jeder Gruppe der Studie verschiedene \textit{End-User}-Typen vertreten sind. Folglich wird abgeleitet, dass eine hohe \textit{Conclusion Validity} vorliegt.

Da bei der \textit{Case-Study} die meisten äußeren Einflüsse nicht überprüfbar oder unbekannt sind, gibt es Einschränkungen für die \textit{Internal Validity}. Allerdings konnten auf Basis von Einflusshypothesen außerhalb der Evaluation von Erklärungen bestimmte Effekte ausgeschlossen oder nicht verwendbare Daten herausgefiltert werden.

\subsubsection{Quasi-Experiment}

Die \textit{Conclusion Validity} des \textit{Experiments} ist sehr gering, weder eine zufällige Verteilung der Teilnehmer auf mehrere Bedingungen oder eine verschiedene Zusammensetzung der Bedingungen durchgeführt wurde, noch bei vier Teilnehmern statistisch signifikante Aussagen getroffen werden können.

Für die \textit{Internal Validity} wurden für alle Teilnehmer die gleiche Umstände geschaffen. Das verschiedene Vorwissen zu \textit{NUNAV Navigation} der Probanden kann allerdings die Ergebnisse beeinflussen. Da dieses allerdings ebenfalls gemessen wurde, ist dies in die Interpretation der Ergebnisse mit eingegangen.

\bigskip

% \textit{Construct Validity} behandelt die Möglichkeit der Verallgemeinerung von Versuchsergebnissen auf ein Konzept oder die Theorie hinter dem Versuch. \textit{External Validity}

% Auch die Ergebnisse, inwiefern sich die entwickelten Erklärungen mit bestimmten Eigenschaften auf Qualitätsaspekte auswirken ist nur für den \textit{Context} einer Navigationsansicht gezeigt worden. Außerdem muss zur Evaluation der Erklärung einschränkend hinzugefügt werden, dass die qualitative Analyse lediglich mit vier \textit{End Usern} im Rahmen eines \textit{Quasi-Experiments} durchgeführt wurde und daher keine statisch signifikanten Ergebnisse liefert, sondern nur als Interpretationshilfe diente.