\subsection{Validität der Evaluation von Erklärungen}

Im Folgenden wird untersucht, inwiefern die Ergebnisse der durchgeführten \textit{Case Study} und des Quasi-Experiments eingeschränkt oder gefährdet sein könnten. Für die beiden Evaluation werden jeweils die von \citeauthor{wohlin2012experimentation} vorgestellten \textit{Threads to Validity} überprüft \cite{wohlin2012experimentation}.

Die \textit{Conclusion Validity} bezieht sich auf die statistische Analyse der Ergebnisse und die Zusammensetzung der Probanden. Unter \textit{Internal Validity} werde Einflüsse, die die unabhängige Variable in Bezug auf die Kausalität beeinflussen können, ohne die Kenntnis der analysierenden Personen.

\subsubsection{Case Study}

\subsubsection{Quasi-Experiment}

% Auch die Ergebnisse, inwiefern sich die entwickelten Erklärungen mit bestimmten Eigenschaften auf Qualitätsaspekte auswirken ist nur für den \textit{Context} einer Navigationsansicht gezeigt worden. Außerdem muss zur Evaluation der Erklärung einschränkend hinzugefügt werden, dass die qualitative Analyse lediglich mit vier \textit{End Usern} im Rahmen eines \textit{Quasi-Experiments} durchgeführt wurde und daher keine statisch signifikanten Ergebnisse liefert, sondern nur als Interpretationshilfe diente.