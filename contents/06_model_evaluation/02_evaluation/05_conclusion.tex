\subsection{Zusammenfassung der Evaluation}

Anhand einer \textit{Case Study} innerhalb der Produktivversion von \textit{NUNAV Navigation} wurden zunächst die Qualitätsanforderungen von \textit{Graphmasters} auf die Einflüsse der integrierten Erklärungen untersucht. Diese Prüfung hat ergeben, dass die Erklärung zum kollaborativen Routing sowie zu den Einflüssen auf den Routing-Algorithmus keinen signifikanten Effekt auf die Metriken der aufgestellten Anforderungen aufweisen (siehe \autoref{sec:explanation_requirements}).

Als positives Ergebnis der \textit{Case Study} kann hervorgehoben werden, dass die Erklärungen zum aktuellen Verkehrsgeschehen und zu Positionsungenauigkeiten während der Navigation einen positiven Einfluss auf die \textit{Route Satisfaction} der \textit{End User} haben. Des Weiteren konnte gezeigt werden, dass \textit{End User}, die diesen Erklärungstyp statt der statischen Erklärungen erhalten, \textit{NUNAV Navigation} im Durchschnitt häufiger pro Woche zur Navigation verwenden.

Außerdem zeigt die Analyse der \textit{Case Study}, dass das Bereitstellen aller Erklärungstypen einen signifikant positiven Effekt auf die \textit{Route Acceptance} hat. Im Vergleich zum Bereitstellen der \textit{Context}-abhängigen Erklärungen ist allerdings ein signifikant negativer Einfluss auf die \text{Route Satisfaction} messbar. Für letzteres konnte in keiner der beiden durchgeführten Studien ein Grund ermittelt werden.

Aufgrund der mangelnden Ergebnisse zu den Eigenschaften der entwickelten Erklärungen wurde im Anschluss ein Quasi-Experiment durchgeführt, welches zusammen mit weiteren Metadaten aus der \textit{Case Study} für die beiden statischen Erklärungen potenzielle Verbesserungen herausgestellt hat.

Für die \textit{Context}-abhängigen Erklärungen konnten keine konkreten Vorschläge oder Ergebnisse aus dem Quasi-Experiment gefolgert werden. Dies hat allerdings die Qualität der Erklärungen, welche aus der \textit{Case Study} gefolgert wurde, bestätigt.

Als Konsequenz sollten die in \autoref{sec:demand_qualitative_evaluation} erfolgten Vorschläge der Teilnehmer des \textit{Quasi-Experiments} in eine zweite Iteration der Erklärungen integriert werden. Durch eine weitere Evaluation kann im Anschluss festgestellt werden, ob diese wie vermutet, zu einer erhöhten Qualität der evaluierten Qualitätsaspekte führt.

