\subsection{Ziel der Evaluation}

Das Ziel der Evaluation ist wie auch das Ziel der gesamten Arbeit (siehe \autoref{sec:goal_definition}) auf Basis der Vorlage von \citeauthor{wohlin2012experimentation} formuliert \cite{wohlin2012experimentation}.

\smallskip

\noindent\fbox{
    \parbox{0.964\textwidth}{
        \smallskip
        \textbf{Analysiere} die integrierten Erklärungen \textbf{zur} Evaluation \textbf{in Bezug auf} die externe Qualität des Systems \textbf{aus der Sicht} von \textit{End Usern} \textbf{im Kontext} der Verwendung von \textit{NUNAV Navigation}.
        \smallskip
    }
}

\smallskip

Grundsätzlich sollte folglich im Rahmen der Analyse überprüft werden, ob die zuvor definierten Qualitätsanforderungen, welch aus den Zielen von Graphmasters abgeleitet wurden durch die entwickelten Erklärungen erfüllt werden können.

\subsubsection{Studienaufbau}

Ein Ergebnis des durchgeführten Workshops war, dass die Erklärungen nach Möglichkeit unter Realbedingungen evaluiert werden. Daraus ist die Anforderung abgeleitet worden, dass eine Studie die \textit{End User} von \textit{NUNAV Navigation} so wenig wie möglich bemerken sollten. Daher kommen lediglich Verhaltensmetriken oder bereits in der Anwendung integrierte Metriken zum Einsatz.

Dabei kommen die Metriken zum Einsatz, welche in \autoref{sec:explanation_requirements} zur Definition der Anforderungen definiert wurden. Diese sind bereits vor dieser Arbeit in \textit{NUNAV Navigation} integriert gewesen.

Neben diesen Metriken wurden allerdings aufgrund von Einfluss-Hypothesen durch Störgrößen neben den Erklärungen zusätzliche Metriken in die Anwendung integriert. Ziel dessen war es, auf der Basis möglicher Einflüsse Datensätze herausfiltern zu können, welche die Ergebnisse der Analyse der Einflüsse durch integrierte Erklärungen zu verzerren.

Insgesamt wurde folglich eine zwei-wöchige Case Study als empirische Strategie gewählt. Um eine die Unterschiede im Bezug auf die externe Qualität messen zu können, wurden nicht allen Teilnehmern die Erklärungen angezeigt. Außerdem sollte nicht nur eine Veränderung durch die Erklärungen insgesamt messbar gemacht werden, sondern auch zwischen den einzelnen entwickelten Erklärungen unterschieden werden. Trotz dessen sollte auch die Kombination der verschiedenen Erklärungen überprüft werden. Um jedoch für jede Bedingung genug Nutzer in als Datengrundlage zu haben, wurden nicht alle verschiedenen Kombinationsmöglichkeiten der Erklärungen überprüft.

Grundsätzlich wurden die Erklärungen in die zwei Gruppen \textit{statische} und \textit{Context}-Abhängige Erklärungen gegliedert. Unter den statischen Erklärungen sind die beiden Erklärungen, welche den Routingalgorithmus und die Einflüsse auf die Routenberechnung erklären, zusammengefasst (siehe \autoref{sec:user_count_definition} und \autoref{sec:route_explanation_definition}).

Ebenfalls sind die \textit{Context}-Erklärungen zum aktuellen Verkehrsaufkommen und zu Positionsungenauigkeiten während der Navigation als eine Bedingung für die Studie zusammengefasst worden.

Insgesamt sind folglich vier verschiedene Studiengruppen entstanden:

\begin{itemize}
    \item \textbf{Gruppe 1}: Nutzer, die keine Erklärungen erhalten
    \item \textbf{Gruppe 2}: Nutzer, die statische Erklärungen erhalten
    \item \textbf{Gruppe 3}: Nutzer, die \textit{Context}-abhängige Erklärungen erhalten
    \item \textbf{Gruppe 4}: Nutzer, die statische als auch \textit{Context}-abhängige Erklärungen erhalten
\end{itemize}

\subsection{Studiendurchführung}