\chapter{Modellevaluation / Modellanwendung}

\section{Integration von Erklärungen anhand des Modells}

\subsection{Ermittlung des Erklärungsbedarfs}

\cite{golledge1999wayfinding}

\cite{bovy2012route}

\cite{kohl_explainability_2019} gives a good overview to the requirement analysis for Explainability as an NFR

Formulierung der Anforderungen nach \cite{rajnish2010quality, wiegers1999writing, alexander2002writing} formuliert.

\begin{table}[]
    \centering
    \begin{tabular}{|p{0.25\textwidth}|p{0.23\textwidth}|p{0.5\textwidth}|}
        \hline
        \textbf{Ebene} & \textbf{Qualitätsaspekt} & \textbf{Anforderungen} \\
        \hline
        \multirow{2}{*}{Business Goal}      & System Acceptance & Der Nutzer soll NUNAV als gute Alternative zu anderen Anbietern akzeptieren.\\
        \cline{2-3}
                                            & Usage Increase & Mehr Nutzer sollen NUNAV häufiger nutzen.\\
        \hline
        \multirow{3}{*}{Users' Perception} & Satisfaction & Der Nutzer soll NUNAV als gute Alternative zu anderen Anbietern akzeptieren.\\
                                            \cline{2-3}
                                            & Trust & Mehr Nutzer sollen NUNAV häufiger nutzen.\\
                                            \cline{2-3}
                                            & Understandability & Mehr Nutzer sollen NUNAV häufiger nutzen.\\
        \hline
        \multirow{2}{*}{Explanation Purpose} & Transparency & Der Nutzer soll NUNAV als gute Alternative zu anderen Anbietern akzeptieren.\\
        \cline{2-3}
                                            & Persuasiveness & Mehr Nutzer sollen NUNAV häufiger nutzen.\\
        \hline
    \end{tabular}
    \caption{Caption}
    \label{tab:my_label}
\end{table}

\subsection{Design der Erklärungen}


Auch wenn besser \cite{riveiro_thats_2021}, haben wir uns gegen Interaktionen entschieden, da es während der Navigation nicht gut ist

Wie in \cite{chazette_end-users_nodate} und \cite{wang_integration_2020} vorgeschlagen, wollten wir wenn Interaktion (außershalb der Navigation) möglich, dafür sorgen, dass auf Erklärungen zugegriffen werden kann.

which explanandum X must be explained \cite{kohl_explainability_2019}

Grafiken von eingehendem Datenstrom und ausgehenden Datenstrom.

idea: Frage am Ende der Route verändern: Es hat sich herausgestellt, dass die User diese nicht lesen.

\section{Evaluation der Erklärungen}

\subsection{Studienaufbau}

\subsection{Studienablauf}

\subsection{Ergebnisse}