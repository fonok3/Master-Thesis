\chapter{Fazit und Ausblick}

\section{Fazit}
% RQ 1 - 4 durchgehen und die ERgebnisse noch mal kürzer und knackiger zusamenfassen

% \section{Einordnung der Ergebnisse}

% \subsection{Beantwortung der Forschungsfragen}

% In dieser Arbeit 

% Mit einem Modell für die Aspekte von Erklärungen, den Zusammenhängen und Einflüssen auf ausgewählte externe Qualitätsaspekte sowie Heuristiken zur Gestaltung von Erklärungen ist zusammenfassend ein Leitfaden zur Unterstützung von Erklärungen

% Aus Studie nochmals klar geworden, dass nur mit Verhaltensmetriken, sobald diese nicht eindeutig sind, und direkt die \glqq richtige\grqq{} Erklörung getroffen wurde, die Erklärungen nicht analysiert werden können.

Final hat der Leitfaden bei allen wichtige Schritten der Integration von Erklärungen (Anforderungserhebung, Umsetzung und Evaluation) geholfen. Darüber hinaus wurden unter Anwendung des Leitfadens positive Einflüsse auf die zuvor gesetzten Ziele erzielt worden. 

\section{Ausblick}

% Our long-term vision is to establish a standardized certification process in tandem with appropriate development techniques to achieve explainability by design. This paper is a starting point towards an overarching and systematic approach to explainability requirements. Unknown source


Es fehlt: A framework for integrating explanatory capabilities in the whole software development life-cycle, from requirements elicitation over design and implementation through to its use \cite{cassens_ambient_2019}

need studys for obtrusiveness \cite{lim_2009_assessing}

Conclusion: Es fehlen noch Artefakte für... z.B. aus den im Modell gesammelten Möglichkeiten der Evaluation muss noch ein konkretes Framework gebaut werden, wie es von \citeauthor{sokol_explainability_2020} gefordert wird.

Es fehlt ein klarer Explainability SIG wie auch für Transparency \cite{do2010software} und Invisibility \cite{carvalho2020developers}. m Entwickler einfacher supporten zu können, sollte also vor allem für die Objectives eine bessere Übersicht geschaffen werden.
Außerdem muss mehr evaluiert werden und die Vollständigkeit des Modells sollte durch mindestens eine weitere unabhngige Arbeit bestätigt werden.

Die gefahren für Usability etc. sollten besser herausgearbeitet werden -> Es wurden vor allem positiv Ergebnisse betrachtet, da in der Literatur, höufig nur die Gefahre diskutiert werden, allerdings nicht behandelt werden.

Technologietransfer, wie bei Gorschek et al.

% Es fehlt ein ausgereifter Katalog, für Zusammenhänge bei Erklärungen