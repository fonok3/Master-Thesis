\chapter{Fazit und Ausblick}

\section{Fazit}

Ziel dieser Arbeit war es, ein Modell zur Unterstützung des Designs von Erklärungen in erklärbaren Systemen zu konzipieren und im Anschluss zu evaluieren. Als Ergebnis dieser Arbeit ist unter anderem einen ein Modell zur Unterstützung der Integration von Erklärungen entstanden, welches in einen Leitfaden integriert ist. Der Leitfaden enthält darüber hinaus einen Katalog über bestehende und verallgemeinerbare Zusammenhänge zwischen den äußeren Abhängigkeiten für Erklärungen, den Eigenschaften und Einflüssen auf die Softwarequalität. Abschließend werden im Leitfaden außerdem einige wichtige Heuristiken für das Design von Erklärungen zusammengefasst. Dieser ist im Rahmen einer Literaturrecherche entstanden, welche die bestehenden Ergebnisse für das Design von Erklärungen in erklärbaren Systemen analysiert hat. Der im Leitfaden für die Integration von Erklärungen enthaltene Modell beinhaltet dabei die folgenden Teile:

\paragraph{RQ1} Unter \textit{External Dependencies} sind alle Rahmenbedingungen zusammengefasst, die einen direkten Einfluss auf die Anforderungen an Erklärungen aufweisen. Als relevante Aspekte sind verschiedene Ausprägungen des \textit{Contexts} von erklärbaren Systemen, sowie \textit{Objectives} auf verschiedenen Abstraktionsebenen für die Integration von Erklärungen in dem Modell enthalten. Somit unterstützt dieser Modellteil die Anforderungserhebung für Erklärungen.

\paragraph{RQ2} Die Umsetzung der Anforderungen wird in dem vorgestellten Modell durch in der Forschung evaluierte \textit{Characteristics} unterstützt. Dabei enthält das Modell Eigenschaften von Erklärungen für die Umsetzung des Bedarfs für Erklärungen (\textit{Demand}), die transportierten Informationen \textit{Content} und die Art der Informationsvermittlung an \textit{End User} (\textit{Presentation}). Die im Modell vorgestellten Ausprägungen geben dabei viele Möglichkeiten, um Erklärungen für verschiedene Kontexte zu gestalten. Aufgeführt sind lediglich Eigenschaften, für die ein Effekt auf die Qualität von Softwaresystemen bereits gezeigt werden konnte.

\paragraph{RQ3} Über die Entwicklung von Erklärungen hinaus bietet das Modell des weiteren Hilfestellungen für die Evaluation von Erklärungen in einem System. Grundsätzlich werden die im \textit{Software Engineering} üblichen Studienformen vorgeschlagen, welche je nach Ziel einer Evaluation Anwendung finden können \cite[vgl.][]{wohlin2012experimentation}. Außerdem wird zwischen der direkten Messung der Qualität von Erklärungen sowie der Messung der Einflüsse von integrierten Erklärungen auf externe Qualitätsaspekte eines Systems.

Um einen ganzheitlichen Überblick über die Qualität von Erklärungen zu erhalten, empfiehlt der Leitfaden, in den das Modell integriert ist, auf Basis existierender Literatur eine Kombination der verschiedenen Evaluationsmöglichkeiten. Einerseits sollten sowohl die Erklärungen sowohl direkt als auch deren Einflüsse auf andere Qualitätsaspekte evaluiert werden. Andererseits sind für Bewertung der Qualität sowohl quantitative als auch qualitative Metriken notwendig, um sowohl die performanz der \textit{End User} als auch deren subjektive Wahrnehmung zu betrachten.

\smallskip

\noindent\fbox{
    \parbox{0.964\textwidth}{
        \smallskip 
        \textbf{RQ4.2} Welchen Einfluss hat die Granularität von Erklärungen auf die externe Qualität eines erklärbaren Systems unter bestimmten Rahmenbedingungen?
        \smallskip
    }
}

\smallskip

Die Einflüsse, die Eigenschaften der Granularität von Erklärungen auf einige externe Qualitätsaspekte haben, werden im zweiten Teil des Katalogs der Zusammenhänge der Aspekte von Erklärungen, zusammengestellt. Ebenfalls wird somit eine Antwort auf die letzte Forschungsfrage gegeben und erfüllt die Anforderung GR4, welche fordert, dass der Leitfaden eine Unterstützung bei der Auswahl von Eigenschaften bei der Umsetzung von Erklärungen geben soll.

% RQ 1 - 4 durchgehen und die ERgebnisse noch mal kürzer und knackiger zusamenfassen

% \section{Einordnung der Ergebnisse}

% \subsection{Beantwortung der Forschungsfragen}

% In dieser Arbeit 

% Mit einem Modell für die Aspekte von Erklärungen, den Zusammenhängen und Einflüssen auf ausgewählte externe Qualitätsaspekte sowie Heuristiken zur Gestaltung von Erklärungen ist zusammenfassend ein Leitfaden zur Unterstützung von Erklärungen

% Aus Studie nochmals klar geworden, dass nur mit Verhaltensmetriken, sobald diese nicht eindeutig sind, und direkt die \glqq richtige\grqq{} Erklörung getroffen wurde, die Erklärungen nicht analysiert werden können.

Final hat der Leitfaden bei allen wichtige Schritten der Integration von Erklärungen (Anforderungserhebung, Umsetzung und Evaluation) geholfen. Darüber hinaus wurden unter Anwendung des Leitfadens positive Einflüsse auf die zuvor gesetzten Ziele erzielt worden. 

\section{Ausblick}

% Our long-term vision is to establish a standardized certification process in tandem with appropriate development techniques to achieve explainability by design. This paper is a starting point towards an overarching and systematic approach to explainability requirements. Unknown source


Es fehlt: A framework for integrating explanatory capabilities in the whole software development life-cycle, from requirements elicitation over design and implementation through to its use \cite{cassens_ambient_2019}

need studys for obtrusiveness \cite{lim_2009_assessing}

Conclusion: Es fehlen noch Artefakte für... z.B. aus den im Modell gesammelten Möglichkeiten der Evaluation muss noch ein konkretes Framework gebaut werden, wie es von \citeauthor{sokol_explainability_2020} gefordert wird.

Es fehlt ein klarer Explainability SIG wie auch für Transparency \cite{do2010software} und Invisibility \cite{carvalho2020developers}. m Entwickler einfacher supporten zu können, sollte also vor allem für die Objectives eine bessere Übersicht geschaffen werden.
Außerdem muss mehr evaluiert werden und die Vollständigkeit des Modells sollte durch mindestens eine weitere unabhngige Arbeit bestätigt werden.

Die gefahren für Usability etc. sollten besser herausgearbeitet werden -> Es wurden vor allem positiv Ergebnisse betrachtet, da in der Literatur, höufig nur die Gefahre diskutiert werden, allerdings nicht behandelt werden.

Technologietransfer, wie bei Gorschek et al.

% Es fehlt ein ausgereifter Katalog, für Zusammenhänge bei Erklärungen