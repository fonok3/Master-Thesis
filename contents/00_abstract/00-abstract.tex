\chapter*{Zusammenfassung}

\subsection*{Konzeption und Evaluation eines Modells zur Unterstützung des Designs von Erklärungen in erklärbaren Systemen}

Kurze Zusammenfassung der Arbeit in ca. 200 Wörtern

\clearpage

\chapter*{Abstract}

\subsection*{Conception and evaluation of a model to support the design of explanations in explainable systems}

With the growing complexity of software systems and their profound integration into users' daily lives, the need for transparent, traceable, and trustworthy awakens. A significant impact of explainability, as a non-functional requirement (NFR), on these quality attributes as well as the overall quality of software systems has already been shown.

However, because it is a relatively new NFR, artifacts such as guidelines or models do not yet exist to assist professionals in identifying and operationalizing requirements related to explainability. Therefore, it is important that these artifacts are in place to facilitate the requirements engineering process for explainability and its implementation.

An overview of the various possibilities for the integration of explanations into software systems may support the design of explanations in explainable systems. For this purpose, I conducted a literature review to identify the external dependencies, characteristics, and evaluation methods of explanations. Finally, an exemplary use in practice served to evaluate the applicability of a developed guideline, which lead to promising results concerning the evaluated quality of the resulting explanations and subjective usefulness of the guideline. 

This thesis presents a guideline containing proposals for the development of explanations, together with a catalog of correlations between characteristics of explanations and software quality aspects, as well as heuristics for explanation design. As a future contribution, the guidelines have to be evaluated in more contexts and another iteration of the developed explanations based on the evaluation results should be developed.

\clearpage