\pagebreak

\section{Ergebnisse}
%Paper Typen, die am Ende herauskamen:

%Die Paper können in Zwei Kategorien eingeteilt werden: Paper, die 

\begin{table}
    \begin{center}
        \begin{tabular}{|p{.22\textwidth}p{.3\textwidth}p{.3\textwidth}|}
            \hline
             & \textbf{Empirische Strategie} & \textbf{Quellen} \\ \hline
             \textbf{Evaluation} & & \\
            Qualitätsaspekte                                                                &
            Kontrolliertes Experiment                                                       &
                 \cite{tintarev_designing_nodate} \cite{sato_context_nodate} \cite{eiband_impact_2019} \cite{tsai_evaluating_2019} \cite{hernandez-bocanegra_effects_2020} \cite{balog_measuring_2020} \cite{kunkel_let_2019} \cite{schaffer_i_2019} \cite{weitz_you_2019} \cite{yamada_evaluating_2016} \cite{sato_action-triggering_2019} \cite{haspiel_explanations_2018} \cite{zahedi_towards_2019} \cite{zolotas_towards_2019} \cite{riveiro_thats_2021}  \cite{martin_evaluating_2021} \cite{tsai_effects_2020}    \cite{neerincx_using_2018} \cite{schrills_color_2020} \cite{wang_is_2018} \cite{zhu_effects_2020} \cite{koo_why_2015} \cite{koo_understanding_2016} \cite{cheng2019explaining}
                 \\
                                                                                            &
            Case Study                                                                      &
                 \cite{martin_developing_2019} \cite{ehsan_human-centered_2020}
                 \\
                                                                                            &
            Umfrage                                                                         &
                \cite{chazette_end-users_nodate} \cite{chazette2020explainability} \cite{sokol_one_2020}
            \\
            Nutzer-Präferenz                                                                &
            Kontrolliertes Experiment                                                       &
                \cite{kouki_user_2017} \cite{mucha_interfaces_2021} \cite{abdulrahman_belief-based_2019} \cite{waa_evaluating_2021} \cite{wiegand_id_2020} \cite{stange_effects_2021} \cite{kaptein_personalised_2017} \cite{wiegand2019drive} \cite{du2019look}
            \\ \hline
            \textbf{Analyse} &  & \\
            Überblick                                                                       &
            Literaturrecherche                                                              &
                \cite{chazette_knowledge_nodate} \cite{sokol_explainability_2020} \cite{tintarev2015explaining} \cite{kohl_explainability_2019} \cite{rosenfeld_explainability_2019} \cite{cassens_ambient_2019} \cite{cirqueira_scenario-based_2020} \cite{rjoob_towards_2021} \cite{thomson_knowledge--information_2020} \cite{chari_explanation_2020} \cite{nunes_systematic_2017} \cite{sovrano_modelling_2020} \cite{ribera2019can} \cite{gunning2019darpa} \cite{doshi2017towards} \cite{lim_2009_assessing} \cite{tintarev2007survey}
                \\
                                                                                            &
            Umfrage                                                                         &
                \cite{brennen_what_2020} 
            \\ \hline
        \end{tabular}
    \end{center}
    \caption{Ergebnisse der Literaturrecherche nach Art der Publikation}
    \label{tab:results_paper_types}
\end{table}

\begin{table}
    \begin{center}
        \begin{tabular}{|p{.35\textwidth}|p{.6\textwidth}|}
            \hline
            \textbf{Kontext}                    & \textbf{Quellen} \\ \hline
            Allgemein                           &
                \cite{chazette_end-users_nodate} \cite{chazette2020explainability} \cite{chazette_knowledge_nodate} \cite{eiband_impact_2019} \cite{kohl_explainability_2019} \cite{ribera2019can} \cite{lim_2009_assessing} \\
            \hline
            Intelligente Systeme (z.B. XAI)      & 
                \cite{waa_evaluating_2021} \cite{mucha_interfaces_2021} \cite{sokol_explainability_2020}  \cite{abdulrahman_belief-based_2019} \cite{brennen_what_2020} \cite{schaffer_i_2019} \cite{weitz_you_2019} \cite{riveiro_thats_2021} \cite{martin_developing_2019} \cite{martin_evaluating_2021} \cite{rosenfeld_explainability_2019} \cite{cassens_ambient_2019} \cite{cirqueira_scenario-based_2020}  \cite{ehsan_human-centered_2020} \cite{rjoob_towards_2021} \cite{thomson_knowledge--information_2020} \cite{chari_explanation_2020} \cite{sokol_one_2020}  \cite{neerincx_using_2018} \cite{schrills_color_2020} \cite{sovrano_modelling_2020} \cite{gunning2019darpa} \cite{doshi2017towards} \cite{cheng2019explaining}
            \\ \hline
            Empfehlungssysteme                  & 
                \cite{tintarev_designing_nodate} \cite{sato_context_nodate} \cite{balog_measuring_2020}  \cite{kouki_user_2017} \cite{tsai_evaluating_2019} \cite{hernandez-bocanegra_effects_2020} \cite{kunkel_let_2019} \cite{tintarev2015explaining} \cite{sato_action-triggering_2019} \cite{tsai_effects_2020} \cite{nunes_systematic_2017} \cite{tintarev2007survey}
            \\ \hline
            Autonomes Fahren                    &
                \cite{wiegand_id_2020} \cite{haspiel_explanations_2018} \cite{koo_understanding_2016} \cite{koo_why_2015} \cite{wiegand2019drive} \cite{du2019look}
            \\ \hline
            Mensch-Roboter-Interaktion          &
                \cite{stange_effects_2021} \cite{kaptein_personalised_2017} \cite{zolotas_towards_2019} \cite{wang_is_2018} \cite{zhu_effects_2020}
            \\ \hline
            Domain-Specific                     &
                \cite{yamada_evaluating_2016} \cite{zahedi_towards_2019}
            \\ \hline
        \end{tabular}
    \end{center}
    \caption{Kontext innerhalb von Erklärbaren Systemen, der von den Arbeiten untersucht wurde}
    \label{tab:paer_explanation_contexts}
\end{table}

