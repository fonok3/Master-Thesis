\newpage

\section{Ergebnisse}

Die Ergebnisse der Literaturrecherche lassen sich zunächst in verschiedene Kontexte einordnen (Genaue Zuordnung siehe \autoref{sec:appendix_search_filter}). Dabei werden die Bereiche Intelligente Systeme (z.B. XAI), Empfehlungssysteme, Autonomes Fahren sowie Mensch-Roboter Interaktion abgedeckt. Außerdem betrachten einige Arbeiten Erklärbarkeit im Allgemeinen. Darüber hinaus gibt es zwei Arbeiten die jeweils eine sehr Domänen- bzw. Anwendungsspezifische Evaluation durchführen.

Ein wichtiges Selektionskriterium war, dass die Arbeiten entweder eine Evaluation von Erklärungen enthalten oder Möglichkeiten der Evaluation zusammenfassen. Daher ist als zweites wichtiges Merkmal der Ergebnisse die empirische Strategie zu betrachten (Siehe \autoref{tab:results_paper_types}). Dabei ist als erstes Unterscheidungsmerkmal zu nennen, ob eine Evaluation durchgeführt oder Ergebnisse vorheriger Evaluationen zusammengefasst wurden.

Bei Evaluationen setzt die Literatur vorwiegend auf kontrollierte Experimente, sowie vereinzelt auf Case Studies und Umfragen. Bei der Analyse bestehender Ergebnisse werden von den resultierenden Arbeiten entweder Literaturrecherchen oder Umfragen eingesetzt.

Auf Basis der finalen Suchergebnisse konnten aus den Arbeiten die Antworten auf die Forschungsfragen als zunächst als Rohantworten extrahiert werden. Diese Datengrundlage dient für den Aufbau des Leitfadens für die Integration von Erklärungen, welcher im folgenden Kapitel vorgestellt wird.

\begin{table}[hbt!]
    \begin{center}
        \begin{tabular}{p{.3\textwidth}p{.3\textwidth}p{.31\textwidth}}
            \hline
             & Empirische Strategie & Quellen \\
             \toprule
             Evaluation &  & \\
            Qualitätsaspekte                                                                &
            Kontrolliertes Experiment                                                       &
                 \cite{tintarev_designing_nodate} \cite{sato_context_nodate} \cite{eiband_impact_2019} \cite{tsai_evaluating_2019} \cite{hernandez-bocanegra_effects_2020} \cite{balog_measuring_2020} \cite{kunkel_let_2019} \cite{schaffer_i_2019} \cite{weitz_you_2019} \cite{yamada_evaluating_2016} \cite{sato_action-triggering_2019} \cite{haspiel_explanations_2018} \cite{zahedi_towards_2019} \cite{zolotas_towards_2019} \cite{riveiro_thats_2021}  \cite{martin_evaluating_2021} \cite{tsai_effects_2020}    \cite{neerincx_using_2018} \cite{schrills_color_2020} \cite{wang_is_2018} \cite{zhu_effects_2020} \cite{koo_why_2015} \cite{koo_understanding_2016} \cite{cheng2019explaining}
                 \\
                &
                Umfrage                                                                         &
                \cite{chazette_end-users_nodate} \cite{chazette2020explainability} \cite{sokol_one_2020}
                \\
                & Case Study                                                                      &
                 \cite{martin_developing_2019} \cite{ehsan_human-centered_2020}
                 \\
            
            Nutzer-Präferenz                                                                &
            Kontrolliertes Experiment                                                       &
                \cite{kouki_user_2017} \cite{mucha_interfaces_2021} \cite{abdulrahman_belief-based_2019} \cite{waa_evaluating_2021} \cite{wiegand_id_2020} \cite{stange_effects_2021} \cite{kaptein_personalised_2017} \cite{wiegand2019drive} \cite{du2019look}
            \\
            \midrule
            Analyse &  & \\
            Überblick                                                                       &
            Literaturrecherche                                                              &
                \cite{chazette_knowledge_nodate} \cite{sokol_explainability_2020} \cite{tintarev2015explaining} \cite{kohl_explainability_2019} \cite{rosenfeld_explainability_2019} \cite{cassens_ambient_2019} \cite{cirqueira_scenario-based_2020} \cite{rjoob_towards_2021} \cite{thomson_knowledge--information_2020} \cite{chari_explanation_2020} \cite{nunes_systematic_2017} \cite{sovrano_modelling_2020} \cite{ribera2019can} \cite{gunning2019darpa} \cite{doshi2017towards} \cite{lim_2009_assessing} \cite{tintarev2007survey}
                \\
                                                                                            &
            Umfrage                                                                         &
                \cite{brennen_what_2020} 
            \\ \toprule
        \end{tabular}
    \end{center}
    \caption{Ergebnisse der Literaturrecherche nach Art der Publikation}
    \label{tab:results_paper_types}
\end{table}
