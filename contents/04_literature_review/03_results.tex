\newpage

\section{Ergebnisse}

Die Ergebnisse der Literaturrecherche lassen sich zunächst in verschiedene Kontexte einordnen (siehe \autoref{fig:appendix_literature_research_results}). Dabei werden die Bereiche Intelligente Systeme (z.B. XAI), Empfehlungssysteme, Autonomes Fahren sowie Mensch-Roboter Interaktion abgedeckt. Außerdem betrachten einige Arbeiten Erklärbarkeit im Allgemeinen. Darüber hinaus gibt es zwei Arbeiten die jeweils eine sehr Domänen- bzw. Anwendungsspezifische Evaluation durchführen.

Ein wichtiges Selektionskriterium war, dass die Arbeiten entweder eine Evaluation von Erklärungen enthalten oder Möglichkeiten der Evaluation zusammenfassen. Daher ist als zweites wichtiges Merkmal der Ergebnisse die empirische Strategie zu betrachten. \autoref{tab:results_paper_types} enthält eine Übersicht über die verschiedenen Ziele der Arbeiten sowie verwendeten empirischen Strategien.

Bei Evaluationen werden in der Literatur vorwiegend kontrollierte Experimente, sowie vereinzelt Case Studies und Umfragen verwendet. Bei der Analyse bestehender Ergebnisse werden in den resultierenden Arbeiten entweder Literaturrecherchen oder Umfragen eingesetzt.

Auf Basis der finalen Suchergebnisse konnten aus den Arbeiten die Antworten auf die Forschungsfragen als zunächst als Rohantworten extrahiert werden. Diese Datengrundlage dient für den Aufbau des Leitfadens für die Integration von Erklärungen, welcher im folgenden Kapitel vorgestellt wird.

\begin{table}[hbt!]
    \begin{center}
        \begin{tabular}{p{.35\textwidth}p{.25\textwidth}p{.31\textwidth}}
            \hline
             Ziel der Arbeit & Empirische Strategie & Quellen \\
             \toprule
            Evaluation des Einflusses von Erklärungen auf bestimmte Qualitätsaspekte                                                                &
            Kontrolliertes Experiment                                                       &
                 \cite{tintarev_designing_nodate} \cite{sato_context_nodate} \cite{eiband_impact_2019} \cite{tsai_evaluating_2019} \cite{hernandez-bocanegra_effects_2020} \cite{balog_measuring_2020} \cite{kunkel_let_2019} \cite{schaffer_i_2019} \cite{weitz_you_2019} \cite{yamada_evaluating_2016} \cite{sato_action-triggering_2019} \cite{haspiel_explanations_2018} \cite{zahedi_towards_2019} \cite{zolotas_towards_2019} \cite{riveiro_thats_2021}  \cite{martin_evaluating_2021} \cite{tsai_effects_2020}    \cite{neerincx_using_2018} \cite{schrills_color_2020} \cite{wang_is_2018} \cite{zhu_effects_2020} \cite{koo_why_2015} \cite{koo_understanding_2016} \cite{cheng2019explaining}
                 \\ \tablerowspacing
                &
                Umfrage                                                                         &
                \cite{chazette_end-users_nodate} \cite{chazette2020explainability} \cite{sokol_one_2020}
                \\ \tablerowspacing
                & Case Study                                                                      &
                 \cite{martin_developing_2019} \cite{ehsan_human-centered_2020}
                 \\
            \midrule
            Evaluation von Nutzer-Präferenzen für verschiedene Erklärungen                                                                &
            Kontrolliertes Experiment                                                       &
                \cite{kouki_user_2017} \cite{mucha_interfaces_2021} \cite{abdulrahman_belief-based_2019} \cite{waa_evaluating_2021} \cite{wiegand_id_2020} \cite{stange_effects_2021} \cite{kaptein_personalised_2017} \cite{wiegand2019drive} \cite{du2019look}
            \\
            \midrule
            Überblick über bestehende Ergebnisse                                                                       &
            Literaturrecherche                                                              &
                \cite{chazette_knowledge_nodate} \cite{sokol_explainability_2020} \cite{tintarev2015explaining} \cite{kohl_explainability_2019} \cite{rosenfeld_explainability_2019} \cite{cassens_ambient_2019} \cite{cirqueira_scenario-based_2020} \cite{rjoob_towards_2021} \cite{thomson_knowledge--information_2020} \cite{chari_explanation_2020} \cite{nunes_systematic_2017} \cite{sovrano_modelling_2020} \cite{ribera2019can} \cite{gunning2019darpa} \cite{doshi2017towards} \cite{lim_2009_assessing} \cite{tintarev2007survey}
                \\ \tablerowspacing
                                                                                            &
            Umfrage                                                                         &
                \cite{brennen_what_2020} 
            \\ \toprule
        \end{tabular}
    \end{center}
    \caption{Ergebnisse der Literaturrecherche nach Art der Publikation}
    \label{tab:results_paper_types}
\end{table}
