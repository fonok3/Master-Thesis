\chapter{Allgemeingültigkeit der Ergebnisse}

Wohlin et al. s THreads to validity \cite{wohlin2012experimentation} used by \cite{ghazi2016exploratory} and \cite{carvalho2020developers}

Alle Paper haben einen Strukturaspekt genauer untersucht und haben nur sporadisch die möglichen Kombination dieser untersucht. Vor allem wurde der Inhalt der gegebenen Darstellung untersucht (x Paper)

Diese Arbeit gibt einen Überblick über bewiesene Zusammenhänge im Kontext von Erklärungen in Erklärbaren Systemen. Diese sind allgemein anwendbar und erzielen im Regelfall positive Effekte. In einzelnen Fällen kann es allerdigns auch vorkommen, dass negative Effekte erzielt werden, wenn spezielle Anwendungsfälle betroffen sind.

Es fehlt: A framework for integrating explanatory capabilities in the whole software development life-cycle, from requirements elicitation over design and implementation through to its use \cite{cassens_ambient_2019}

need studys for obtrusiveness \cite{lim_2009_assessing}