\subsection{Struktur der Übersicht}

Da einige Aspekte in der Literatur zum Teil zusammen dargestellt wurden, sind diese auch im Modell hierarchisch angeordnet. Außerdem orientiert sich die Struktur der Übersicht über die Aspekte von Erklärbarkeit an dem Prinzip von Qualitätsmodellen wie es \citeauthor{schneider2012abenteuer} beschreibt \cite{schneider2012abenteuer} (Mehr siehe \autoref{sec:basics_quality_models}). Dabei werden abstrakte Ziele immer weiter konkretisiert bis sie schlussendlich mit konkreten Metriken messbar sind. Das zugrunde liegende Modell (\glqq Goal-Driven and Property-Based Definition Approach for Product Metrics\grqq{} \cite{briand1995goal}) von \citeauthor{briand1995goal} definiert außerdem unter anderem die nötigen Abhängigkeiten von äußeren Faktoren, die im Modell dem bereits erwähnten \textit{Context} entsprechen. Da die Metriken der Evaluation zum Teil von den entwickelten Erklärungen abhängen, müssen die Eigenschaften im Gegensatz zu dem erwähnten Modell von \citeauthor{schneider2012abenteuer} allerdings zuerst definiert werden. Anderenfalls wäre es nicht möglich, Metriken zu entwickeln, welche direkt die Eigenschaften der integrierten Erklärungen messen. Metriken, welche der Messung von Auswirkungen von integrierten Erklärungen in einem System dienen, können allerdings bereits zuvor aufgestellt werden. Für die Formalisierung des Messkonzeptes werden laut \citeauthor{briand1995goal} außerdem vorhandene Abstraktionen benötigt. Als Artefakt sollen hier die Ausprägungen der Kategorie \textit{Evaluation} in \autoref{sec:model_evaluation} dienen.

\smallbreak

Aus Vorherigem folgend werden die Punkte \textit{Context} und \textit{Objective} aus \autoref{tab:model_explaination_aspects} daher unter \textit{External Dependencies} zusammengefasst. Dies verdeutlicht, dass die \textit{Objectives} für das Integrieren von Erklärungen stark mit anderen äußeren Einflüssen (\textit{Context}) zusammenhängen (\autoref{sec:model_external_dependencies}).

Auch werden \textit{Demand}, \textit{Content} und \textit{Presentation} vereint dargestellt, da sich diese drei Kategorien direkt auf die Eigenschaften von Erklärungen beziehen: \textit{Characteristics}. Damit wird der starke Zusammenhang zwischen den verschiedenen Merkmalen einer Erklärung verdeutlicht.

\smallbreak

Folglich ist die Übersicht in die drei Kategorien \textit{External Dependencies}, \textit{Characteristics} und \textit{Evaluation} gegliedert, welche aufgrund der vorherigen Argumentation in dieser Reihenfolge in der Übersicht dargestellt sind. Ein Überblick ist in \autoref{fig:model_overview} zu sehen.

In den folgenden Abschnitten werden die Ausprägungen der einzelnen Kategorien sowie deren Anwendung in der Literatur beschrieben. Eine Übersicht ist in \autoref{fig:model_overview_complete} zu finden. Diese ist grafisch an der Taxonomie für Erklärungen von \citeauthor{nunes_systematic_2017} angelehnt und enthält auch einige Aspekte \cite{nunes_systematic_2017}. Die erwähnte Taxonomie ist allerdings nur auf den Einsatz von Erklärungen in Empfehlungssystemen bezogen und beschränkt sich daher auf bestimmte Darstellungstypen. Außerdem kann sie nicht ohne Weiteres in andere Kontexte übertragen werden. Der Aspekt der Evaluation fehlt darüber hinaus. Er wird allerdings nicht nur von den Autoren selbst, sondern auch von weiteren als wichtig erachtet \cite{cirqueira_scenario-based_2020, martin_evaluating_2021}.

\begin{figure}[htb!]
    \begin{center}
        \includegraphics[width=0.9\linewidth]{contents/05_model_description/res/model-overview.pdf}
    \end{center}
    \caption{Oberkategorien der Aspekte von Erklärungen}
    \label{fig:model_overview}
\end{figure}