\subsection{Übersicht}

In diesem Abschnitt werden die zuvor vorgestellten Aspekte definiert.

\subsubsection{External Dependencies}

Unter \textit{External Dependencies} sind die Aspekte zusammengefasst, die eine Auswirkung auf Erklärungen in einem System haben. Außerdem können von den hier aufgeführten Punkten Anforderungen abgeleitet und zusammen mit Metriken Hypothesen aufgestellt werden. Daraus kann dann auch abgeleitet werden, welche Funktionen des Systems einer Erklärung bedürfen \cite{kohl_explainability_2019}.

\paragraph{Context} Der \textit{Context} einer Erklärung wird durch die Situation gegeben, welche durch die Interaktion eines Nutzers, seiner Aufgabe, dem System und der Umgebung entsteht. (vgl. \cite{chazette_knowledge_nodate, kohl_explainability_2019}).

\paragraph{Objectives} Unter \textit{Objectives} werden die Ziele verstanden, welche eine Erklärung spezifisch erreichen soll und aufgrund derer Erklärungen in ein System integriert werden.

\subsubsection{Characteristics}

\textit{Characteristics} fassen die unmittelbaren Eigenschaften von Erklärungen zusammen. Dies sind die Möglichkeiten, die dem Ersteller dieser zur Verfügung stehen, um die zuvor aufgestellten Anforderungen zu erfüllen. Dies lassen sich in die folgenden drei Entscheidungsmöglichkeiten gliedern.

\paragraph{Demand}

\paragraph{Content}

\paragraph{Presentation}

\subsubsection{Evaluation}

\paragraph{Evaluation}

\smallbreak

In den folgenden Abschnitten werden die genannten Kategorien näher beschrieben sowie Beispiele für die Ausprägung der Merkmale gegeben, die in der Literatur bereits untersucht wurden.