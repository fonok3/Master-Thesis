\subsection{Design implications}

\cite{carvalho2020developers} Nutzt einen Katalog, der die Genauen Zusammenhänge darstellt.

kontext ist sehr wichtig! \cite{sato_context_nodate}

Es sollte darauf geachtet werden, dass vor allem die wahrgenommene Performanz des Systems erhöht wird \cite{riveiro_thats_2021}

should get explanation if possible \cite{wiegand_id_2020}

Indication of system confidence \cite{wiegand_id_2020, golledge1999wayfinding}

Display context informaiton \cite{wiegand_id_2020}

\cite{weitz_you_2019} proposes to use user triggered explanations

– Low-level explanations methods allow the user to visualise key information that provide insight to system decision-making and support interpretation. \cite{martin_evaluating_2021}

– High-level explanation methods augment one or more low-level explanations with contextual information to enable more comprehensive explanation. \cite{martin_evaluating_2021}

- Beim Design einer Erklärung muss genau darauf geachtet werden, welche Kontextinformationen den Nutzer wirklich interessieren. (Beispielsweise im Kontext von AI welche Features) \cite{rjoob_towards_2021}

Umso einfacher und kürzer eine Erklärung ist, umso früher kann sie präsentiert werden. \cite{hleg2019policy, sovrano_modelling_2020}

\glqq Not only should the developers consider the quality, form, and granularity of the explanations, but also the dynamic learning process of the users \grqq{} \cite{wang_integration_2020}
