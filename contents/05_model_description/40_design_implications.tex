\section{Design Auswirkungen}
\label{sec:model_design_implications}

Dieser Abschnitt des Leitfadens zur Integration von Erklärungen in ein System enthält konkrete Empfehlungen für die Wahl der Eigenschaften von Erklärungen. Design meint dabei nicht unbedingt ein visuelles Design, sondern die generelle Umsetzung.

\subsection{10 Heuristiken für Erklärungsdesign}

\begin{enumerate}
    \item \textbf{Accessibility} Ein erklärbares System sollte \textit{End Usern} zu jeder Zeit die Möglichkeit geben, auf Erklärungen zugreifen zu können. \cite{wiegand2019drive, chazette_end-users_nodate, wiegand_id_2020, weitz_you_2019}. Dies umfasst auch, dass optional noch tiefere Informationen angeboten werden sollten für \textit{End User} mit einem größeren Bedürfnis für Informationen. \cite{martin_evaluating_2021}
    \item \textbf{Context Sensitivity} Erklärungen sollten notwendigerweise den Kontext betrachten \cite{sato_context_nodate, rjoob_towards_2021,chazette_end-users_nodate}. %Beim Design einer Erklärung muss genau darauf geachtet werden, welche Kontextinformationen den Nutzer wirklich interessieren. (Beispielsweise im Kontext von AI welche Features) \cite{rjoob_towards_2021}
    \item \textbf{Key Information Delivery} \cite{martin_evaluating_2021} Key but complete \cite{riveiro_thats_2021}
    \item \textbf{Visibility of the System Confidence} Indication of system confidence \cite{wiegand_id_2020, golledge1999wayfinding}
    \item \textbf{Minimum viable Explanation} \cite{wiegand_id_2020, wiegand2019drive} zeitkritische Aufgabe / Entscheidungen dürfen keine langen Erklärungen enthalten. (3) There is a danger in showing explanations to self-confident users in that situation awareness might be negatively impacted – this can be mitigated by requiring interaction with an agent. \cite{schaffer_i_2019}
    \item \textbf{User Perception} Es sollten wowohl die wahrgenommene als auch die tatsächtlichen Verbesserungen betrachtet werden. \cite{riveiro_thats_2021}
    \item \textbf{User Performance} \cite{riveiro_thats_2021}
    \item \textbf{Earliest viable Time} Umso einfacher und kürzer eine Erklärung ist, umso früher kann sie präsentiert werden. \cite{hleg2019policy, sovrano_modelling_2020}
    \item \textbf{Education} Lernprozess beachten \cite{wang_integration_2020}
    \item \textbf{Explanation Combination} \cite{sato_action-triggering_2019, kunkel_let_2019, sato_action-triggering_2019, schrills_color_2020, lim_2009_assessing}
\end{enumerate}
