\subsection{Eigenschaften}

Hier aufgeführt, wenn die Eigenschaft explizit untersucht wurde und nicht nur implizit in der Erklärung verwendet wurde.

\subsubsection*{Bedarf}

\begin{longtable}{|p{.2\textwidth}|p{.5\textwidth}|p{.2\textwidth}|}
    \hline
    \textbf{Eigenschaft}    & \textbf{Ausprägung}   & \textbf{Quellen} \\ \hline
    Initiative              &  User / Manual        & \cite{chazette_end-users_nodate} \cite{tintarev_designing_nodate} \cite{wiegand_id_2020} \\
                            &  System / Automatic   & \cite{chazette_end-users_nodate} \cite{eiband_impact_2019} \cite{wiegand_id_2020} \cite{schaffer_i_2019} \cite{yamada_evaluating_2016} \\
    \hline
    Occasion                &  Before               & \cite{rosenfeld_explainability_2019} \cite{wiegand_id_2020} \cite{kunkel_let_2019} \cite{koo_why_2015} \cite{haspiel_explanations_2018} \\
                            &  While                & \cite{rosenfeld_explainability_2019} \cite{wiegand_id_2020} \cite{kunkel_let_2019} \\
                            &  After                & \cite{rosenfeld_explainability_2019} \cite{wiegand_id_2020} \cite{kunkel_let_2019} \cite{koo_why_2015} \cite{haspiel_explanations_2018} \cite{wiegand2019drive} \\
    \hline
\caption{Bedarf einer Erklärung}
\label{tab:explanation_demands}
\end{longtable}

\subsubsection*{Inhalt}

Die hier vorgestellten Ziele sollen dabei keine abgeschlossene und vollständige Liste darstellen, sondern dienen im Rahmen des Leitfadens für Erklärbarkeit als Überblick über bereits häufig betrachtete Ziele, für deren Erreichung Erklärungen bereits erfolgreich eingesetzt wurden.

Determine what what information should be conveyed to the user \cite{nunes_systematic_2017}

Granularität

Information type description: Constrastic / differences (Why not), belief-bases (quelle) vs. ...-based

User specified requirements (Personalisation) \cite{tintarev_designing_nodate, sokol_explainability_2020}

Goal-Based: Desired outcome (Why?) \cite{kaptein_personalised_2017, abdulrahman_belief-based_2019}

Belief-Based: Based on which information (How?) \cite{kaptein_personalised_2017, abdulrahman_belief-based_2019}

the different types of content answer the gulfs of execution and evaluation of \cite{norman1988psychology} as described by \cite{ribera2019can}

(Why, Why not -> Why, situation specific / local explanation) (How, what if -> global explanation) (what else -> context) \cite{lim_2009_assessing}

. It encompasses data transparency, which answers what information is needed and who are the stakeholders; process transparency answers how something is performed; and policy transparency answers why an action is performed in the context of transparency. \cite{chazette2020explainability}

\begin{longtable}{|p{.2\textwidth}|p{.5\textwidth}|p{.2\textwidth}|}
    \hline
    \textbf{Aspekt}     & \textbf{Synonym} & \textbf{Quellen} \\ \hline
    Information Type    & Why Mit welchem Ziel & \cite{chazette2020explainability} \cite{abdulrahman_belief-based_2019} \cite{yamada_evaluating_2016} \cite{sato_action-triggering_2019} \cite{zahedi_towards_2019} \cite{zahedi_towards_2019} \cite{zolotas_towards_2019} \cite{cassens_ambient_2019} \cite{thomson_knowledge--information_2020} \cite{chari_explanation_2020} \cite{neerincx_using_2018} \cite{nunes_systematic_2017} (Rationale) \cite{zhu_effects_2020} \cite{ribera2019can} \cite{lim_2009_assessing} \\
                        & What (context) & \cite{chazette2020explainability} \cite{zahedi_towards_2019} \cite{cassens_ambient_2019} \cite{zahedi_towards_2019} \cite{zolotas_towards_2019} \cite{chari_explanation_2020} \cite{nunes_systematic_2017} (Data) \cite{ribera2019can} (data) \\
                        & How welcher Glaube wurde abgeleitet & \cite{chazette2020explainability} \cite{sato_action-triggering_2019} \cite{thomson_knowledge--information_2020} \cite{chari_explanation_2020} \cite{neerincx_using_2018} \cite{lim_2009_assessing} \\
                        & Verhalten & \cite{zhu_effects_2020} \cite{ribera2019can}\\
                        & Kontext &  \\
    \hline
    Information Density & Quantity & \cite{ribera2019can} \cite{kouki_user_2017} \\
                        & Level of justification & \cite{hernandez-bocanegra_effects_2020} \\
                        & Aggregation & \cite{hernandez-bocanegra_effects_2020} \cite{martin_developing_2019} \\
                        & Abstraction Level & \cite{thomson_knowledge--information_2020} \\
    \hline
    Adaptivity          & User specified requirements & \cite{tintarev_designing_nodate} \\
                        & Interactiv  & \cite{abdulrahman_belief-based_2019} \cite{cheng2019explaining} \\
                        & Static & \cite{abdulrahman_belief-based_2019} \\
                        & Personalized & \cite{kaptein_personalised_2017} \cite{cassens_ambient_2019}  \cite{sokol_one_2020} \\
                        & Contextualised & \cite{kaptein_personalised_2017} \cite{cassens_ambient_2019} \\
    \hline
\caption{Inhalt einer Erklärung}
\label{tab:content_of_explanations}
\end{longtable}

Die Anzahl der Paper bestätigt das Ergebnis von \cite{chazette_end-users_nodate}, dass die Reihenfolge der Häufigkeiten mit der die Inhalte untersucht wurden so ist.

Was -> System Verthalten

\subsubsection*{Darstellung}

\begin{longtable}{|p{.2\textwidth}|p{.5\textwidth}|p{.2\textwidth}|}
    \hline
    \textbf{Aspekt}     & \textbf{Synonym} & \textbf{Quellen} \\ \hline
    Presentation Type / Explanatory medium
                        & Textual  & \cite{sokol_explainability_2020} \cite{balog_measuring_2020} \cite{tintarev_designing_nodate} \cite{sato_action-triggering_2019} \cite{eiband_impact_2019} \cite{eiband_impact_2019} \cite{abdulrahman_belief-based_2019} \cite{cassens_ambient_2019} \cite{nunes_systematic_2017} \\
                        & Visualisation (auch Formal und statistic summary) & \cite{sokol_explainability_2020} \cite{sato_action-triggering_2019} \cite{sokol_explainability_2020}  \cite{mucha_interfaces_2021} \cite{abdulrahman_belief-based_2019} \cite{nunes_systematic_2017} \cite{schrills_color_2020} \\
                       
                        & Audio & \cite{wiegand2019drive} \cite{nunes_systematic_2017} \cite{wang_is_2018} \\
    \hline
    Explainer           & Factual / Impersonal & \cite{eiband_impact_2019} \cite{abdulrahman_belief-based_2019} \cite{kunkel_let_2019} \cite{neerincx_using_2018} \\
                        & Personal & \cite{abdulrahman_belief-based_2019} \cite{kunkel_let_2019} \cite{weitz_you_2019} \cite{zahedi_towards_2019} \cite{neerincx_using_2018} \\
    \hline
    Grouping            & Single & \cite{nunes_systematic_2017} \cite{balog_measuring_2020} \cite{sato_action-triggering_2019} \cite{eiband_impact_2019} \cite{abdulrahman_belief-based_2019} \\
                        & Grouped & \cite{nunes_systematic_2017} \cite{balog_measuring_2020} \cite{tintarev_designing_nodate}  \\
                        & Multi-Modal / Hybrid & \cite{sato_action-triggering_2019} \cite{abdulrahman_belief-based_2019} \cite{cassens_ambient_2019} \\
    \hline
\caption{Darstellung einer Erklärung}
\label{tab:presentation_of_explanations}
\end{longtable}