\section{Abhängigkeiten}
\label{sec:model_proved_relations}

Zusätzlich zum Modell über die in der Literatur bereits betrachteten Aspekte von Erklärungen sind im Folgenden die Ergebnisse der Auswirkungen dieser Aspekte auf die externe Qualität von Erklärungen zusammengefasst. Die aufgelisteten Einflüsse sind im Rahmen der Literaturrecherche erarbeitet worden. Die Quellen der Zusammenhänge werden in der folgenden Übersicht jeweils mit angegeben.

Enthalten sind jene Resultate, die entweder bereits bewiesen sind oder die Autoren eine starke Vermutung für eine These haben, die allerdings noch einer Überprüfung bedarf. Noch nicht verifizierte Resultate sind mit (*) gekennzeichnet. Auch sind nicht alle Ergebnisse vorangegangener Arbeiten \textit{Context}-unabhängig gezeigt worden. Somit können die Ergebnisse zum Teil nur mit Einschränkungen auf andere Kontexte übertragen werden. Außerdem sind manche Einflüsse nur in Vergleich zu alternativen Erklärungen gezeigt worden. Auch diese Einschränkungen werden in der Auflistung mit aufgeführt.

Generell ist einerseits das Ziel dieser Zusammenfassung von Abhängigkeiten, den Anwendern dieses Leitfadens eine Unterstützung bei der Ableitung von konkreten Anforderungen an Erklärungen aus den \textit{Explanation Purposes} zu geben. Ebenso soll dieser Überblick über zuvor geleistete Ergebnisse auch bei der Umsetzung der Anforderungen im System unterstützen. Folglich dient dieser Teil des Leitfadens der Beantwortung der Forschungsfragen \textbf{RQ4.1} und \textbf{RQ4.2}, welche explizit die Zusammenhänge zwischen verschiedenen Aspekten des Modells abdecken. Damit gibt der Leitfaden zusätzliche Informationen zu den Einflüssen, die einzelne Eigenschaften von Erklärungen unter bestimmten Rahmenbedingungen auf externe Qualitätsaspekte haben. Je nach \textit{Context} können dabei auch Konflikte zwischen den externen Qualitätsaspekten, auf die Erklärbarkeit Auswirkungen hat, entstehen. Die Erkennung dieser soll durch den folgenden Katalog der Zusammenhänge unterstützt werden.

Wie schon das Modell der Aspekte von Erklärungen hat auch dieser Katalog mit den Abhängigkeiten keinen Anspruch auf Vollständigkeit, sondern soll einen guten Überblick über bestehende Ergebnisse schaffen, die sich auf viele Anwendungsfälle übertragen lassen.

Die Formulierung der Abhängigkeiten orientiert sich an der Formulierungsweise von \citeauthor{carvalho2020developers} bei der generellen Untersuchung verschiedener NFRs \cite{carvalho2020developers}. Das Konzept beschreibt, dass einzelne Eigenschaften von Softwaresystemen einen positiven, negativen oder neutralen Effekt auf Qualitätsaspekte haben können. Bedarf ein Zusammenhang einer Bedingung, so wird diese nachgestellt.

Gegliedert ist der Katalog der Abhängigkeiten in zwei Abschnitte. Ersterer behandelt die Auswirkungen, die die Rahmenbedingungen auf den \textit{Demand} für Erklärungen haben. Im zweiten Teil werden die Einflüsse der Rahmenbedingungen eines Systems auf die Granularität von Erklärungen, also den \textit{Content} und die \textit{Presentation} vorgestellt. Außerdem sind die verschiedenen Einflüsse in positive, negative und neutrale Einflüsse gegliedert.

\subsection*{Demand}

Im Folgenden sind die Einflüsse aufgeführt, die sich darauf beziehen, ob und wann Erklärungen benötigt werden.

\subsubsection*{Positive Zusammenhänge}

\begin{itemize}
    \item Das Präsentieren von Erklärungen hat einen positiven Einfluss auf die \textit{Perceived Transparency}, \textit{wenn End User geringes Vertrauen in Technologie haben} \cite{tsai_effects_2020}. 
    \item Das Präsentieren von Erklärungen hat einen positiven Einfluss auf die \textit{Perceived Transparency}, \textit{wenn End User Datenschutzbedenken haben} \cite{tsai_effects_2020}.
    \item Das Präsentieren von Erklärungen hat einen positiven Einfluss auf \textit{Trust}, \textit{wenn End User sich in unbekannten Situationen befinden} \cite{haspiel_explanations_2018}.
    \item Das Präsentieren einer Erklärung vor einer unbekannten Situation (\textit{before}) hat einen positiveren Einfluss auf \textit{Trust}, als Präsentation danach \cite{haspiel_explanations_2018}.
    \item Das proaktive Präsentieren von Erklärungen (\textit{System-Initiative}) in unbekannten Situationen hat einen positiven Einfluss auf \textit{Trust} \cite{zhu_effects_2020}.
    \item Das Präsentieren von Erklärungen für sicherheitsrelevante Systemfunktionen hat einen positiven Einfluss auf \textit{Trust}  \cite{wiegand2019drive}.
    \item Das Präsentieren von Erklärungen für sicherheitsrelevante Systemfunktionen hat einen positiven Einfluss auf \textit{Satisfaction}  \cite{wiegand2019drive}.
    \item Die Möglichkeit Erklärungen anzufordern \textit{User-initiative} hat einen positiven Einfluss auf \textit{Satisfaction} \cite{chazette_end-users_nodate}.
\end{itemize}

\subsubsection*{Negative Zusammenhänge}

\begin{itemize}
    \item Das Präsentieren von Erklärungen kann einen negativen Einfluss auf die \textit{Effectivity} von \textit{End Usern} haben, \textit{wenn diese sehr selbstbewusst sind}* \cite{schaffer_i_2019}.
    \item Das Präsentieren von Erklärungen kann einen negativen Einfluss auf die \textit{Satisfaction} haben, \textit{wenn End User hohen Trust in das System haben}* \cite{rosenfeld_explainability_2019, doshi2017towards}.
\end{itemize}

\subsubsection*{Weitere Zusammenhänge}

\begin{itemize}
    \item Der Zusammenhang zwischen dem Grad der Systembeeinflussung durch \textit{End User} und dem Erklärungsbedarf ist antiproportional. Umso mehr Beeinflussungsmöglichkeiten \textit{End User} in einem System haben, umso weniger Erklärungsbedarf haben diese \cite{rosenfeld_explainability_2019}.
\end{itemize}

\smallskip

\noindent\fbox{
    \parbox{0.964\textwidth}{
        \smallskip
        \textbf{RQ4.1} Welchen Einfluss hat der Bedarf von Erklärungen auf die externe Qualität eines erklärbaren Systems unter bestimmten Rahmenbedingungen?
        \smallskip
    }
}

\smallskip

Der erste Teil des hier gezeigten Kataloges von Zusammenhängen der Aspekte des Modells für Erklärungen beantwortet die Forschungsfrage \textbf{RQ4.1}. Er stellt dar, unter welchen Bedingungen das Präsentieren von Erklärungen, welche Einflüsse auf ausgewählte externe Softwarequalitätsaspekte hat. Die Frage wird dabei insofern beantwortet, als die Antwort aus der Literaturrecherche entstandene verallgemeinerbare Einflüsse zusammenfasst. 

\subsection*{Granularität}

Auf welche Weise der \textit{Content} und die \textit{Presentation} durch Rahmenbedingungen wie dem \textit{Context} und den \textit{Objectives} für Erklärungen beeinflusst werden, ist im Folgenden dargestellt.

\subsubsection*{Positive Zusammenhänge}

\begin{itemize}
    \item Erklärungen, welche mehrere Medien kombinieren, haben einen größeren positiven Einfluss auf die \textit{Persuasiveness} eines Systems, als das Präsentieren der einzelnen Erklärungen  \cite{sato_action-triggering_2019, kunkel_let_2019, sato_action-triggering_2019, schrills_color_2020, lim_2009_assessing}.
    \item Das Präsentieren von Erklärungen, welche die zugrunde liegenden Daten für ein Systemverhalten darlegen (\textit{Context}), haben einen positiven Einfluss auf die \textit{Persuasiveness} eines Systems* \cite{sato_action-triggering_2019, abdulrahman_belief-based_2019}. Gezeigt ist dies nur für \textit{Recommender Systems}.
    \item Das Präsentieren von Erklärungen, welche die zugrunde liegenden Daten für ein Systemverhalten darlegen (\textit{Context}), haben einen positiven Einfluss auf die \textit{Usefulness} einer Erklärung* \cite{sato_action-triggering_2019, abdulrahman_belief-based_2019}. Gezeigt ist dies nur für \textit{Recommender Systems}.
    \item Das Präsentieren von kausalen Erklärungen, die den Grund angeben, warum eine Alternative nicht genommen wird, hat einen positiveren Einfluss auf die \textit{Perceived Transparency} als das Präsentieren solcher, die den Grund angeben, warum eine Entscheidung getroffen wurde \cite{martin_evaluating_2021, schrills_color_2020, neerincx_using_2018}.
    \item Das Präsentieren von Erklärungen durch Agenten (\textit{Human}) hat einen positiven Einfluss auf den \textit{Trust} in Systemen* \cite{weitz_you_2019}. Gezeigt ist dies nur für \textit{XAI-Systems}.
    \item Das Präsentieren von Erklärungen, welche die zugrunde liegenden Daten für ein Systemverhalten darlegen (\textit{Context}), hat einen positiveren Einfluss auf die \textit{Satisfaction} mit einem System, als das Präsentieren solcher, die die Gründe für eine Entscheidung erläutern, \textit{wenn die End User wenig Domänenwissen auf dem Einsatzgebiet des Systems haben} \cite{kaptein_personalised_2017, martin_evaluating_2021}
    \item Einfache Erklärungen (\textit{Amount / Abstraction Level}) haben einen positiven Einfluss auf die \textit{System Acceptance} \cite{hleg2019policy, sovrano_modelling_2020}.
    \item Einfache Erklärungen (\textit{Amount / Abstraction Level}) haben einen positiven Einfluss auf die \textit{Satisfaction} mit dem System \cite{hleg2019policy, sovrano_modelling_2020}.
    \item Interaktive Erklärungen haben einen positiven Einfluss auf die \textit{Perceived Transparency} eines Systems \cite{cheng2019explaining}.
\end{itemize}

\subsubsection*{Negative Zusammenhänge}

\begin{itemize}
    \item Das Präsentieren von Erklärungen hat einen negativen Einfluss auf den \textit{Trust} in ein System, \textit{wenn der Content der Erklärung nicht dazu führt, dass das Mental Model der End User danach mit der wirklich Funktionsweise des Systems übereinstimmt (Completeness)} \cite{schrills_color_2020, chazette_end-users_nodate}.
    \item Das Präsentieren von Erklärungen hat einen negativen Einfluss auf den \textit{Trust} in ein System, \textit{wenn die Erklärungen zu unseriös sind} \cite{wang_is_2018}.
    \item Das Präsentieren von Erklärungen, welche das aktuelle Systemverhalten beschreiben, kann einen negativen Effekt auf die \textit{Satisfaction} mit dem System haben* \cite{koo_why_2015}.
    \item Das Präsentieren von Erklärungen, welche das aktuelle Systemverhalten beschreiben, kann einen negativen Effekt auf die \textit{Effectivity} haben* \cite{koo_why_2015}.
    \item Das Präsentieren von Erklärungen kann einen so großen positiven Effekt auf \textit{Trust} haben, dass es einen negativen Effekt auf die \textit{Scrutability} und daher einen negativen Effekt auf die \textit{Effectivity} der \textit{End User} hat* \cite{kohl_explainability_2019, gunning2019darpa}.
\end{itemize}

\subsubsection*{Neutrale Zusammenhänge}

\begin{itemize}
    \item Das Präsentieren von kausalen Erklärungen hat keinen Einfluss auf die \textit{Effectivity} von \textit{End Usern}, \textit{wenn diese subjektiv unbekannt mit der Aufgabe sind - unabhängig von der objektiven Kompetenz} \cite{schaffer_i_2019}.
    \item Das Präsentieren von Erklärungen, die die Funktionsweise von Algorithmen im Allgemeinen erklären (\textit{Inner Logic}), haben einen geringen Einfluss auf weitere Qualitätsaspekte \cite{chazette_end-users_nodate}.
    \item Die wirkliche \textit{Transparency} hat keinen Einfluss auf den \textit{Trust} der \textit{End User} in das System, \textit{wenn das Mental Model der End User mit der wirklichen Funktionsweise des Systems übereinstimmt} \cite{eiband_impact_2019, riveiro_thats_2021}.
    \item Die wirkliche \textit{Transparency} von Erklärungen hat keinen Einfluss auf den \textit{Trust} der \textit{End User} in das System, \textit{wenn die End User mit der Erklärung zufrieden sind} \cite{eiband_impact_2019, riveiro_thats_2021}.
\end{itemize}

\smallskip

\noindent\fbox{
    \parbox{0.964\textwidth}{
        \smallskip 
        \textbf{RQ4.2} Welchen Einfluss hat die Granularität von Erklärungen auf die externe Qualität eines erklärbaren Systems unter bestimmten Rahmenbedingungen?
        \smallskip
    }
}

\smallskip

Die Einflüsse, die Eigenschaften der Granularität von Erklärungen auf einige externe Qualitätsaspekte haben, werden im zweiten Teil des Katalogs der Zusammenhänge der Aspekte von Erklärungen, zusammengestellt. Ebenfalls wird somit eine Antwort auf die letzte Forschungsfrage gegeben und erfüllt die Anforderung GR4, welche fordert, dass der Leitfaden eine Unterstützung bei der Auswahl von Eigenschaften bei der Umsetzung von Erklärungen geben soll.

\smallskip

Zusammenfassend sind mit dem Katalog der Zusammenhänge sowie dem zuvor vorgestellten Modell die Forschungsfragen beantwortet worden. Um die Anwendung des Leitfadens weiter zu vereinfachen, die Anforderungen an diesen und damit das Ziel dieser Arbeit besser zu erfüllen, werden im Folgenden konkrete Design-Auswirkungen für Erklärungen vorgestellt.