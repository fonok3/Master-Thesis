\section{Anforderungen an den Leitfaden}

Für die Integration von Komponenten in ein System werden Anforderungen an eben diese Komponenten benötigt. Folglich verweisen \citeauthor{chazette_end-users_nodate, doshi2017towards} darauf, dass es einer Untertützung bedarf, Anforderungen an Erklärungen zu formulieren. Hier gehört unter anderem die Identifikation von Problemen, welche durch Erklärungen gelöst werden können \cite{chazette_end-users_nodate, doshi2017towards}. Unterstützt wird dies durch \citeauthor{waa_evaluating_2021}, welche dies nutzen wollen, um jeder Evaluation von Erklärungen klare Hypothesen voraus zustellen und somit die Ergebnisse verschiedener Arbeiten besser zusammenfassen und daraus Empfehlungen ableiten zu können. Auch \citeauthor{kohl_explainability_2019} unterstriechen den Aspekt, dass vor allem die Ziele und Anforderungen auf Basis des aktuellen Kontextes klar sein müssen, bevor Erklärungen in ein System integriert werden können \cite{kohl_explainability_2019}. Diese in der Literatur gestellten Anforderungen an einen Leitfaden ergeben folglich die Grundlagen, um die gestellte Forschungsfrage nach dem Einfluss der Rahmenbedingungen eines erklärbaren Systems zu beantworten (RQ1, siehe \autoref{sec:goal_definition}).

RQ3 fragt nach den Metriken, durch welche die Qualität von Erklärungen gemessen werden kann (\autoref{sec:goal_definition}). Nach unabhängigen Untersuchungen verschiedener Erklärungstypen stellen mehrere Autoren den Bedarf für eine Vereinheitlichung der Evaluation von Erklärungen fest \cite{cirqueira_scenario-based_2020,zahedi_towards_2019, nunes_systematic_2017, martin_evaluating_2021}. Dies soll es unter anderem ermöglichen, die Vergleichbarkeit zwischen verschiedenen Lösungsansätzen zu gewährleisten. In einigen Arbeiten fordern die Autoren dabei explizit, dass ein Framework zur Evaluation benötigt wird, anhand dessen die Qualität von Erklärungen bestimmt werden kann \cite{nunes_systematic_2017,sokol_explainability_2020,chari_explanation_2020}.

Außerdem ergibt sich aus dem Ziel der Arbeit (\autoref{sec:goal_definition}), dass der Leitfaden auch Gestaltungsempfehlungen für Erklärungen enthalten und somit einen Teil der Forderung von \citeauthor{waa_evaluating_2021} nach einem Überblick über bisherige Ergebnisse erfüllen soll. Basis hierfür sind die Forschungsfragen RQ4.1 und RQ4.2, welche nach den Einflüssen auf verschiedene Eigenschaften von Erklärungen fragen. Voraussetzung dafür ist allerdings auch, dass mögliche Eigenschaften von Erklärungen zuvor definiert werden (RQ2, \autoref{sec:goal_definition}).

\smallskip

Zusammenfassend muss das Modell folgende Aspekte enthalten, um die Anforderungen aus der Literatur zu erfüllen und die Forschungsfragen zu beantworten:

\begin{enumerate}
    \item[MR1] Das Modell muss eine Unterstützung für das Erheben von Anforderungen an Erklärungen und das Aufstellen von Hypothesen über den Einfluss enthalten.
    \item[MR2] Im Modell sollten mögliche Vorschläge zur Umsetzung von Erklärungen gegeben werden.
    \item[MR3] Das Modell muss einen Überblick über verschiedene Evaluationsmöglichkeiten geben.
    \item[MR4] Es müssen existente Ergebnisse aus der Literatur so zusammengefasst sein, dass es möglich ist, zu prüfen, ob entwickelte Erklärungen mit diesen übereinstimmen.
\end{enumerate}

\bigskip

Für die Umsetzung dieser Anforderungen ist der Leitfaden in zwei Teile gegliedert. Ersterer enthält einen Überblick über die verschiedenen Aspekte, die bei der Entwicklung von Erklärungen berücksichtigt werden sollten. Dieser Überblick wird in Form eines Modells gegeben, in dem die verschiedenen Aspekte gegliedert sind. Mithilfe dieser Übersicht soll grundlegend die Möglichkeit geschaffen werden, Anforderungen und Hypothesen aufzustellen, diese umzusetzen und schlussendlich zu evaluieren.

Unterstützt werden soll die Entwicklung von Erklärungen außerdem durch bereits erwiesene Abhängigkeiten zwischen den im ersten Teil des Modells enthaltenen Aspekten sowie daraus abgeleiteten Auswirkungen auf das Design. Daher werden diese Auswirkungen als zweiter Teil des Leitfadens beigefügt.

