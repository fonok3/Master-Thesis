\section{Anforderungen an den Leitfaden}

% \glqq Through clear definitions and motivation, the contribution of the evaluation becomes more apparent. \grqq{} \cite{waa_evaluating_2021}

Für die Integration von Komponenten in ein System werden Anforderungen an ebendiese Komponenten benötigt. Folglich verweisen \citeauthor{chazette_end-users_nodate} darauf, dass es einer Unterstützung bedarf, Anforderungen an Erklärungen zu formulieren. Hierzu gehört unter anderem die Identifikation von Problemen, welche durch Erklärungen gelöst werden können \cite{chazette_end-users_nodate, doshi2017towards}. Unterstützt wird dies durch \citeauthor{waa_evaluating_2021}, welche dies nutzen wollen, um jeder Evaluation von Erklärungen klare Hypothesen voraus zustellen und somit die Ergebnisse verschiedener Arbeiten besser zusammenfassen und daraus Empfehlungen ableiten zu können \cite{waa_evaluating_2021}. Auch \citeauthor{kohl_explainability_2019} unterstreichen den Aspekt, dass vor allem die Ziele und Anforderungen auf Basis des aktuellen Kontextes klar sein müssen, bevor Erklärungen in ein System integriert werden können \cite{kohl_explainability_2019}. Diese in der Literatur gestellten Anforderungen an einen Leitfaden ergeben folglich die Grundlagen, um die gestellte Forschungsfrage nach dem Einfluss der Rahmenbedingungen eines erklärbaren Systems zu beantworten (RQ1, siehe \autoref{sec:goal_definition}).

Außerdem fordern \citeauthor{waa_evaluating_2021}, dass es einem Überblick über bereits entwickelte Erklärungstypen bedarf, um sich bei der Integration von Erklärungen daran orientieren zu können. Zusammen mit RQ2 folgt daraus die Forderung nach einer Zusammenfassung der bestehenden Ergenisse zum Einsatz verschiedener Erklärungstypen.

RQ3 fragt nach den Metriken, durch welche die Qualität von Erklärungen gemessen werden kann (\autoref{sec:goal_definition}). Nach unabhängigen Untersuchungen verschiedener Erklärungstypen stellen mehrere Autoren den Bedarf für eine Vereinheitlichung der Evaluation von Erklärungen fest \cite{cirqueira_scenario-based_2020,zahedi_towards_2019, nunes_systematic_2017, martin_evaluating_2021}. Dies soll es unter anderem ermöglichen, die Vergleichbarkeit unterschiedlicher Lösungsansätzen zu gewährleisten. In einigen Arbeiten fordern die Autoren dabei explizit, dass ein Framework zur Evaluation benötigt wird, anhand dessen die Qualität von Erklärungen bestimmt werden kann \cite{nunes_systematic_2017,sokol_explainability_2020,chari_explanation_2020}.

% A theory of the dialogic process rather than a monologic product (is missing \cite{cassens_ambient_2019}) A framework for evaluation measures that is: – intrinsic (deciding on a strategy for explanation generation) – dialogic (measuring the reaction to an explanation and providing further explanation if needed)

Außerdem ergibt sich aus dem Ziel der Arbeit (\autoref{sec:goal_definition}), dass der Leitfaden auch Gestaltungsempfehlungen für Erklärungen enthalten soll. Auch wird dies von \citeauthor{waa_evaluating_2021} im Rahmen einer Übersicht bestehender Ergebnisse gefordert \cite{waa_evaluating_2021}. Basis hierfür sind die Forschungsfragen RQ4.1 und RQ4.2, welche nach den Einflüssen auf verschiedene Eigenschaften von Erklärungen fragen. Diese Einflüsse sollten so dargestellt werden, dass mögliche konkurrierende Qualitätsaspekte bei der Wahl verschiedener Eigenschaften für Erklärungen klar werden.

\smallskip

Zusammenfassend muss der Leitfaden mit dem Modell folgende Aspekte enthalten, um die Anforderungen aus der Literatur zu erfüllen und die Forschungsfragen zu beantworten (\textit{Guideline Requirements, GR}):

\begin{enumerate}
    \item[GR1] Der Leitfaden muss eine Unterstützung für das Erheben von Anforderungen an Erklärungen und das Aufstellen von Hypothesen über den Einfluss enthalten.
    \item[GR2] Der Leitfaden muss mögliche Vorschläge zur Umsetzung von Erklärungen geben.
    \item[GR3] Der Leitfaden muss einen Überblick über verschiedene Evaluationsmöglichkeiten für Erklärungen im Kontext externer Softwarequalität geben.
    \item[GR4] Der Leitfaden muss in der Literatur gezeigte Einflüsse einzelner Eigenschaften auf die externe Qualität von Erklärungen zusammenfassen.
\end{enumerate}

\bigskip

Für die Umsetzung dieser Anforderungen ist der Leitfaden in zwei Teile gegliedert. Ersterer enthält ein Überblick über die verschiedenen Aspekte, die bei der Entwicklung von Erklärungen berücksichtigt werden sollten. Dieser Überblick wird in Form eines Modells gegeben, in dem die verschiedenen Aspekte gegliedert sind. Mithilfe dieses Modells soll die Möglichkeit geschaffen werden, Anforderungen und Hypothesen aufzustellen, diese umzusetzen und schlussendlich zu evaluieren.

Unterstützt werden soll die Entwicklung von Erklärungen außerdem durch bereits gezeigte Abhängigkeiten zwischen den im ersten Teil des Modells enthaltenen Aspekten sowie daraus abgeleiteten Auswirkungen auf das Design. Daher werden diese Auswirkungen als zweiter Teil des Leitfadens beigefügt.

