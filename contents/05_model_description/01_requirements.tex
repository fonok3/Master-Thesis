\section{Anforderungen}

Nach einer Untersuchung von Erklärungen stellen mehrere Autoren den Bedarf für eine Vereinheitlichung der Untersuchung von Erklärungen fest \cite{cirqueira_scenario-based_2020,zahedi_towards_2019, nunes_systematic_2017, martin_evaluating_2021}. Dies soll es ermöglichen, den Vergleich zwischen verschiedenen Lösungsansätzen zu vereinfachen. In einigen Arbeiten fordern die Autoren, dass ein einheitliches Framework zur Evaluation benötigt wird, anhand dessen die Qualität von Erklärungen bestimmt werden kann \cite{nunes_systematic_2017,sokol_explainability_2020,chari_explanation_2020}. Selbige Anforderung ergibt sich auch durch RQ2 (Siehe \autoref{sec:goal_definition}).

Allerdings wird auch darauf verwiesen, dass es einer Unterstützung bedarf, um Anforderungen an die Erklärungen zu formulieren bzw. die Probleme, welche durch Erklärungen gelöst werden sollen zu identifizieren \cite{chazette_end-users_nodate, doshi2017towards}. Unterstützt wird dies durch \citeauthor{waa_evaluating_2021}, welche dies nutzen wollen, um jeder Prüfung von Erklärungen klare Hypothesen voraus zustellen und somit die Ergebnisse verschiedener Arbeiten besser zusammenfassen und daraus Empfehlungen ableiten zu können. Auch \citeauthor{kohl_explainability_2019} unterstreicht den Aspekt, dass vor allem die Ziele und Anforderungen auf Basis des aktuellen Kontextes klar sein müssen, bevor Erklärungen in ein System integriert werden. Folglich sollte das Modell auch hierfür Unterstützung bieten. Außerdem ergibt sich aus dem Ziel der Arbeit (\autoref{sec:goal_definition}), dass das Modell auch Gestaltungsempfehlungen für Erklärungen enthalten und somit einen Teil der Forderung von \citeauthor{waa_evaluating_2021} nach einem Überblick über bisherige Ergebnisse erfüllen soll. Basis hierfür sind die Forschungsfragen RQ3 und RQ4, welche nach den Auswirkungen verschiedener Eigenschaften von Erklärungen fragen. Voraussetzung dafür ist allerdings auch, dass mögliche Eigenschaften von Erklärungen definiert werden (RQ1).

\smallskip

Zusammenfassend muss das Modell folgende Aspekte enthalten, um die Anforderungen aus der Literatur zu erfüllen und die Forschungsfragen zu beantworten:

\begin{enumerate}
    \item[MR1] Das Modell muss eine Unterstützung für das Erheben von Anforderungen an Erklärungen und das Aufstellen von Hypothesen über den Einfluss enthalten.
    \item[MR2] Das Modell muss einen Überblick über verschiedene Evaluationsmöglichkeiten geben.
    \item[MR3] Im Modell sollten mögliche Vorschläge zur Umsetzung von Erklärungen gegeben werden.
    \item[MR4] Es müssen existente Ergebnisse aus der Literatur so zusammengefasst sein, dass es möglich ist zu prüfen, ob entwickelte Erklärungen mit diesen übereinstimmen.
\end{enumerate}

Für die Umsetzung dieser Anforderungen ist das Modell in zwei Teile gegliedert. Ersterer enthält einen Überblick über die verschiedenen Aspekte, die bei der Entwicklung von Erklärungen berücksichtigt werden sollten. Mithilfe dieser Übersicht soll grundlegend die Möglichkeit geschaffen werden, Anforderungen und Hypothesen aufzustellen, diese Umzusetzen und schlussendlich zu evaluieren.

Unterstützt werden soll die Entwicklung von Erklärungen außerdem durch bereits erwiesene Abhängigkeiten zwischen den im ersten Teil des Modells enthaltenen Aspekten sowie daraus abgeleiteten Auswirkungen auf das Design. Daher werden diese Auswirkungen als zweiter Teil des Modells beigefügt.

