\section{Aspekte von Erklärungen}
\label{sec:model_explanation_aspects}

Literatur, die einen Überblick über Erklärbarkeit im Allgemeinen oder in einem bestimmten Anwendungsfeld gibt, betrachtet in der Regel fünf Aspekte von Erklärbarkeit \cite{rosenfeld_explainability_2019, nunes_systematic_2017,chazette_knowledge_nodate}. Dies sind der Kontext der Erklärung, die Zielsetzung dieser, welche Erklärung angezeigt werden soll und wann diese angezeigt werden soll. In einigen Arbeiten wird bei der angezeigten Erklärung außerdem zwischen dem Inhalt und der Darstellung unterschieden bzw. diese Eigenschaften einzeln untersucht \cite{nunes_systematic_2017,abdulrahman_belief-based_2019}. Außerdem wird in allen hier betrachteten Arbeiten auch die Evaluation von Erklärungen thematisiert (Siehe \autoref{sec:literature_review}).

Für die zuvor genannten Aspekte werden in der Literatur verschiedene Unterkategorien konkret benannt oder Synonyme verwendet. \autoref{tab:model_explaination_aspects} fasst die verwendeten Synonyme aus den Veröffentlichungen, welche diese explizit erwähnt haben, unter der final gewählten Benennung des Aspekts zusammen. Neben den dort aufgezeigten Begriffen haben mehrere Autoren (z.B. \cite{rosenfeld_explainability_2019, chazette2020explainability}) die verschiedenen Aspekte von Erklärbarkeit zusätzlich mit Fragewörtern verknüpft. Die vollständigen Fragen dahinter verweisen allerdings auf verschiedene Unterpunkte, weswegen das vorgestellte Modell auf Fragewörter verzichtet, um Verwechselungen vorzubeugen. (Beispiel: \glqq \textbf{Wie} kann die Erklärung evaluiert werden?\grqq (\textit{Evaluation})\cite[vgl.][]{rosenfeld_explainability_2019} und \glqq \textbf{Wie} viele Informationen sollte jede Erklärung enthalten?\grqq (\textit{Content}) \cite[vgl.][]{kouki_user_2017}). Folgend werden die Aspekte näher erläutert bzw. definiert.

\begin{table}
    \begin{tabular}{|p{.2\textwidth}|p{.4\textwidth}|p{.3\textwidth}|}
        \hline
        \textbf{Aspekt}          & \textbf{Synonyme} & \textbf{Quellen} \\ \hline
        1. Context      & (Experimental) Context & \cite{chazette_knowledge_nodate} \cite{chazette_end-users_nodate} \cite{sato_context_nodate} \cite{waa_evaluating_2021} \cite{kohl_explainability_2019} \cite{neerincx_using_2018} \cite{sovrano_modelling_2020} \cite{doshi2017towards} \\
                        & (Explanation) Scope & \cite{wohlin2012experimentation} \cite{eiband_impact_2019} \cite{doshi2017towards} \\
                        & Use Case & \cite{waa_evaluating_2021} \\
                        & Stakeholder & \cite{rosenfeld_explainability_2019} \\
        \hline
        2. Objectives   & Objectives & \cite{nunes_systematic_2017} \\
                        & Construct & \cite{waa_evaluating_2021} \\
                        & Purpose & \cite{nunes_systematic_2017} \cite{wohlin2012experimentation} \\
                        & (Stakeholder) Goals & \cite{cirqueira_scenario-based_2020} \cite{sovrano_modelling_2020} \cite{ribera2019can} \\
                        & Main Drive & \cite{anjomshoae2019explainable} \\
                        & Intended Effect & \cite{balog_measuring_2020} \\
        \hline
        3. Demand          & Demand            & \cite{chazette_knowledge_nodate} \\
        \hline
        4. Content         & User Interface Component(s) & \cite{nunes_systematic_2017}
                                                        \cite{rosenfeld_explainability_2019} \\
                        & Content               & \cite{ribera2019can} \\
                        & Granularity           & \cite{chazette_knowledge_nodate}
                                                  \cite{kohl_explainability_2019} \\
        \hline
        5. Presentation    & Presentation          & \cite{rosenfeld_explainability_2019,kouki_user_2017} \\
                        & (Explanation) Type    & \cite{ribera2019can} \cite{rosenfeld_explainability_2019} \\
        \hline
        6. Evaluation      & Evaluation            & \cite{kohl_explainability_2019} \cite{doshi2017towards} \\
                        & Measurements          & \cite{waa_evaluating_2021} \cite{balog_measuring_2020} \\
                        & Metrics               & \cite{nunes_systematic_2017} \cite{anjomshoae2019explainable}
                                                  \cite{chari_explanation_2020} \cite{waa_evaluating_2021}\\
        \hline
    \end{tabular}
\caption{Übergeordnete Aspekte von Erklärungen in der Literatur}
\label{tab:model_explaination_aspects}
\end{table}

\paragraph{Context} Der \textit{Context} einer Erklärung wird durch die Situation gegeben, welche durch die Interaktion eines Nutzers, seiner Aufgabe, dem System und der Umgebung entsteht. (vgl. \cite{chazette_knowledge_nodate, kohl_explainability_2019}).

\paragraph{Objectives} Unter \textit{Objectives} werden die Ziele verstanden, welche eine Erklärung spezifisch erreichen soll und solche, aufgrund derer Erklärungen in ein System integriert werden sollen.

\paragraph{Demand} Unter dem Punkt \textit{Demand} muss in der Entwicklung entschieden werden, wann die Nutzer des Systems überhaupt eine Erklärung benötigen. Dies bezieht sich nicht nur auf den Zeitpunkt beziehungsweise den Systemteil, welcher erklärt werden soll, sondern auch ob die Initiative vom Nutzer ausgeht oder das System selbständig eine Erklärung anzeigt.

\paragraph{Content} Wenn der Bedarf einer Erklärung geklärt ist, muss festgelegt werden, welche und viele Informationen dem Nutzer eines erklärbaren Systems angezeigt werden sollen.

\paragraph{Presentation} Der Inhalt einer Erklärung kann Nutzern auf verschiedenen Wegen zugänglich gemacht werden. Welche Möglichkeiten bereits in der Literatur verwendet wurden, ist unter \textit{Presentation} zusammengefasst.

\paragraph{Evaluation} Unter \textit{Evaluation} werden Möglichkeiten zusammengetragen, wie gemessen werden kann, ob die in ein System integrierten Erklärungen das ursprüngliche Ziel erreichen. Die \textit{Evaluation} dient also vor allem der Bewertung der Qualität von Erklärungen.

\subsection{Struktur der Übersicht}

Da einige Aspekte in der Literatur zum Teil zusammen dargestellt wurden, sind diese auch im Modell hierarchisch angeordnet. Außerdem orientiert sich die Struktur der Übersicht über die Aspekte von Erklärbarkeit an dem Prinzip von Qualitätsmodellen wie es \citeauthor{schneider2012abenteuer} beschreibt \cite{schneider2012abenteuer} (Mehr siehe \autoref{sec:basics_quality_models}). Dabei werden abstrakte Ziele immer weiter konkretisiert bis sie schlussendlich mit konkreten Metriken messbar sind. Das zugrunde liegende Modell (\glqq Goal-Driven and Property-Based Definition Approach for Product Metrics\grqq{} \cite{briand1995goal}) von \citeauthor{briand1995goal} definiert außerdem unter anderem die nötigen Abhängigkeiten von äußeren Faktoren, die im Modell dem bereits erwähnten \textit{Context} entsprechen. Da die Metriken der Evaluation zum Teil von den entwickelten Erklärungen abhängen, müssen die Eigenschaften im Gegensatz zu dem erwähnten Modell von \citeauthor{schneider2012abenteuer} allerdings zuerst definiert werden. Anderenfalls wäre es nicht möglich, Metriken zu entwickeln, welche direkt die Eigenschaften der integrierten Erklärungen messen. Metriken, welche der Messung von Auswirkungen von integrierten Erklärungen in einem System dienen, können allerdings bereits zuvor aufgestellt werden. Für die Formalisierung des Messkonzeptes werden laut \citeauthor{briand1995goal} außerdem vorhandene Abstraktionen benötigt. Als Artefakt sollen hier die Ausprägungen der Kategorie \textit{Evaluation} in \autoref{sec:model_evaluation} dienen.

\smallbreak

Aus Vorherigem folgend werden die Punkte \textit{Context} und \textit{Objective} aus \autoref{tab:model_explaination_aspects} daher unter \textit{External Dependencies} zusammengefasst. Dies verdeutlicht, dass die \textit{Objectives} für das Integrieren von Erklärungen stark mit anderen äußeren Einflüssen (\textit{Context}) zusammenhängen (\autoref{sec:model_external_dependencies}).

Auch werden \textit{Demand}, \textit{Content} und \textit{Presentation} vereint dargestellt, da sich diese drei Kategorien direkt auf die Eigenschaften von Erklärungen beziehen: \textit{Characteristics}. Damit wird der starke Zusammenhang zwischen den verschiedenen Merkmalen einer Erklärung verdeutlicht.

\smallbreak

Folglich ist die Übersicht in die drei Kategorien \textit{External Dependencies}, \textit{Characteristics} und \textit{Evaluation} gegliedert, welche aufgrund der vorherigen Argumentation in dieser Reihenfolge in der Übersicht dargestellt sind. Ein Überblick ist in \autoref{fig:model_overview} zu sehen.

In den folgenden Abschnitten werden die Ausprägungen der einzelnen Kategorien sowie deren Anwendung in der Literatur beschrieben. Eine Übersicht ist in \autoref{fig:model_overview_complete} zu finden. Diese ist grafisch an der Taxonomie für Erklärungen von \citeauthor{nunes_systematic_2017} angelehnt und enthält auch einige Aspekte \cite{nunes_systematic_2017}. Die erwähnte Taxonomie ist allerdings nur auf den Einsatz von Erklärungen in Empfehlungssystemen bezogen und beschränkt sich daher auf bestimmte Darstellungstypen. Außerdem kann sie nicht ohne Weiteres in andere Kontexte übertragen werden. Der Aspekt der Evaluation fehlt darüber hinaus. Er wird allerdings nicht nur von den Autoren selbst, sondern auch von weiteren als wichtig erachtet \cite{cirqueira_scenario-based_2020, martin_evaluating_2021}.

\begin{figure}[htb!]
    \begin{center}
        \includegraphics[width=0.9\linewidth]{contents/05_model_description/res/model-overview.pdf}
    \end{center}
    \caption{Oberkategorien der Aspekte von Erklärungen}
    \label{fig:model_overview}
\end{figure}

\subsection{Externe Abhängigkeiten}
\label{sec:model_external_dependencies}

Unter \textit{External Dependencies} sind die Aspekte zusammengefasst, die eine Auswirkung auf die Erklärungen in einem System haben. Außerdem können von den hier aufgeführten Punkten Anforderungen abgeleitet und später zusammen mit Metriken Hypothesen aufgestellt werden. Daraus kann dann auch abgeleitet werden, welche Funktionen des Systems einer Erklärung bedürfen \cite{kohl_explainability_2019}. Im folgenden werden die beiden zusammenhängenden Unteraspekte \textit{Context} des Systems und \textit{Objectives} erläutert.

\subsubsection{Context}

Der \textit{Context} einer Erklärung beschreibt die äußeren Einflüsse, die unmittelbar auf das erklärbare System wirken und somit Anforderungen an die Eigenschaften von Erklärungen stellen.

Dies beinhaltet die Aktivität, die der Endbenutzer in einer bestimmten Umgebung durchführt. Aus den Eigenschaften der drei Aspekte (Aktivität, Endbenutzer und Umgebung) leiten sich dabei direkte Einflüsse auf den Bedarf, den Inhalt und die Darstellung einer Erklärung ab. \autoref{tab:impact_of_context_on_explanation} stellt wie bereits zuvor die verwendeten Synonyme für die verschiedenen Facetten des \textit{Context} eines Systems dar. Insbesondere der Begriff Stakeholder wurde in der Literatur verschieden eingesetzt. \citeauthor{cirqueira_scenario-based_2020} nutzen den Begriff für den Nutzer einer Software \cite{cirqueira_scenario-based_2020} während \citeauthor{nunes_systematic_2017} diesen als Oberbegriff für Personengruppen, die ein Interesse an einem System haben, verwenden, den Nutzer allerdings ausschließen \cite{nunes_systematic_2017}. Diese Arbeit verwendet den Begriff, um alle Personengruppen mit Interesse am System inklusive aller Nutzer zu beschreiben (vgl. \cite{schneider2012abenteuer,chazette_knowledge_nodate}). Folgend werden nun die drei Aspekte sowie typische Ausprägungen oder Charakteristiken näher erläutert.

\begin{table}[bht!]
    \begin{tabular}{|p{.2\textwidth}|p{.5\textwidth}|p{.2\textwidth}|}
        \hline
        \textbf{Aspekt} & \textbf{Synonyme} & \textbf{Quellen} \\ \hline
        End User        &  (Targt / End)  User & \cite{chazette2020explainability} \cite{kaptein_personalised_2017} \cite{sokol_one_2020} \cite{wiegand_id_2020} \\
                        & Stakeholder & \cite{chazette_knowledge_nodate} \\
                        & Consumer & \cite{ehsan_human-centered_2020} \\
                        & Explainee & \cite{chazette_knowledge_nodate} \cite{kohl_explainability_2019} \\
                        & Explanation Audience & \cite{sokol_explainability_2020} \\
        \hline
        Task            & Task & \cite{chazette_knowledge_nodate} \cite{sokol_explainability_2020} \cite{gunning2019darpa} \\
                        & Activity & \cite{wohlin2012experimentation} \\
        \hline
        Environment     & Environment & \cite{chazette_knowledge_nodate} \cite{wiegand_id_2020} \cite{wiegand2019drive} \\
                        & Application Area & \cite{sokol_explainability_2020} \cite{wiegand2019drive} \cite{wiegand_id_2020} \\
        \hline
    \end{tabular}
    \caption{Relevante Aspekte des Kontextes eines erklärbaren Systems zu Integration von Erklärungen.}
    \label{tab:impact_of_context_on_explanation}
\end{table}

\paragraph{End User} Der \textit{End User} ist diejenige Person, die mit dem System interagiert und auf welchen somit die Erklärungen zugeschnitten sein müssen. Dieser entspricht in den Definitionen von Erklärbarkeit von \citeauthor{chazette_knowledge_nodate} und \citeauthor{kohl_explainability_2019} dem \textit{Explainee} \cite{chazette_knowledge_nodate,kohl_explainability_2019}.

Generell kann ein System verschiedene Nutzer(-typen) haben, die sich auch in ihrem Bedarf für Erklärungen unterscheiden. Im Folgenden werden die drei am häufigsten erwähnten Eigenschaften vorgestellt (unter anderem in \cite{chazette_knowledge_nodate,tintarev_designing_nodate,yamada_evaluating_2016}).

Sowohl beim generellen technischen Verständnis (\textit{Technical Background}) als auch beim Domänen-Wissen (\textit{Domain Expertise}) \cite{yamada_evaluating_2016} können \textit{End User} verschieden viel Hintergrundwissen vorweisen. Somit können sich in einem System mit unterschiedlichen Eigenschaftzen der  Nutzer auch verschiedene Anforderungen an Erklärungen ergeben.

Der \textit{Cultural Background} fasst darüber hinaus die kulturellen Hintergründe des Nutzers zusammen, die sich zum Beispiel auf die Verwendung von Metaphern in Software bzw. Erklärungen auswirken können \cite{salgado_cultural_2015}.

\paragraph{Task} Der \textit{Task} definiert die Aufgabe(n), welche durch den \textit{End User} mithilfe des Systems durchgeführt werden sollen. Auch hier spielen verschiedene Eigenschaften eine Rolle. Genannt werden in der Literatur beispielsweise die Zeitabhängigkeit (\textit{Time Dependency}), die Komplexität (\textit{Complexity}) und die Dauer der Aufgabe (\textit{Length}). Die drei genannten Ausprägungen werden als wichtige Betrachtungsgegenstände für die Eigenschaften von Erklärungen beschrieben. Folglich sind auch diese Aspekte für die Anforderungserhebung im Kontext der Erklärbarkeit relevant \cite{sokol_explainability_2020}.

\paragraph{Environment} Sehr eng mit der zu erledigenden Aufgabe hängt auch dessen Umgebung zusammen. Das \textit{Environment} ist durch die äußeren Umstände des Systems definiert. Dies beinhaltet den generellen Anwendungsbereich des Systems (\textit{Area of Application}), welcher unter anderem die Kritikalität des Systems definiert. Aber auch die Art Nutzerinteraktion fällt in diesen Bereich (\textit{Interaction Type}) und hat eine Auswirkung auf die Anforderungen an Erklärungen \cite{wiegand_id_2020}. 

\bigskip

Zusammenfassend kann als Zwischenergebnis für \textbf{RQ1} festgehalten werden, dass der \textit{Context} eines Systems einer der zu betrachtenden Rahmenbedingungen mit einem Einfluss auf die Anforderungen an Erklärungen ist.

\subsubsection{Zielsetzung}
\label{subsec:model_objective}

Als zweiter Aspekt neben dem \textit{Context} werden in der Literatur die \textit{Objectives} hinter der Integration von Erklärungen mit einem Einfluss auf die Anforderungen an Erklärungen genannt \cite{rosenfeld_explainability_2019, nunes_systematic_2017}.

\citeauthor{chazette_knowledge_nodate} haben in ihrem Modell für Erklärbarkeit die Qualitätsaspekte zusammengefasst, die mit Erklärbarkeit in einem Zusammenhang stehen und somit auch als \textit{Objectives} für die Integration von Erklärungen gesehen werden können. Zusammen mit weiteren Ergebnissen der Literaturrecherche haben sich vor allem acht Qualitätsaspekte herausgestellt, welche auf die Auswirkungen durch Erklärungen untersucht wurden \cite{nunes_systematic_2017, tintarev2007survey}. Die Qualitätsaspekte sind in den meisten Fällen dafür genutzt worden, über das Messen dieser Aspekte die Qualität der integrierten Erklärungen indirekt zu bestimmten. \autoref{tab:quality_aspects_of_explanation} enthält eine nach der in der Literatur vorkommenden Häufigkeit sortierte Liste zusammen mit den Definitionen der Qualitätsaspekte. Diese enthält alle Arbeiten, welche den jeweiligen Aspekt explizit untersuchen oder Ergebnisse von Untersuchen dazu zusammenfassen.

\begin{table}[htb!]
    \begin{center}
        \begin{tabular}{|p{.24\textwidth}|p{.5\textwidth}|p{.2\textwidth}|}
            \hline
            \textbf{Qualitätsziel}    & \textbf{Beschreibung} & \textbf{Quellen} \\ \hline
            Trust                       & Das Vertrauen des Nutzers in das System erhöhen \cite[vgl.][]{balog_measuring_2020}
                                        & \cite{nunes_systematic_2017} \cite{chazette_knowledge_nodate} \cite{tintarev_designing_nodate} \cite{balog_measuring_2020} \cite{eiband_impact_2019} \cite{tintarev2015explaining} \cite{hernandez-bocanegra_effects_2020} \cite{stange_effects_2021} \cite{weitz_you_2019} \cite{yamada_evaluating_2016} \cite{haspiel_explanations_2018} \cite{martin_developing_2019} \cite{martin_evaluating_2021} \cite{tsai_effects_2020}  \cite{sokol_one_2020}  \cite{wang_is_2018} \cite{koo_understanding_2016} \cite{wiegand2019drive} \cite{gunning2019darpa} \cite{lim_2009_assessing} \cite{tintarev2007survey} \cite{kunkel_let_2019} \\ \hline
            Satisfaction                & Die Benutzerfreundlichkeit und generelle Zufriedenheit von Nutzern mit dem System erhöhen. \cite[vgl.][]{balog_measuring_2020}
                                        & \cite{nunes_systematic_2017} \cite{chazette_knowledge_nodate} \cite{tintarev_designing_nodate} \cite{balog_measuring_2020} \cite{tsai_evaluating_2019} \cite{tintarev2015explaining} \cite{riveiro_thats_2021} \cite{martin_developing_2019} \cite{martin_evaluating_2021} \cite{tsai_effects_2020} \cite{ehsan_human-centered_2020} \cite{sovrano_modelling_2020} \cite{koo_understanding_2016} \cite{ribera2019can} \cite{gunning2019darpa} \cite{lim_2009_assessing}  \cite{tintarev2007survey} \cite{sato_context_nodate} \\ \hline
            Transparency                & Erklären, wie das System funktioniert. \cite[vgl.][]{balog_measuring_2020}
                                        & \cite{nunes_systematic_2017} \cite{chazette_knowledge_nodate} \cite{tintarev_designing_nodate} \cite{chazette_end-users_nodate} \cite{balog_measuring_2020} \cite{chazette2020explainability} \cite{tintarev2015explaining} \cite{hernandez-bocanegra_effects_2020} \cite{tsai_effects_2020} \cite{rjoob_towards_2021}  \cite{sokol_one_2020} \cite{wang_is_2018} \cite{koo_understanding_2016} \cite{tintarev2007survey}\\ \hline
            Understandability           & Das Verständnis von Nutzern über das System erhöhen \cite[vgl.][]{chazette_end-users_nodate}
                                        & \cite{chazette_knowledge_nodate} \cite{chazette_end-users_nodate} \cite{martin_evaluating_2021}  \cite{ehsan_human-centered_2020} \cite{rjoob_towards_2021}  \cite{sokol_one_2020} \cite{cheng2019explaining} \\ \hline
            Scrutability                & Nutzern die Möglichkeit geben, dem System einen Fehler mitzuteilen \cite[vgl.][]{balog_measuring_2020}
                                        & \cite{nunes_systematic_2017} \cite{chazette_knowledge_nodate} \cite{tintarev_designing_nodate} \cite{balog_measuring_2020} \cite{tintarev2015explaining} \cite{martin_developing_2019} \cite{gunning2019darpa}  \cite{tintarev2007survey} \cite{martin_evaluating_2021} \\ \hline
            Efficiency                  & Nutzern helfen ihre Aufagaben schneller zu erledigen \cite[vgl.][]{balog_measuring_2020} 
                                        & \cite{nunes_systematic_2017} \cite{chazette_knowledge_nodate} \cite{tintarev_designing_nodate} \cite{balog_measuring_2020} \cite{tsai_evaluating_2019} \cite{tintarev2015explaining} \cite{hernandez-bocanegra_effects_2020} \cite{tintarev2007survey}\\ \hline
            Effectiveness               & Die Qualität der Aufgaben von Nutzern erhöhen \cite[vgl.][]{balog_measuring_2020}
                                        & \cite{nunes_systematic_2017} \cite{chazette_knowledge_nodate} \cite{tintarev_designing_nodate} \cite{balog_measuring_2020} \cite{tintarev2015explaining} \cite{zolotas_towards_2019} \cite{hernandez-bocanegra_effects_2020} \cite{martin_evaluating_2021} \cite{rjoob_towards_2021} \cite{tintarev2007survey} \\ \hline
            Persuasiveness              & Die akzeptanz der Entscheidungen des Systems durch die Nutzer erhöhen \cite[vgl.][]{chazette_knowledge_nodate}
                                        & \cite{nunes_systematic_2017} \cite{tintarev_designing_nodate} \cite{balog_measuring_2020} \cite{sato_context_nodate} \cite{sato_context_nodate} \cite{abdulrahman_belief-based_2019} \cite{tintarev2015explaining} \cite{sato_action-triggering_2019} \cite{tintarev2007survey} \\ \hline
        \end{tabular}
    \end{center}
    \caption{Qualitätsziele für Erklärungen – sortiert nach der Anzahl der Veröffentlichungen in der Literaturrecherche, die den zugehörigen Qualitätsaspekt untersucht haben.}
    \label{tab:quality_aspects_of_explanation}
\end{table}

\begin{table}[htb!]
    \begin{center}
        \begin{tabular}{|p{.2\textwidth}|p{.5\textwidth}|p{.2\textwidth}|}
            \hline
            \textbf{Aspekt}     & \textbf{Synonym} & \textbf{Quellen} \\ \hline
            Business Goals      & Stakeholder Goals & \cite{nunes_systematic_2017} \\
                                & (Intended) Purpose & \cite{waa_evaluating_2021} \\
                                & Higher-level Goals & \cite{nunes_systematic_2017} \\
                                & Application Level & \cite{sokol_explainability_2020} \\
            \hline
            Users'              & User Perceived Quality Factors & \cite{nunes_systematic_2017} \\
            Perception          & (Consumer) Needs & \cite{ehsan_human-centered_2020} \cite{chazette_end-users_nodate} \\
                                & User Goals & \cite{ehsan_human-centered_2020} \\
                                & Intermediate Requirements & \cite{waa_evaluating_2021} \\
                                & Human Level & \cite{sokol_explainability_2020} \\
            \hline
            Explanation         & (Explanation) Purpose & \cite{nunes_systematic_2017} \\
            Purpose             & Explanatory Goal & \cite{tintarev_designing_nodate} \cite{balog_measuring_2020} \\
                                & Function Level & \cite{sokol_explainability_2020} \\
            \hline
        \end{tabular}
    \end{center}
    \caption{Abstraktionsebenen der Zielsetzung bei der Integration von Erklärungen in ein System.}
    \label{tab:impact_of_objective_on_explanation}
\end{table}

In ihrer Definition von Erklärbarkeit sehen \citeauthor{chazette_knowledge_nodate} vor allem \textit{Transparency} und \textit{Understandability} als zentrale Ziele von Erklärbarkeit \cite{chazette_end-users_nodate}. Sie schreiben, dass diese unmittelbar durch die Integration von Erklärungen erreicht werden können. Wie in \autoref{tab:quality_aspects_of_explanation} zu sehen ist, werden Untersuchungen aber häufiger zu anderen Aspekten durchgeführt. Dies ist ein Indiz dafür, dass diese Aspekte nicht in einer flachen Hierarchie zueinander stehen. Eine hierarchische Anordnung der Qualitätsziele findest sich bei auch bei weiteren Autoren \cite{nunes_systematic_2017,tintarev2007survey}. Folglich werden die \textit{Objectives} in drei Abstraktionsebenen gegliedert: \textit{Business Goals}, \textit{Users' Perception} und \textit{Explanation Purpose}. Diese Ebenen, die von \citeauthor{nunes_systematic_2017} sowie \citeauthor{tintarev2007survey} vorgeschlagen werden, sind in diesem Modell als konkrete Abstraktionslevel für Qualitätsmodelle (vgl. \cite{schneider2012abenteuer}) zu interpretieren. Die Zuordnung der Qualitätsaspekte zu den Kategorien ist in der Gesamtübersicht des Modells abgebildet (\autoref{fig:model_overview_complete}). \autoref{tab:impact_of_objective_on_explanation} fasst die Erwähnungen der drei Zielebenen in der Literatur mit den jeweiligen Synonymen zusammen.

\paragraph{Business Goals} \textit{Business Goals} sind die Ziele, für das System im Ganzen gelten. \citeauthor{schneider2012abenteuer} nennt sie im Kontext von Qualitätsmodellen \glqq Allgemeine Qualitätsziele\grqq \cite{schneider2012abenteuer}.

\paragraph{User Perception Goals} \textit{User Perception Goals} sind jene Ziele, die direkt durch den \textit{End User} des Sytems wahrgenommen werden sollen. Die trägt der Erreichung der allgemeinen Ziele auf der höheren Ebene bei. Diese Ziele sind als Zwischenziele hin zu einem konkreten Ziel für zu integrierende Erklärungen zu verstehen.

\paragraph{Explanation Goals} Die \textit{Explanation Goals} sind die \glqq konkreten Qualitätsziele\grqq{} im Rahmen von Erklärbarkeit (vgl. \cite{schneider2012abenteuer}). Diese sind allerdings noch nicht konkrete Ziele für eine bestimmte Situation, in der die Erklärungen eingesetzt werden sollen. Hierfür müssen diese weiter verfeinert werden bis zu ihrer Messbarkeit, um daraus Anforderungen zu entwickeln (Siehe \autoref{sec:model_evaluation_description}).

\bigskip

Die hier vorgestellten Ziele bei der Integration von Erklärungen stellen keine abgeschlossene oder vollständige Liste dar, sondern dienen im Rahmen des Leitfadens als Überblick über in der Literatur häufig betrachtete Ziele, für deren Erreichung Erklärungen erfolgreich eingesetzt wurden.

\smallskip

\noindent\fbox{
    \parbox{0.964\textwidth}{
        \smallskip
        \textbf{RQ1} Welche Rahmenbedingungen haben einen Einfluss auf die Anforderungen für Erklärungen?
        \smallskip
    }
}

\smallskip

Mit dem \textit{Context} des Systems und den \textit{Objectives} für die Integration von Erklärungen wurden im Rahmen des ersten Modellteils die Rahmenbedingungen, welche einen direkten Einfluss auf Anforderungen an Erklärungen haben, vorgestellt. Die unter \textit{External dependencies} zusammengefassten Aspekte und Ausprägungen sind folglich die Antwort auf die erste Forschungsfrage.

Als Hilfsmittel kann dieser erste Teil des Modells vor allem bei der Anforderungserhebung (\textit{Requirements Elicitation}) und -Analyse (Requirements Analysis) helfen \cite{schneider2012abenteuer}. Auf dieser Basis können Anforderungen formuliert und die Grundlagen für Hypothesen gelegt werden. Folglich wird durch diesen ersten Modellteil auch die erste Modellanforderung erfüllt ([MR1]).

\smallskip

Im folgenden Abschnitt werden die Eigenschaften von Erklärungen vorgestellt, welche in der Literatur bereits verwendet wurden, um die Anforderungen umzusetzen.


\subsection{Eigenschaften}

Die \textit{Characteristics} umfassen jene Eigenschaften von Erklärungen, die einen Einfluss auf die externe Qualität eben dieser haben. Unter dem Aspekt sind die Möglichkeiten zusammengefasst, die bei der Ausgestaltung von Erklärungen bestehen. Gegliedert sind die Eigenschaften in den Bedarf der Erklärung (\textit{Demand}), die ausgelieferten Informationen (\textit{Content}) und die Bereitstellung (\textit{Presentation}). Diese drei Unterkategorien werden in den folgenden Abschnitten vorgestellt. Die vorangegangenen Arbeiten, in denen einzelne Eigenschaften untersucht wurden, werden in diesem Abschnitt jeweils in eigenen Tabellen mit aufgeführt, um als Beispiele für diese zu dienen. Wichtig zu beachten ist, dass sich die vorgestellten Möglichkeiten nicht unbedingt ausschließen. Das heißt vor allem, dass nicht jede Erklärung genau eine Ausprägung eines Aspektes erfüllen muss, sondern auch neue oder zwischen zwei Möglichkeiten liegende Eigenschaften aufweisen kann. Darüber hinaus muss auch nicht zwangsweise für jede Eigenschaftskategorie eine Eigenschaft ausgewählt werden, da nicht alle Kategorien auf jeden Kontext zutreffen.

\subsubsection{Bedarf}

Der \textit{Demand}  für Erklärungen kann aus verschiedenen Perspektiven betrachtet werden. Bei welcher Aufgabe und bei welchen Ereignissen Erklärungen überhaupt benötigt werden, muss in den Anforderungen für Erklärungen festgehalten werden. Diese entstehen auf Basis der \textit{External Dependencies}. Unter der Kategorie \textit{Demand} in diesem Modell ist zusätzlich festgehalten, mit welcher Initiative (\textit{Initiative}) und wann genau in Bezug auf ein Ereignis im System dem \textit{End User} Erklärungen vom System zur Verfügung gestellt werden sollen. Einen Überblick über die Verwendung verschiedener Ausprägungen der Aspekte ist in \autoref{tab:explanation_demands} dargestellt.

\begin{table}[bht!]
    \begin{center}
        \begin{tabular}{p{.25\textwidth}p{.25\textwidth}p{.41\textwidth}}
            \hline
            Aspekt    & Ausprägung   & Quellen \\
            \toprule
            initiative  &  Manual       & \cite{chazette_end-users_nodate} \cite{tintarev_designing_nodate}
                                            \cite{wiegand_id_2020} \\
                        &  Automatic    & \cite{chazette_end-users_nodate} \cite{eiband_impact_2019}
                                            \cite{wiegand_id_2020} \cite{schaffer_i_2019}
                                            \cite{yamada_evaluating_2016} \\
            \tablerowspacing
            Time        &  Before       & \cite{rosenfeld_explainability_2019} \cite{wiegand_id_2020}
                                            \cite{kunkel_let_2019} \cite{koo_why_2015} \cite{haspiel_explanations_2018} 
                                            \cite{haspiel_explanations_2018} \\
                        &  while        & \cite{rosenfeld_explainability_2019} \cite{wiegand_id_2020}
                                            \cite{kunkel_let_2019} \\
                        &  After        & \cite{rosenfeld_explainability_2019} \cite{wiegand_id_2020}
                                            \cite{kunkel_let_2019} \cite{koo_why_2015} \cite{haspiel_explanations_2018}
                                            \cite{wiegand2019drive} \cite{haspiel_explanations_2018} \\
            \toprule
        \end{tabular}
    \end{center}
    \caption{Der Bedarf einer Erklärung zusammen mit in der Literatur untersuchten Einflüssen auf die Qualität von Erklärungen}
    \label{tab:explanation_demands}
\end{table}

\paragraph{Initiative} Die \textit{Initiative} einer Erklärung ist der Auslöser für das Geben von Erklärungen in einem System. Eine Möglichkeit ist eine automatische Auslieferung der Erklärung an den \textit{End User}. Das System trifft dann allein die Entscheidung, wann und ob \textit{End User} eine Erklärung bekommt (\textit{Automatic}). Alternativ können Erklärungen vom \textit{End User} manuell angefordert werden (\textit{Manual}). Auch Mischformen, bei der \textit{End User} beispielsweise festlegen können, welche Art von Erklärungen sie erhalten möchten, sind denkbar.

\paragraph{Time} Die \textit{Time} ist der Zeitpunkt im Verhältnis zu einem Ereignis, zu dem das System eine Erklärung bereitstellt. \citeauthor{rosenfeld_explainability_2019} sowie \citeauthor{wiegand_id_2020} haben explizit untersucht, wann Erklärungen angezeigt werden sollten, wenn ein Ereignis im System auftritt oder das System eine Aktion durchführt. Dies umfasst die Möglichkeiten vor dem Ereignis (\textit{Before}), während (\textit{While}) oder nach dem Ereignis (\textit{After}) eine Erklärung zu diesem zu liefern \cite{rosenfeld_explainability_2019, wiegand_id_2020}. Letzteres wird in der Literatur zum Teil auch als \textit{Posthoc-Explanation} referenziert \cite{sokol_explainability_2020}.

\smallskip

Im Rahmen von \textbf{RQ2} kann an dieser Stelle folglich abgeleitet werden, dass der \textit{Demand} mit den verschiedenen Möglichkeiten für \textit{End User} eine Erklärung zu bestimmten Zeitpunkten zu erhalten eine Eigenschaft mit einem Einfluss auf die externe Erklärungsqualität ist.

\subsubsection{Inhalt}

Unter dem Punkt \textit{Content} wird definiert, mit welchen Inhalten die \textit{End User} durch Erklärungen versorgt werden \cite{nunes_systematic_2017}. Dieser ist einer von zwei Teilen des Modells, welcher sich auf die Granularität von Erklärungen bezieht. Der \textit{Content} beinhaltet nicht nur den Informationstyp (\textit{Information Type}) den das System vermittelt, sondern auch wie viel Inhalt (\textit{Information Density}). Ein weiterer Aspekt ist Anpassungsfähigkeit der Inhalte (\textit{Adaptivity}). \autoref{tab:content_of_explanations} enthält die verschiedenen Ausprägungen dieser Aspekte zusammen mit deren Anwendungen in der Literatur, welche einen Einfluss auf die externe Qualität von Erklärungen haben.

\begin{table}[bht!]
    \begin{center}
        \begin{tabular}{p{.25\textwidth}p{.25\textwidth}p{.41\textwidth}}
            \hline
            Aspekt    & Ausprägung   & Quellen \\
            \toprule
            Information Type        & Context     & \cite{chazette2020explainability} \cite{zahedi_towards_2019}
                                                \cite{cassens_ambient_2019} \cite{zahedi_towards_2019}
                                                \cite{zolotas_towards_2019} \cite{chari_explanation_2020}
                                                \cite{nunes_systematic_2017} \cite{ribera2019can} \\
                            & Inner Logic & \cite{chazette2020explainability} \cite{sato_action-triggering_2019} 
                                                \cite{thomson_knowledge--information_2020}
                                                \cite{chari_explanation_2020} \cite{neerincx_using_2018}
                                                \cite{ribera2019can} \cite{cassens_ambient_2019} \\
                                & Causality &   \cite{chazette2020explainability} \cite{abdulrahman_belief-based_2019}
                                                \cite{yamada_evaluating_2016} \cite{sato_action-triggering_2019}
                                                \cite{zahedi_towards_2019} \cite{zahedi_towards_2019}
                                                \cite{zolotas_towards_2019} \cite{cassens_ambient_2019}
                                                \cite{thomson_knowledge--information_2020}
                                                \cite{chari_explanation_2020} \cite{neerincx_using_2018}
                                                \cite{nunes_systematic_2017}\cite{zhu_effects_2020}
                                                \cite{ribera2019can} \cite{lim_2009_assessing} 
                                                \cite{kaptein_personalised_2017} \\
            \tablerowspacing
            Information          & Amount &      \cite{ribera2019can} \cite{kouki_user_2017}
                                                \cite{hernandez-bocanegra_effects_2020} \cite{martin_developing_2019} \\
            Density              & Abstraction Level & \cite{thomson_knowledge--information_2020}
                                                \cite{hernandez-bocanegra_effects_2020} \\
            \tablerowspacing
            Adaptivity          & Context-Sensitivity & \cite{kaptein_personalised_2017} \cite{cassens_ambient_2019} \\
                                & Controllability & \cite{abdulrahman_belief-based_2019} \cite{cheng2019explaining} \\
                                & Personalization & \cite{kaptein_personalised_2017} \cite{cassens_ambient_2019}
                                                    \cite{sokol_one_2020} \cite{tintarev_designing_nodate}
                                                    \cite{sokol_explainability_2020} \\
            \toprule
        \end{tabular}
    \end{center}
    \caption{Eigenschaften einer Erklärung bezogen auf den Inhalt einer Erklärung mit in der Literatur gezeigtem Einfluss auf die externe Qualität von Erklärungen}
    \label{tab:content_of_explanations}
\end{table}

\paragraph{Information Type} Der \textit{Information Type} beschreibt die Inhalte, die \textit{End Usern} mithilfe der Erklärung übermittelt werden. Unter diesem Aspekt sind in der Literatur sehr verschiedene Ansätze zu finden, die unterschiedliche Typen definieren. Beispielsweise stellen \citeauthor{chazette_end-users_nodate} mithilfe von Fragewörtern verschiedene Informationstypen dar \cite{chazette_end-users_nodate}, während \citeauthor{rosenfeld_explainability_2019} selbige Fragewörter nutzt, um andere Inhalte zu beschreiben und weitere ergänzt. Grundsätzlich kann zwischen globalen Erklärungen, die immer gültig sind und situationsabhängigen (lokalen) Erklärungen unterschieden werden \cite{lim_2009_assessing}. Zusammen mit weiteren \cite{kaptein_personalised_2017, abdulrahman_belief-based_2019} wurden die verschiedenen Definitionen in drei verschiedene Informationstypen gebündelt.

Kontextinformationen in einer Erklärung geben Auskunft über die zugrundeliegenden Daten (\textit{Context}). Dabei werden die eingehenden Informationen auf Basis derer das System Entscheidungen trifft, dem \textit{End User} dargelegt.

Eine weitere Möglichkeit ist das Erklären der Funktionsweise von Algorithmen eines Systems (\textit{Inner Logic}). Dies enthält die Informationen, wie genau ein System die ihm zur Verfügung stehenden Informationen verarbeitet und interpretiert.

Ein dritter Weg ist die Erklärung von Zusammenhängen zwischen den Eingaben und Ausgaben des Systems (\textit{Causality}). In einer solchen Erklärung wird den \textit{End Usern} der Grund für ein bestimmtes Systemverhalten oder eine Entscheidung erläutert. Dieser Punkt lässt sich in weitere Möglichkeiten gliedern, Gründe zu erläutern. Eine Erklärung dieser Art kann die Information enthalten, warum ein bestimmtes Systemverhalten in einer Situation erfolgt ist. Auch kann eine Erklärung klarstellen, warum ein alternatives Verhalten oder eine alternative Ausgabe des Systems nicht erfolgt ist \cite{martin_evaluating_2021}. % \cite{lim_2009_assessing} sagt, dass counterfactual nicht trivial ist.

% the different types of content answer the gulfs of execution and evaluation of \cite{norman1988psychology} as described by \cite{ribera2019can}

\paragraph{Information Density} Die \textit{Information Density} beschreibt die Menge und die Kompaktheit an Informationen die eine Erklärung enthält. Dabei ist einerseits wichtig, ob \textit{End Usern} alle vorliegenden Erklärungsmöglichkeiten vom System angezeigt werden (\textit{Amount}). Andererseits spielt es eine Rolle auf welcher Detailebene die Informationen dargestellt werden (\textit{Abstraction Level}).

\paragraph{Adaptivity} \textit{Adaptivity} definiert, wie statisch die Erklärungen in einem System sind. Eine Ausprägung ist dabei der Grad, zu dem eine Erklärung auf den aktuellen \textit{Context} angepasst ist (\textit{Context-Sensitivity}). Die beinhaltet auch die \textit{Personalisation}, welche darstellt, inwiefern Erklärungen auf den aktuellen \textit{End User} anpassbar sind. \textit{Controllability} beschreibt dabei, welchen Einfluss \textit{End User} haben, mit der Erklärung zu interagieren.

\bigskip

Als weiteres Ergebnis für \textbf{RQ2} kann hier hinzugefügt werden, dass auch der \textit{Content} einer Erklärung einen großen Einfluss auf die externe Qualität von Erklärungen hat. Dies lässt sich unter anderem an der Anzahl an Autoren festmachen, die verschiedene Facetten des \textit{Contents} auf die Einflüsse auf die Qualität untersuchen.

\subsubsection{Presentation}

Nachdem nun sowohl der \textit{Demand} als auch der an den \textit{End User} übermittelte Inhalt als zentrale \textit{Characteristics} von Erklärungen vorgestellt wurden, fehlt im Modell die Art der Präsentation der Erklärung an den \textit{End User}. Diese ist mit ihren zugehörigen Ausprägungen unter \textit{Presentation} zusammengefasst. Sie stellt den zweiten Teil der Granularität von Erklärungen dar. Zu dem Aspekt gehören in diesem Modell für Erklärungen das Medium (\textit{Medium}), über das die Erklärung \textit{End Usern} bereitgestellt wird, der verwendete Ton (\textit{Tone}) des \textit{Explainers} (siehe \autoref{02_basics:explainability}) \cite[vgl.][]{chazette_knowledge_nodate} und die Gruppierung von Erklärungen bzw. Erklärungstypen (\textit{Grouping}). \autoref{tab:presentation_of_explanations} stellt die Ausprägungen zusammen mit der Literatur, die den entsprechenden Aspekt in Bezug auf die externe Qualität von Erklärungen untersucht, dar.

\begin{table}
    \begin{center}
        \begin{tabular}{p{.25\textwidth}p{.25\textwidth}p{.41\textwidth}}
            \hline
            Aspekt     & Ausprägung & Quellen \\
            \toprule
            Medium              & Textual  &    \cite{sokol_explainability_2020} \cite{balog_measuring_2020}
                                                \cite{tintarev_designing_nodate} \cite{sato_action-triggering_2019}
                                                \cite{eiband_impact_2019} \cite{eiband_impact_2019}
                                                \cite{abdulrahman_belief-based_2019} \cite{cassens_ambient_2019}
                                                \cite{nunes_systematic_2017} \\
                                & Visual    &   \cite{sokol_explainability_2020} \cite{sato_action-triggering_2019} 
                                                \cite{mucha_interfaces_2021} \cite{abdulrahman_belief-based_2019}
                                                \cite{nunes_systematic_2017} \cite{schrills_color_2020} \\
                                & Auditory     &   \cite{wiegand2019drive} \cite{nunes_systematic_2017}
                                                \cite{wang_is_2018} \\
            \tablerowspacing
            Tone                & Factual   &   \cite{eiband_impact_2019} \cite{abdulrahman_belief-based_2019}
                                                \cite{kunkel_let_2019} \cite{neerincx_using_2018} \\
                                & Human     &   \cite{abdulrahman_belief-based_2019} \cite{kunkel_let_2019}
                                                \cite{weitz_you_2019} \cite{zahedi_towards_2019}
                                                \cite{neerincx_using_2018} \\
            \tablerowspacing
            Grouping            & Single    &   \cite{nunes_systematic_2017} \cite{balog_measuring_2020}
                                                \cite{sato_action-triggering_2019} \cite{eiband_impact_2019}
                                                \cite{abdulrahman_belief-based_2019} \\
                                & Grouped   &   \cite{nunes_systematic_2017} \cite{balog_measuring_2020}
                                                \cite{tintarev_designing_nodate}  \\
            \toprule
        \end{tabular}
    \end{center}
    \caption{Verschiedene Übermittlungsmöglichkeiten für Erklärungen an den \textit{End User}, die in der Literatur einen Effekt auf die externe Qualität von Erklärungen gezeigt haben}
    \label{tab:presentation_of_explanations}
\end{table}

\paragraph{Medium} Das \textit{Medium} einer Erklärung ist der Inforamtionsträger für die \textit{Presentation} der Inhalte. Dabei können verschiedene Möglichkeiten aus dem Multimedia-Bereich verwendet werden. In der Literatur untersucht wurden Texterklärungen (\textit{Textual}), visuelle Darstellungen wie z.B. Diagramme (\textit{Visual}) sowie auditive Erklärungen (\textit{Auditory}). Insbesondere dieser Aspekt bietet Möglichkeiten für Mischformen und Kombinationen \cite{kouki_user_2017}.

\paragraph{Tone} Der \textit{Tone} einer Erklärung bestimmt die Art, wie \textit{End Usern} der Inhalt einer Erklärung näher gebracht wird. Das Spektrum der Möglichkeiten erstreckt sich dabei vor allem zwischen sehr faktisch gehaltenen Erklärungen (\textit{Factual}) und persönlichen bzw. menschlichen Erklärungen (\textit{Human}).

\paragraph{Grouping} Mit dem \textit{Grouping} von Erklärungen wird bestimmt, wie viele Erklärungen \textit{End Usern} gleichzeitig beziehungsweise kombiniert präsentiert werden. Grundsätzlich gibt es dabei die Möglichkeiten eine Erklärung zur Zeit anzuzeigen (\textit{Single}) oder mehrere Erklärungen bzw. Erklärungstypen zu gruppieren (\textit{Grouped}). Bei letzterem können entweder Erklärungen mit verschiedener \textit{Presentation} aber dem gleichen \textit{Content} kombiniert werden \cite{kouki_user_2017} oder mehrere einzelne Erklärungen zusammen dargestellt werden \cite{balog_measuring_2020}.

\bigskip

Als letzter Teil der Antwort auf \textbf{RQ2} kann resultierend die \textit{Presentation} mit den dazugehörigen Aspekten als Eigenschaft von Erklärungen mit einem Einfluss auf die externe Qualität von Erklärungen ergänzt werden.

\smallskip

\noindent\fbox{
    \parbox{0.964\textwidth}{
        \smallskip
        \textbf{RQ2} Welche Eigenschaften von Erklärungen haben einen Einfluss auf die externe Qualität eines erklärbaren Systems?
        \smallskip
    }
}

\smallskip

Der in diesem Abschnitt vorgestellte Teil des Modells für Erklärungen (\textit{Characteristics}) beinhaltet die Antwort auf die zweite Forschungsfrage. Dabei sind explizit die Eigenschaften \textit{Demand}, \textit{Content} und \textit{Presentation} mit einem Einfluss auf die externe Qualität von Erklärungen zu benennen. Die genauen Ausprägungen dieser Eigenschaften wurden als Unterpunkte der jeweiligen Aspekte vorgestellt. Die Ergebnisse, die in dem Modell zusammengefasst sind, entspringen dabei der Evaluation von Erklärungen mit verschiedenen Eigenschaften in der Literatur. Der Einfluss auf die externe Qualität ist dabei insofern gewährleistet, als die in dem Modell erwähnten Eigenschaften nur dann Einzug erhalten haben, wenn in mindestens einer wissenschaftlichen Arbeit ein solcher Einfluss festgestellt wurde.

\subsection{Evaluation}
\label{sec:model_evaluation_description}

Qualität von Erklärungen zu bestimmen ist nicht einfach.

Behavioral vs. subjective???? / \glqq Importantly however, such measures often only measure one aspect of behavior. Ideally, a combination of both measurement types should be used to assess effects on both the user’s perception and behavior. In this way, a complete perspective on a construct can be obtained.\grqq{} (Qualitative / Quantitavie) \cite{waa_evaluating_2021}

Die Literaturrecherche hat wie bereits \cite{nunes_systematic_2017} nur empirische Studien zur Evaluation von Erklärungen gefunden.

Entsprechend \cite{wohlin2012experimentation} habe ich die verschiedenen Evaluationsmethoden in \textit{Qualitaative Research} und \textit{Quantitative Research} gegliedert.

Die Metriken und die Vorgehensweise, die für die \textit{Evaluation} ausgewählt wird, hängt folglich sehr eng mit den zuvor festgelegten \textit{Objectives} zusammen.

within und between subject erklären

\paragraph{Target} Zunächst muss bei der Evaluation geklärt werden, was der Prüfgegenstand ist. Dabei gibt es vor allem zwei große Möglichkeiten im Kontext der Erklärbarkeit. Entweder werden die integrierten Erklärungen an sich evaluiert und die Studienteilnehmer darauf explizit angesprochen oder es werden die Auswirkungen auf verschiedene System-Metriken ausgewertet. Auch eine Kombination ist möglich.

\paragraph{Strategy} Beim Festlegen der \textit{Strategy} der Evaluation gibt es verschiedene Möglichkeiten, die unter anderem vom \textit{Context} abhängen. Je nachdem, welche Ergebnisse die Stakeholder, die Erklärungen in ein System integrieren möchten, benötigen, muss die Evaluation kontrollierter oder weniger kontrolliert sein \cite[vgl.][]{wohlin2012experimentation}.

\paragraph{Metrics} \textit{Metrics} sind klar definierte Messungen, die durchgeführt werden, um die zuvor festglegten \textit{Objectives} zu überprüfen.

3 Types of Evaluation according to \cite{ribera2019can, doshi2017towards}: (1) applicationgrounded evaluation with real humans and real tasks; (2) human-grounded evaluation with real humans but simplified tasks; and (3) functionally-grounded evaluation without humans and proxy tasks; all of them always inspired by real tasks and real humans’ observations. \cite{wohlin2012experimentation}

\subsubsection{Qualitative Evaluation}

Bei der direkten Messung der Qualität von Erklärungen werden in der Literatur ledig verschiedene Möglichkeiten zur subjektiven Evaluation vorgestellt.

Neben den in \autoref{sec:model_external_dependencies} vorgestellten Qualitätsaspekten, die als Qualitätsziele für die Integration von Erklärungen definiert definiert wurden, gibt es weitere Aspekte, die in der Literatur zur Messung der Qualität von Erklärungen vorgestellt wurden \cite[sato_action-triggering_2019], die im folgenden erläutert werden.

\paragraph{Usefulness} \textit{Usefulness} oder auch \textit{Helpfulnesss} ist der Grad zu dem \textit{End User}, die Erklärungen erhalten, das subjektive Empfinden haben, dass eine Erklärung sie bei der Nutzung oder dem Verständnis über ein System unterstützt haben. 

\paragraph{Completeness} \textit{Completeness} ist das subjektive Empfinden von \textit{End Usern}, die gegebene Erklärungen vollständigen Aufschluss über den erklärten Systembestandteil geben und keine Informationen weglassen.

\smallskip

Bei der Messung der direkten Messung der Erklärugnsqualität setzt die Literatur vor allem Likert-Skalen ein. Dabei handelt es sich um einen Ordinalskala mit in der Regel fünf oder sieben einzelnen Bewertungsschritten, auf denen einen Aussage bewertet wird. Die genaue Benennung der Bewertungsschritte erfolgt in der Literatur verschieden. Allerdings werden meist solche mit einer inhaltlichen Übereinstimmung zu \glqq Volle Zustimmung\grqq{}, \glqq Teilweise Zustimmung\grqq{}, \glqq Neutral\grqq{},\glqq Teilweise Ablehnung\grqq{} und \glqq Volle Ablehnung\grqq{} verwendet \cite{sato_action-triggering_2019, sato_context_nodate, wang_is_2018, hoffman_metrics_nodate, koo_understanding_2016, koo_why_2015}. Aussagen, welche nicht dem Muster entsprechen, werden im im folgenden angepasst. \autoref{tab:evaluation_direct_measures_evaluation} stellt eine Übersicht von verwendeten Aussagen für die Messung der Qualitätsaspekte dar, über die die Erklärungsqualität messbar ist. Aufgelistet sind nur verallgemeinerbare Aussagen und nicht einen spezifischen \textit{Context} betreffende Aussagen. In spitzen Klammer sind Platzhalter dargestellt, um die aufgelisteten Aussagen besser auf den eigenen \textit{Context} anpassbar zu machen.


\begin{table}[htb!]
    \begin{center}
        \begin{tabular}{|p{0.25\textwidth} p{0.65\textwidth}|}
            \hline
            \textbf{Qualitätsaspekt} & \textbf{Aussage} \\
            \hline
            \hline
            Satisfaction    & Ich bin mit der Nutzung von <System> zufrieden .
                                \cite[vgl.][]{balog_measuring_2020} \\
                            & Ich werde <Sytem> wieder benutzen.
                                \cite[vgl.][]{balog_measuring_2020} \\
            \hline
            Persuasiveness  & Ich bin von den Entscheidungen von <System> überzeugt.
                                \cite[vgl.][]{tsai_effects_2020} \\
            \hline
        \end{tabular}
    \end{center}
    \caption{Aussagen zur qualitativen Evaluation ausgewählter Qualitätsaspekte im Bezug auf ein System bzw. Systemteile.}
    \label{tab:evaluation_qualitative_system_measures}
\end{table}

\begin{table}[htb!]
    \begin{center}
        \begin{tabular}{|p{0.25\textwidth} p{0.65\textwidth}|}
            \hline
            \textbf{Qualitätsaspekt} & \textbf{Aussage} \\
            \hline
            \hline
            Transparency    & Die Informationen der Erklärung waren ausreichend, um <Aufgabe> gut zu erfüllen. 
                                \cite[vgl.][]{wang_is_2018, balog_measuring_2020} \\
            \hline
            Satisfaction    & <Erklärung>, stellt mich mit meinem Verständnis über <System> zufrieden.
                                \cite[vgl.][]{riveiro_thats_2021} \\
                            & <Erklärung>, ist zufriedenstellend.
                                \cite[vgl.][]{riveiro_thats_2021, hoffman_metrics_nodate, balog_measuring_2020} \\
            \hline
            Persuasiveness  & <Erklärung>, ist überzeugend.
                                \cite[vgl.][]{sato_action-triggering_2019, sato_context_nodate} \\
                            & <Erklärung>, weckt Interesse. 
                                \cite[vgl.][]{sato_action-triggering_2019, sato_context_nodate} \\
            \hline
            Usefulness      & <Erklärung>, ist einfach zu verstehen. 
                                \cite[vgl.][]{sato_action-triggering_2019, sato_context_nodate} \\
            Helpfulness     & <Erklärung>, ist nützlich bei der Erfüllung von <Aufgabe>.
                                \cite[vgl.][]{sato_action-triggering_2019, sato_context_nodate, hoffman_metrics_nodate, balog_measuring_2020} \\
            \hline
            Completeness    & <Erklärung>, ist hinreichend vollständig.
                                \cite[vgl.][]{hoffman_metrics_nodate, riveiro_thats_2021} \\
                            & <Erklärung>, ist hinreichend detailliert.
                                \cite[vgl.][]{riveiro_thats_2021} \\
            \hline
        \end{tabular}
    \end{center}
    \caption{Aussagen zur qualitativen Evaluation ausgewählter Qualitätsaspekte im Bezug auf Erklärungen in einem System.}
    \label{tab:evaluation_qualitative_explanation_measures}
\end{table}

\begin{table}[htb!]
    \begin{center}
        \begin{tabular}{|p{0.25\textwidth} p{0.65\textwidth}|}
            \hline
            \textbf{Qualitätsaspekt} & \textbf{Aussage} \\
            \hline
            \hline
            Effectivity     & Ich konnte <Aufgabe> [mithilfe von <Erklärung>] erfolgreich[er] erledigen.
                                \cite[vgl.][]{balog_measuring_2020, hernandez-bocanegra_effects_2020} \\
            \hline
            Efficiency      & Ich konnte <Aufgabe> [mithilfe von <Erklärung>] schnell[er] erledigen.
                                \cite[vgl.][]{balog_measuring_2020, hernandez-bocanegra_effects_2020} \\
            \hline
            Trust           & Ich kann [mithilfe von <Erklärung>] [besser] sagen, wie vertrauenswürdig das System ist.
                                \cite[vgl.][]{hoffman_metrics_nodate, balog_measuring_2020, weitz_you_2019, hernandez-bocanegra_effects_2020} \\
            \hline
            Transparency    & Ich kann den Entscheidungsprozess von <System> [mithilfe von <Erklärung>] [besser]
                                verstehen.
                                \cite[vgl.][]{wang_is_2018, balog_measuring_2020} \\
                            & Ich kann [mithilfe von <Erklärung>] [besser] verstehen, wie <System> funktioniert.
                                \cite[vgl.][]{riveiro_thats_2021, hoffman_metrics_nodate, hernandez-bocanegra_effects_2020} \\
            \hline
            Scrutability    &  Ich kann [mithilfe von <Erklärung>] [besser] sagen, wie zuverlässig <System> ist.
                                \cite[vgl.][]{hoffman_metrics_nodate, balog_measuring_2020} \\
            \hline
        \end{tabular}
    \end{center}
    \caption{Aussagen zur qualitativen Evaluation ausgewählter Qualitätsaspekte im Bezug auf ein System bzw. Systemteile, Erklärungen in einem System oder dem Vergleich verschiedener Studienbedingungen.}
    \label{tab:evaluation_qualitative_explanation_system_measures}
\end{table}

Die Aussagen aus \autoref{tab:evaluation_qualitative_explanation_system_measures} können zum Teil auch als Entscheidungsfragen formuliert werden. Wenn dies der Fall war, wurde mitunter für eine der beiden Antwortmöglichkeiten zusätzlich ein Freitextfeld für eine Begründung bereitgestellt, z.~B. \glqq Hätten Sie sich gewünscht, dass die Erklärungen zusätzliche Informationen enthalten? Wenn ja, welche Art von Informationen und wann, d. h. in welchen Situationen?\grqq \cite[übersetzt vgl.][]{riveiro_thats_2021}.

Für Evaluationen, die als \textit{Wthin-Subject} Studiendesign angelegt sind, gibt es außerdem die Möglichkeit, die Aussagen so umzuformulieren, dass die jeweils abdecken, ob ein Aspekt besser unter einer Studienbedingung als unter der anderen ist.

Auch rankings zwischen den sachen. (Preference) \cite{kouki_user_2017} \cite{mucha_interfaces_2021} 
\cite{abdulrahman_belief-based_2019} 
\cite{waa_evaluating_2021} \cite{wiegand_id_2020} ,
\cite{stange_effects_2021} \cite{kaptein_personalised_2017} 

Neben diesen Einflüssen von Erklärungen die gemessen werden sollten auch andere  Mental Workload     (NASA TLX)  \cite{wiegand2019drive, wiegand_id_2020,du2019look} +  Nach Usability suchen (zumindest den Verweis, dass es gemacht werden sollte)

Think aloud \cite{wiegand_id_2020} \cite{yamada_evaluating_2016} \\

\subsubsection{Quantitative direkte Evaluation von Erklärungen}

Anzahl der Anforderungen

Dauer des Fokus auf eine Erklärung

\subsubsection{Qualitative Evaluation der Auswirkungen von Erklärungen}

\begin{table}[htb!]
    \begin{center}
        \begin{tabular}{|p{.45\textwidth}|p{.45\textwidth}|}
            \hline
            \textbf{Empirische Strategie} & \textbf{Quellen} \\ \hline
            Subjective Perception Questionaire &  \cite{balog_measuring_2020} \cite{sato_context_nodate}
                                                                \cite{waa_evaluating_2021} \cite{eiband_impact_2019}  \cite{kouki_user_2017} \cite{tsai_evaluating_2019}
                                                                \cite{hernandez-bocanegra_effects_2020}
                                                                \cite{zahedi_towards_2019} \cite{tsai_effects_2020} 
                                                                \cite{ribera2019can} \\
            Acceptance                        & \cite{tintarev_designing_nodate}
                                                            \cite{hernandez-bocanegra_effects_2020}
                                                            \cite{kunkel_let_2019} \\
            
            Preference                        & \cite{kouki_user_2017} \cite{mucha_interfaces_2021} 
                                                            \cite{abdulrahman_belief-based_2019} 
                                                            \cite{waa_evaluating_2021} \cite{wiegand_id_2020} ,
                                                            \cite{stange_effects_2021} \cite{kaptein_personalised_2017} \\
            Mental Model Understanding        & \cite{gunning2019darpa} \\
             \\
            \hline
        \end{tabular}
    \end{center}
    \caption{Evaluation}
    \label{tab:evaluation_of_explanations}
\end{table}

\subsubsection{Quantitative Evaluation der Auswirkungen von Erklärungen}

\cite{tintarev2007survey}:

\begin{itemize}
    \item Trust: Questionaires, Loyalty: Number of Logins
    \item Persuasiveness: Questionaires + Domain-Specific Performance metrics
    \item  Effectiveness: accuracy measures (Domain-Specific) 
    \item Efficiency: Task completion time, Number of times an explanation is called
\end{itemize}

\begin{table}[htb!]
    \begin{center}
        \begin{tabular}{|p{.45\textwidth}|p{.45\textwidth}|}
            \hline
            \textbf{Metrik} & \textbf{Quellen} \\ \hline
            Explanation exposure delta & \\
            Accuracy                          & \cite{tintarev_designing_nodate}
                                                            \cite{waa_evaluating_2021} \cite{mucha_interfaces_2021}
                                                            \cite{kunkel_let_2019} \cite{zolotas_towards_2019} \\
            Learning Rate                     & \cite{tintarev_designing_nodate} \cite{gunning2019darpa} \\
            Task Performance                  & \cite{waa_evaluating_2021}  \cite{mucha_interfaces_2021}  
                                                            \cite{abdulrahman_belief-based_2019} 
                                                            \cite{zolotas_towards_2019} \cite{martin_developing_2019} 
                                                            \cite{martin_evaluating_2021} \cite{gunning2019darpa} \\
            \hline
        \end{tabular}
    \end{center}
    \caption{Evaluation}
    \label{tab:evaluation_of_explanations}
\end{table}

\cite{tintarev_designing_nodate} haben Messliste gebaut.

Domain-Specific:

\begin{itemize}
    \item Special measures like ICM for ML-Models which is tied to the input and output of the model \cite{waa_evaluating_2021, neerincx_using_2018}
    \item Specific trust items e.g. for human-robot interaction used by \cite{zhu_effects_2020} originally developed by \cite{schaefer2013perception}
    \item 
    Domain specifc metrics. For example for explainable AI (Predictive systems) TYN (Trust-Your-Neighbours) or Meet in the Mittle (MITM) \cite{martin_evaluating_2021}
\end{itemize}


Finale Empfehlung: Einen vollständigen Überblick über die Qualität einer Erklärung bekommt man, wenn man Satisfaction, Scrutability und Translarency misst \cite{balog_measuring_2020}. Darauf sollte der Fokus liegen