\subsection{Kontext}

Der \textbf{Kontext} einer Erklärung beschreibt die äußeren Einflüsse, die unmittelbar auf das erklärbare System wirken und somit damit Anforderungen an die Eigenschaften einer Erklärung stellen. Dies beinhaltet die Aktivität, die der Endbenutzer in einer bestimmten Umgebung durchführt. Aus den Eigenschaften der drei genannten Aspekte leiten sich dabei direkte Einflüsse auf den Bedarf, den Inhalt und die Darstellung einer Erklärung ab.

Der Begriff Stakeholder wurde hier verschieden eingesetzt. Beispielsweise (quelle) nutzen den Begriff für den Nutzer einer Software und \cite{nunes_systematic_2017} als Oberbegriff für Personengruppen, die ein Interesse an einem System haben, den Nutzer allerdings ausschließen. Diese Arbeit geht von der Definition nach (Quelle suchen) aus, welch alle interessieren Gruppen beschreibt. 

Das Modell von \cite{nunes_systematic_2017} enthält allerdings weder den Kontext noch den Evaluationspart, welcher für einen Überblick über Erklärbarkeit wichtig ist. Der Fokus ist eher auch auf Recommender Systems gelegt

Nur die Kontexte, die explizit erwähnt wurden in \autoref{tab:impact_of_context_on_explanation}

\begin{longtable}{|p{.2\textwidth}|p{.5\textwidth}|p{.2\textwidth}|}
    \hline
    \textbf{Aspekt} & \textbf{Synonyme} & \textbf{Quellen} \\ \hline
    End User        & End User & \cite{chazette2020explainability} \cite{kaptein_personalised_2017} \cite{sokol_one_2020} \\
                    & Stakeholder & \cite{chazette_knowledge_nodate} \\
                    & Social Factor & \cite{ehsan_human-centered_2020} \\
                    & Consumer & \cite{ehsan_human-centered_2020} \\
                    & Target User (Group) & \cite{chazette2020explainability} \cite{wiegand_id_2020} \\
                    & Exlplainee & \cite{chazette_knowledge_nodate} \\
    \hline
    Environment     & Environment & \cite{chazette_knowledge_nodate} \cite{wiegand_id_2020} \cite{wiegand2019drive} \\
                    & Application Area & \cite{sokol_explainability_2020} \cite{wiegand2019drive} \cite{wiegand_id_2020} \\
    \hline
    Task            & Task & \cite{chazette_knowledge_nodate} \cite{sokol_explainability_2020} \cite{gunning2019darpa} \\
                    & Activity & \cite{wohlin2012experimentation} \\
                    
    \hline
\caption{Kontexte einer Erklärung}
\label{tab:impact_of_context_on_explanation}
\end{longtable}

\paragraph{Endnutzer}

Different synonyms for it like....

\paragraph{Äußere Bedingungen}

\paragraph{Aufgabe}