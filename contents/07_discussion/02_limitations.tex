\section{Allgemeingültigkeit der Ergebnisse}

Mit dem in dieser Arbeit entwickelten Leitfaden für die Integration von Erklärungen in Softwaresysteme fasst diese Arbeit die Ergebnisse aus vorangegangenen Arbeiten zusammen, die in Bezug auf den Einfluss von Erklärungen in erklärbaren Systemen existieren. Folglich sind in dem entstandenen Modell für Erklärungen, dem Katalog über die Einflüsse von Erklärungen auf bestimmte Qualitätsaspekte und den daraus abgeleiteten Design Heuristiken keine neuen Ergebnisse entstanden, sondern nur bestehende Resultate in einen Zusammenhang gestellt worden. Daher kann davon ausgegangen werden, dass die Empfehlungen des Leitfadens gültig sind. Allerdings erhebt der Leitfaden keinen Anspruch auf Vollständigkeit. Vor allem für spezifische Kontexte bieten bereits entwickelte Modelle einen tieferen Einblick \cite{nunes_systematic_2017, sokol_explainability_2020}. Eine Limitierung, die bei der Entwicklung des Leitfandens auf Basis einer Literaturrecherche gemacht werden muss, ist dass diese nur von einer Person durchgeführt und daher nicht geprüft wurde. Daher können Aspekte übersehen worden sein. Dies unterscheidet die im Rahmen dieser Arbeit durchgeführte Literaturrecherche vor allem von systematischen Literaturrecherchen \cite[vgl.][]{nunes_systematic_2017,chazette_knowledge_nodate}.

Für die Verallgemeinerbarkeit des Leitfadens spricht auch, dass bei der Anwendung verschiedene Ergebnisse und Empfehlungen aus vorangegangenen Arbeiten reproduziert werden konnten. Beispielsweise konnte die Notwendigkeit von qualitativer und quantitativer Evaluation gezeigt werden, um interpretierbare Ergebnisse zu erlangen \cite{}. Auch konnte gezeigt werden, dass die Performanz der \textit{End User} durch Erklärungen erhöht aber die \textit{Satisfaction} sinken kann \cite{}. 

\smallskip

In dieser Arbeit wurde versucht, den Leitfaden ganzheitlich als Methode in der Wirtschaft anzuwenden. Dabei hat sich allerdings herausgestellt, dass nicht alle Teile des Leitfadens für jedes Team einsetzbar sind.

Beispielsweise wurden die Ziele zur Integration von Erklärungen von Graphmasters sehr allgemein gehalten, da eine Konkretisierung der Ziele zum Beispiel mittels Qualitätsmodellen \cite{schneider2012abenteuer} in den Prozessen des Unternehmens nicht vorgesehen ist. Um allerdings den Katalog der Zusammenhänge des Leitfadens anzuwenden wurden die Anforderungen im Rahmen dieser Arbeit trotz dessen konkretisiert. Auch kommen wie bereits zuvor erläutert nicht immer alle Evaluationsmöglichkeiten des Modells infrage, da es äußere Beschränkungen gibt (siehe \autoref{sec:02_evaluation_explanations}).

Als Einschränkung für die Allgemeingültigkeit des Leitfadens muss auch erwähnt werden, dass der Einsatz mit dieser Arbeit nur in einem agil arbeitenten Unternehmen im \textit{Context} von mobilen Anwendungen untersucht wurde. Die Ergebnisse, in welchen Bereichen Teile des Leitfadens sinnvoll eingesetzt werden konnten, sind folglich nicht verallgemeinerbar. Auch die Ergebnisse, inwiefern sich die entwickelten Erklärungen mit bestimmten Eigenschaften auf Qualitätsaspekte auswirken ist nur für den \textit{Context} einer Navigationsansicht gezeigt worden. Außerdem muss zur evaluation der Erklärung einschränkend hinzugefügt werden, dass die qualitative Analyse lediglich mit vier \textit{End Usern} im Rahmen eines \textit{Quasi-Experiments} durchgeführt wurde und daher keine statisch signifikanten Ergebnisse liefert, sondern nur als Interpretationshilfe diente.

Die letzte hier erwähnte Beschränkung der Verallgemeinerbarkeit der Anwendung des Leitfadens ist, dass keine direkte Evaluation des Leitfadens erfolgt ist, sondern dies nur indirekt zur dessen Anwendung erfolgt ist. Es können folgich Verständnisprobleme oder Herausforderungen bei der Nutzung in verschiedenen Prozessen bei der Entwicklung von Erklärungen auftreten.

Final ist zusammenzufassen, dass die Ergebnisse, welche der Leitfaden enthält, möglichst allgemein gehalten sind oder mit entsprechenden Einschränkungen versehen sind. Somit ist es für Anwender des Leitfadens möglich anhand der Einschränkungen des eigenen \textit{Contexts} und äußeren Bedingungen die Teile des Leitfadens zu wählen, die ihnen bei der Integration von Erklärungen helfen können. Daher bietet der Leitfaden eine gewisse Flexibilität.