\section{Limitierungen des Leitfadens für Erklärungen}

Der in dieser Arbeit entwickelte Leitfaden für die Integration von Erklärungen in Softwaresysteme fasst bereits gezeigte Ergebnisse zusammen und strukturiert diese. Folglich sind in dem entstandenen Modell für Erklärungen, dem Katalog über die Einflüsse von Erklärungen auf bestimmte Qualitätsaspekte und den daraus abgeleiteten Design Heuristiken keine neuen Ergebnisse entstanden, sondern nur bestehende Resultate in einen Zusammenhang gestellt worden. 
Der Leitfaden erhebt allerdings keinen Anspruch auf Vollständigkeit. Vor allem für spezifische Kontexte bieten bereits entwickelte Modelle einen tieferen Einblick \cite{nunes_systematic_2017, sokol_explainability_2020}.

Eine weitere Limitierung, die bei der Entwicklung des Leitfadens auf Basis einer Literaturrecherche gemacht werden muss, ist, dass diese nur von einer Person durchgeführt und daher nicht überprüft wurde. Daher können Aspekte übersehen oder fehlerhaft eingeordnet worden sein. Dies unterscheidet die im Rahmen dieser Arbeit durchgeführte Literaturrecherche von bereits durchgeführten systematischen Literaturrecherchen zum Thema \textit{Explainability} \cite[vgl.][]{nunes_systematic_2017,chazette_knowledge_nodate}.

In dieser Arbeit wurde versucht, den Leitfaden ganzheitlich als Methode in der Wirtschaft anzuwenden. Dabei hat sich herausgestellt, dass nicht alle Teile des Leitfadens für jedes Team einsetzbar sind. Beispielsweise wurden die Ziele zur Integration von Erklärungen von \textit{Graphmasters} sehr allgemein gehalten, da eine Konkretisierung der Ziele zum Beispiel mittels Qualitätsmodellen \cite{schneider2012abenteuer} in den Prozessen des Unternehmens nicht vorgesehen ist. Um allerdings den Katalog der Zusammenhänge des Leitfadens anzuwenden, wurden die Anforderungen im Rahmen dieser Arbeit trotz dessen konkretisiert und mit \textit{Graphmasters} abgesprochen. Auch kommen nicht immer alle Evaluationsmöglichkeiten des Modells für einen Kontext infrage, da es äußere Beschränkungen gibt (siehe \autoref{sec:02_evaluation_explanations}).

Als Einschränkung für die Allgemeingültigkeit des Leitfadens muss auch erwähnt werden, dass der Einsatz mit dieser Arbeit nur in einem agil arbeitenden Unternehmen im \textit{Context} von mobilen Anwendungen untersucht wurde. Um den Nutzen der einzelnen Leitfadenteile allgemein beurteilen zu können, müssen folglich weitere Anwendungen des Leitfadens innerhalb von verschiedenen Prozessen erfolgen.

Die Ergebnisse, in welchen Bereichen Teile des Leitfadens sinnvoll eingesetzt werden können, sind folglich nicht verallgemeinerbar.

Eine weitere Einschränkung bei der Verallgemeinerung des Leitfadens ist, dass keine direkte Evaluation des Leitfadens erfolgt ist. Diese ist nur indirekt durch die Anwendung des Leitfadens geschehen. Es können folglich Verständnisprobleme oder Herausforderungen bei der Nutzung in verschiedenen Prozessen bei der Entwicklung von Erklärungen auftreten.

Für die Verallgemeinerbarkeit des Leitfadens spricht, dass bei der Anwendung des Leitfadens mehrere Ergebnisse und Empfehlungen aus vorangegangenen Arbeiten reproduziert werden konnten. Beispielsweise konnte die Notwendigkeit von qualitativer und quantitativer Evaluation gezeigt werden, um interpretierbare Ergebnisse zu erlangen \cite{sokol_explainability_2020}. Auch konnte gezeigt werden, dass die Performanz der \textit{End User} durch Erklärungen erhöht und gleichzeitig die \textit{Satisfaction} sinken kann \cite{koo_understanding_2016}. 

Final bieten die Ergebnisse einen allgemeinen Überblick für die Integration von Erklärungen in erklärbare Systeme. Somit ist es für Anwender des Leitfadens möglich, anhand von Einschränkungen des eigenen \textit{Contexts} und äußeren Bedingungen die Teile des Leitfadens zu wählen, die ihnen bei der Integration von Erklärungen helfen können. Der Leitfaden bietet folglich eine gewisse Flexibilität.

\newpage

