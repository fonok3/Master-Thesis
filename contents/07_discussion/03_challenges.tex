\section{Herausforderungen}

Im Folgenden werden die Herausforderungen vorgestellt, die während der Forschung für diese Arbeit entstanden sind.

Eine der Hauptaufgaben dieser Arbeit ist das Zusammentragen der bestehenden Ergebnisse zur neuen NFR \textit{Explainability} gewesen. Vor allem durch die sehr subjektive und in verschiedenen Kontexten sehr unterschiedliche Wahrnehmung von Erklärungen, war die Vereinheitlichung der Ergebnisse nicht trivial. Da vergange Forschung vor allem einzelne Aspekte von Erklärungen ohne ein einheitliches Evaluationsschema analysiert hat, war auch die Aufstellung der Zusammenhänge zwischen den einzelnen Aspekten des in der vorliegenden Arbeit vorgestellten Leitfadens nicht leicht. Hier konnten vor allem das existierende Modell von \citeauthor{nunes_systematic_2017} für Erklärungen im Kontext von Empfehlungssystemen und die Definition für Erklärbarkeit von \citeauthor{nunes_systematic_2017} eine Orientierung geben \cite{,nunes_systematic_2017, chazette_knowledge_nodate}. Aufgrund der Diversität von \textit{Explainability} sind auch weniger allgemeine Resultate im Rahmen dieser Arbeit entstanden als zu Beginn der Literaturrecher erwartet und der Leitfaden bietet vor allem einen Überblick über verschiedene Aspekte von Erklärungen.

Neben den Herausforderungen bei der Entwicklung des Leitfadens sind während dem Technologietransfer in die Praxis weitere Hürden zu überwinden gewesen. Der Leitfaden ist mit Zielen konzipiert, welche für die Anwendung in Qualitätsmodellen zur Ableitung von Anforderungen dienen. Die agile Arbeitsweise von Graphmasters ist allerdings nicht für eine konkrete Anforderungserhebung ausgelegt und daher ist diese Methode auch unbekannt gewesen. Folglich war es schwer in Absprache mit dem Team von Graphmasters konkret überprüfbare Anforderungen zu entwickeln. Insbesondere das Festlegen von Sollwerten für die entwickelten Metriken konnte nicht erfolgen. Allerdings sind trotz dessen statistisch überprüfbare Qualitätsanforderungen entstanden (siehe \autoref{sec:explanation_requirements}). Hier ist also vor allem der Unterschied zwischen in der Wissenschaft entwickelten Modellen und Verwendung in der Wirtschaft offenbar geworden. Interessant war allerdings, dass der entwickelte trotz dessen eine Unterstzützung für Aktivitäten wie der Gestaltung der Erklärungen oder der Evaluation war. Folglich wäre der Leitfaden als Artefakt außerhalb dieser Arbeit in anderer Weise in die agilen Prozesse von Graphmasters integriert worden.