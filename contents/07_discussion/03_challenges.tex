\section{Herausforderungen}

Generalisierung von Aspekten der Erklärbarkeit, da eine große Kontextabhängigkeit vorherscht. Chazette et al. haben ja bereits die Vielzahl an Abhöngigkeiten dargestellt. Auf dieser Aufbauen die Eigenschaften, welche die se Abhängigkeiten unterstützen zu finden ist ganz schon schwer. Daher ist dieser Leitfaden ein erster Überblick und sollte keinesfalls als fest gesetzt oder vollständig betrachtet werden. Auch hat man gesehen, dass es nur wenige Empfehlungen durch den Leitfaden gibt, welche Allgemeingültig sind.

Methode vs. die gängige Praxis bei Graphmasters

Es gibt auch gefahren, da zum Beispiel egal ist, ob die Erklärung richtig ist.

Es fehlt ein ausgereifter Katalog, für Zusammenhänge bei Erklärungen
Es fehlt ein klarer Explainability SIG wie auch für Transparency \cite{do2010software} und Invisibility \cite{carvalho2020developers}. m Entwickler einfacher supporten zu können, sollte also vor allem für die Objectives eine bessere Übersicht geschaffen werden.
Außerdem muss mehr evaluiert werden und die Vollständigkeit des Modells sollte durch mindestens eine weitere unabhngige Arbeit bestätigt werden.