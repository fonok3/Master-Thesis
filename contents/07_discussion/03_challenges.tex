\section{Herausforderungen}

Im Folgenden werden die Herausforderungen vorgestellt, die während der Forschung für diese Arbeit entstanden sind.

Eine der Hauptaufgaben dieser Arbeit ist das Zusammentragen der bestehenden Ergebnisse zur neuen NFR \textit{Explainability} gewesen. Vor allem durch die sehr subjektive und in verschiedenen Kontexten sehr unterschiedliche Wahrnehmung von Erklärungen war die Vereinheitlichung der Ergebnisse nicht trivial. Da vergangene Forschung vor allem einzelne Aspekte von Erklärungen ohne ein einheitliches Evaluationsschema analysiert hat, war auch die Aufstellung der Zusammenhänge zwischen den einzelnen Aspekten des in der vorliegenden Arbeit vorgestellten Leitfadens eine Herausforderung. Hier konnten vor allem das existierende Modell wie von \citeauthor{nunes_systematic_2017} für Erklärungen im Kontext von Empfehlungssystemen und die Definition für Erklärbarkeit von \citeauthor{chazette_knowledge_nodate} eine Orientierung geben \cite{nunes_systematic_2017, chazette_knowledge_nodate}. Aufgrund der Diversität von \textit{Explainability} sind auch weniger allgemeine Resultate im Rahmen dieser Arbeit entstanden als zu Beginn der Literaturrecherche erwartet. Daher bietet der Leitfaden vor allem einen Überblick über verschiedene Aspekte von Erklärungen.

Neben den Herausforderungen bei der Entwicklung des Leitfadens sind während dem Technologietransfer in die Praxis weitere Hürden zu überwinden gewesen. Der Leitfaden ist mit Zielen konzipiert, welche für die Anwendung in Qualitätsmodellen zur Ableitung von Anforderungen dienen. Die agile Arbeitsweise von Graphmasters ist allerdings nicht für eine konkrete Anforderungserhebung ausgelegt. Daher war diese Methode unbekannt. Folglich bedurfte die Festlegung der konkreten und überprüfbaren in Absprache mit Graphmasters einem besonderen Augenmerk. Insbesondere das Festlegen von Sollwerten für die entwickelten Metriken konnte nicht erfolgen. Final sind statistisch überprüfbare Qualitätsanforderungen entstanden (siehe \autoref{sec:explanation_requirements}). Hier hat sich vor allem der Unterschied zwischen in der Wissenschaft entwickelten Modellen und Verwendung in der Wirtschaft gezeigt. Trotz dessen hatte der entwickelte Leitfaden eine unterstützende Wirkung für die Aktivitäten der Integration von Erklärungen (Anforderungserhebung, Umsetzung, Evaluation). Für eine erneute Verwendung des Leitfadens bei Graphmasters muss allerdings eine bessere Integration in den agilen Prozess von Graphmasters erfolgen.