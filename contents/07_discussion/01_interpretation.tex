\section{Interpretation der Ergebnisse}

\subsection*{Beantwortung der Forschungsfragen}

Mit einem Modell für die Aspekte von Erklärungen, den Zusammenhängen und Einflüssen auf ausgewählte externe Qualitätsaspekte sowie Heuristiken zur Gestaltung von Erklärungen ist zusammenfassend ein Leitfaden zur Unterstützung von Erklärungen

Aus Studie nochmals klar geworden, dass nur mit Verhaltensmetriken, sobald diese nicht eindeutig sind, und direkt die \glqq richtige\grqq{} Erklörung getroffen wurde, die Erklärungen nicht analysiert werden können.

\subsection*{Bewertung des Leitfadens für die Integration von Erklärungen}

Anhand der Rückfragen zum Leitfaden für die Integration von Erklärungen konnte abgeleitet werden, dass dieser zum Zeitpunkt des Workshops nur wenige offensichtliche Unverständlichkeiten aufwies. Lediglich die verschiedenen Inhaltstypen von Erklärungen haben nicht alle Teilnehmer des Workshops direkt verstanden. Die Typen waren zu dem Zeitpunkt noch als Fragewörter voneinander abgegrenzt, was für das finale Modell in der Folge geändert wurde (siehe \autoref{fig:model_overview_complete}).

Diese können folglich ohne ernste Bedenken über die Studie hinaus in \textit{NUNAV Navigation} integriert bleiben.

Der Leitfaden an sich enthält keine Einführung in Erklärbarkeit, diese wurde aber im Rahmen einer Präsentation als Einführung des Workshops gegeben. Dabei ist aufgefallen, dass eine wichtige Verständnisfrage für die Teilnehmer war, wo die genaue Abgrenzung zwischen Erklärbarkeit und gutem User-Interface-Design ist. Für die Anwendung des Leitfadens wird dieses Wissen über Erklärbarkeit vorausgesetzt. Folglich hat in diesem Fall eine Einführung in das Thema Erklärbarkeit sehr geholfen.

Außerdem ist als Beobachter aufgefallen, dass insbesondere die verschiedenen Möglichkeiten, Erklärungen zu gestalten aus dem Leitfaden die Diskussion über Umsetzungsideen angeregt haben. Da eines der Ergebnisse war, dass die Evaluation vor allem durch Verhaltensmetriken erfolgen sollte, konnten viele der im Leitfaden vorgestellten qualitativen Metriken nicht zum Einsatz kommen.

es konnten nicht für jeden Nutzer alle Fragen beantwortet werden. Lösung -> Support Desk innerhalb der App.

weitere Metriken entwickeln. Daten bekommen ist nicht einfach.