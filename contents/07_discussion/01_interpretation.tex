\section{Interpretation der Ergebnisse}

Für die Einordnung der Ergebnisse aus den beiden Teilen dieser Arbeit ist vor allem interessant, inwiefern der im ersten Teil entwickelte Leitfaden in der Praxis anwendbar ist. Für die Beurteilung des Nutzens des enthaltenen Modells und der Richtlinien zur Integration von Erklärungen in ein bestehendes System wurde der Leitfaden bei der Firma Graphmasters eingesetzt. Um die Anwendbarkeit des entwickelten Leitfadens zu analysieren, werden im Folgenden zum einen die Erfahrungen, die während der Anwendung gemacht wurden, ausgewertet. Zum anderen wird die Qualität der im Rahmen der Anwendung entstandenen Erklärungen betrachtet.

Der Einstieg in die Anwendung ist wie beschrieben innerhalb eines Workshops erfolgt, welcher zusätzlich zur Vorstellung des Leitfadens mit einer Einführung in das Thema Erklärbarkeit begonnen hat (siehe \autoref{sec:explanation_demand_generation}). Dabei ist aufgefallen, dass eine wichtige Verständnisfrage für die Teilnehmer war, wo die genaue Abgrenzung zwischen Erklärbarkeit und gutem User-Interface-Design ist. Für die Anwendung des Leitfadens wird dieses Wissen über Erklärbarkeit vorausgesetzt. Folglich hat in diesem Fall eine Einführung in das Thema Erklärbarkeit sehr geholfen. Der Leitfaden an sich enthält allerdings keine Einführung in das Thema. Die Zielgruppe dieses ersten Entwurfs sind folglich Stakeholder, welchen die Thematik \textit{Explainability} bekannt ist.

Anhand der Rückfragen bei der Vorstellung des Leitfadens zur Integration von Erklärungen konnten außerdem Verständnisprobleme innerhalb des Leitfadens behoben werden. Insbesondere die Abgrenzung zwischen verschiedenen Inhaltstypen von Erklärungen im Modell für Erklärungen haben nicht alle Teilnehmer des Workshops direkt verstanden. Die Typen waren zum Zeitpunkt des Workshops als Fragewörter voneinander abgegrenzt. Für das finale Modell wurden daher einzelne Begriffe des Modells überarbeitet (siehe \autoref{fig:model_overview_complete}). Mit diesen integrierten Änderungen kann darauf geschlossen werden, dass Nutzer des Leitfadens mit Vorwissen über das Thema Erklärbarkeit diesen zu großen Teilen verstehen können. Eine direkte Evaluation der Verständlichkeit des Leitfadens ist allerdings im Rahmen dieser Arbeit nicht erfolgt.

Außerdem ist als Beobachter während des Workshops aufgefallen, dass insbesondere die verschiedenen Möglichkeiten im Leitfaden, Erklärungen zu gestalten, die Diskussion zu verschiedenen Umsetzungsideen angeregt haben. Somit haben sich vor allem die verschiedenen \textit{Characteristics} zusammen mit den Einflüssen auf verschiedene Qualitätsaspekte als sehr hilfreich für die Entwicklung von Erklärungen herausgestellt. Folglich hat 

Zusätzlich hat das Modell geholfen, die Ideen zur Evaluation zu systematisieren und folglich Metriken festzulegen. Allerdings konnten bei der Evaluation der integrierten Erklärungen viele vorgeschlagene Metriken des im Leitfaden enthaltenen Modells nicht verwendet werden, da eine Anforderung durch Graphmasters war, in der Case Study vor allem Verhaltensmetriken zu nutzen, um die \textit{End User} möglichst wenig zu beeinflussen.

Des Weiteren hat der Katalog der Zusammenhänge von bestimmten Eigenschaften von Erklärungen aus einiger Erfahrung dabei geholfen, aus Rohanforderungen und Entwurfsideen für Erklärungen, welche in dem Workshop entstanden sind, zu konkretisieren. Dabei haben insbesondere die vorgeschlagenen Ziele für die Integration von Erklärungen zusammen mit der Anwendung eines konkreten Qualitätsmodells für die Integration von Erklärungen in \textit{NUNAV Navigation} eine große Rolle gespielt.

Folglich kann abgeleitet werden, dass aus den ursprünglich an den Leitfaden formulierten Anforderungen ein Leitfaden entstanden ist, welcher eine Unterstützung bei der Integration von Erklärung bietet. Dabei wurden bei der Anwendung des Leitfadens sowohl die äußeren Einflüsse für Erklärungen, die Eigenschaften sowie deren Evaluation angewendet. Auch sind die Zusammenhänge insbesondere während der Umsetzung der Erklärungen untersützend verwendet worden.

Aus der Evaluation der entstandenen Erklärungen kann des Weiteren gefolgert werden, dass durch die Anwendung des Leitfadens entwickelt werden konnten, welche positive Einflüsse auf die Softwarequalität eines Systems haben. Die Untersuchung des Einflusses der Erklärungen hat aber auch gezeigt, dass nicht alle entwickelten Erklärungen einen positiven Einfluss auf die Softwarequalität erreichen konnten. Hier hat allerdings die anhand des Leitfadens entwickelte qualitative Evaluation mögliche Probleme und Verbesserungsmöglichkeiten aufgedeckt, welche in einer weiteren Iteration der Erklärungen umgesetzt werden können. Darüber hinaus deuten weder qualitative noch quantitative Ergebnisse der Evaluation auf negative Effekte durch die Erklärungen hin. Daher können diese ohne Bedenken über die Studie hinaus in \textit{NUNAV Navigation} integriert bleiben.

Weitere positive Ergebnisse, welche die Anwendung des Leitfadens in der Graphmasters GmbH mit sich gebracht hat, sind die entstandene Sensibilität für den Bedarf von Erklärungen bei \textit{End Usern} und das entstandene \textit{Feature-Flag}-System, welches es ermöglich in Zukunft einfacher Evaluationen für neu entwickelte Funktionen der mobilen Anwendungen durchzuführen.

% Bezug zu den Anforderungen des Leitfadens stellen (Forschungsfragen hat der Leitfaden an sich beantwortet)

\newpage

% \begin{enumerate}
%     \item[GR1] Der Leitfaden muss eine Unterstützung für das Erheben von Anforderungen an Erklärungen und das Aufstellen von Hypothesen über den Einfluss enthalten.
%     \item[GR2] Der Leitfaden muss mögliche Vorschläge zur Umsetzung von Erklärungen geben.
%     \item[GR3] Der Leitfaden muss einen Überblick über verschiedene Evaluationsmöglichkeiten für Erklärungen im Kontext externer Softwarequalität geben.
%     \item[GR4] Der Leitfaden muss in der Literatur gezeigte Einflüsse einzelner Eigenschaften auf die externe Qualität von Erklärungen zusammenfassen.
% \end{enumerate}