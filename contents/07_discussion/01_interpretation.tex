\section{Interpretation der Ergebnisse}

Für die Einordnung der Ergebnisse aus den beiden Teilen dieser Arbeit ist vorwiegend interessant, inwiefern der im ersten Teil entwickelte Leitfaden in der Praxis anwendbar ist. Dabei werden zunächst die enthaltenen Ergebnisse zusammengefasst. Für die Beurteilung des Nutzens des enthaltenen Modells, der Auswirkungen auf Qualitätsaspekte und der Heuristiken zur Integration von Erklärungen in ein bestehendes System wurde der Leitfaden bei der Firma \textit{Graphmasters} eingesetzt. Um die Anwendbarkeit des entwickelten Leitfadens zu analysieren, werden im Folgenden zum einen die Erfahrungen, die während der Anwendung gemacht wurden, ausgewertet. Zum anderen wird die Qualität der im Rahmen der Anwendung entstandenen Erklärungen betrachtet.

Der Leitfaden ist im Rahmen einer Literaturrecherche entstanden, in welcher die bestehenden Ergebnisse für das Design von Erklärungen in erklärbaren Systemen analysiert wurden. Das im Leitfaden für die Integration von Erklärungen enthaltene Modell beinhaltet die folgenden Teile:

\paragraph{RQ1} Unter \textit{External Dependencies} sind alle Rahmenbedingungen zusammengefasst, die einen direkten Einfluss auf die Anforderungen an Erklärungen aufweisen. Als relevante Aspekte sind verschiedene Ausprägungen des \textit{Contexts} von erklärbaren Systemen, sowie die \textit{Objectives} für die Integration von Erklärungen auf verschiedenen Abstraktionsebenen in dem Modell enthalten. Somit unterstützt dieser Modellteil die Anforderungserhebung für Erklärungen.

\paragraph{RQ2} Die Umsetzung der Anforderungen wird in dem vorgestellten Modell durch in der Forschung evaluierte \textit{Characteristics} unterstützt. Dabei enthält das Modell Eigenschaften von Erklärungen für die Umsetzung des Bedarfs für Erklärungen (\textit{Demand}), die transportierten Informationen (\textit{Content}) und die Art der Informationsvermittlung an \textit{End User} (\textit{Presentation}). Die im Modell vorgestellten Ausprägungen bieten unterschiedliche Möglichkeiten, um Erklärungen für verschiedene Kontexte zu gestalten. Aufgeführt sind lediglich Eigenschaften, für die ein Effekt auf die Qualität von Softwaresystemen bereits gezeigt werden konnte.

\paragraph{RQ3} Über die Entwicklung von Erklärungen hinaus bietet das Modell Hilfestellungen für die Evaluation von Erklärungen in einem System. Grundsätzlich werden die im \textit{Software Engineering} üblichen Studienformen vorgeschlagen, welche je nach Ziel einer Evaluation Anwendung finden können \cite[vgl.][]{wohlin2012experimentation}. Außerdem wird zwischen der direkten Messung der Qualität von Erklärungen sowie der Messung der Einflüsse von integrierten Erklärungen auf externe Qualitätsaspekte eines Systems unterschieden.

Um einen ganzheitlichen Überblick über die Qualität von Erklärungen zu erhalten, empfiehlt der Leitfaden, in den das Modell integriert ist, eine Kombination der verschiedenen Evaluationsmöglichkeiten. Die Erklärungen können sowohl direkt evaluiert werden als auch deren Einflüsse auf andere Qualitätsaspekte gemessen werden. Für die Bewertung der Qualität sind außerdem sowohl quantitative als auch qualitative Metriken notwendig, um die Performanz der \textit{End User} wie auch deren subjektive Wahrnehmung zu betrachten.

\paragraph{RQ4} Neben dem Modell für Erklärungen, enthält der Leitfaden einen Katalog der Erklärungen, welcher die Zusammenhänge der Eigenschaften von Erklärungen auf abhängige Qualitätsaspekte zusammenfasst. Dies beantwortet zum einen die Frage, unter welchen Bedingungen das Präsentieren von Erklärungen, welche Einflüsse auf ausgewählte externer Softwarequalitätsaspekte hat. Zum anderen werden ebenfalls die Auswirkungen der Granulariät von Erklärungen auf selbige Qualitätsaspekte zusammengefasst.

\bigskip

Abgeleitet aus den ersten Teilen des Leitfadens werden des Weiteren Heuristiken für die Gestaltung von Erklärungen. Dabei werden allgemein anwendbare Zusammenhänge vorgeschlagen, welche im Regelfall für die Integration von Erklärungen beachtet werden sollten.

\bigskip

Die im entwickelten Leitfaden enthaltenen Ergebnisse sind nicht als vollständiger Überblick, über die Möglichkeiten der Gestaltung von Erklärungen zu sehen. Viel mehr soll der Leitfaden eine Unterstützung und Anregungen zur Integration von Erklärungen bieten. Diese Empfehlung geschieht auf Basis von existierenden Ergebnissen, wodurch eine Orientierung an den Ergebnissen im Leitfaden in der Regel zu positiven Resultaten bei der Integration von Ergebnissen führen kann. Welche Einschränkungen für diese Aussage gelten muss allerdings zunächst in zukünftigen Arbeiten überprüft werden.

\bigskip

Als erste Prüfung der Aussage wurde der Leitfaden initial in der Firma \textit{Graphmasters} eingesetzt. Der Einstieg in die Anwendung des Leitfadens ist innerhalb eines Workshops erfolgt, welcher zusätzlich zur Vorstellung des Leitfadens mit einer Einführung in das Thema Erklärbarkeit begonnen hat (siehe \autoref{sec:explanation_demand_generation}). Anhand der Rückfragen, zum Beispiel, wo die genaue Abgrenzung zwischen Erklärbarkeit und gutem User-Interface-Design liegt, wurde deutlich, dass eine Anwendung des Leitfadens nicht ohne Vorwissen zum Thema \textit{Explainability} erfolgen kann. Der Leitfaden an sich enthält diese Einführung allerdings nicht.

Anhand von weiteren Rückfragen bei der Vorstellung des Leitfadens zur Integration von Erklärungen konnten außerdem Verständnisprobleme innerhalb des Leitfadens identifiziert werden. Insbesondere die Abgrenzung zwischen verschiedenen Inhaltstypen von Erklärungen haben nicht alle Teilnehmer des Workshops direkt verstanden. Die Typen waren zum Zeitpunkt der Durchführung als Fragewörter voneinander abgegrenzt. Für das finale Modell wurden daher einzelne Begriffe des Modells überarbeitet (siehe \autoref{fig:model_overview_complete}). Mit diesen integrierten Änderungen wird darauf geschlossen, dass Nutzer des Leitfadens mit Vorwissen über das Thema Erklärbarkeit diesen zu großen Teilen verstehen können. Eine direkte Evaluation der Verständlichkeit des Leitfadens ist allerdings im Rahmen dieser Arbeit nicht erfolgt.

Außerdem war während des Workshops zu beobachten, dass insbesondere die verschiedenen Möglichkeiten im Leitfaden, Erklärungen zu gestalten, die Diskussion zu Umsetzungsideen angeregt haben. Somit haben sich die verschiedenen \textit{Characteristics} zusammen mit den Einflüssen auf verschiedene Qualitätsaspekte als sehr hilfreich für die Entwicklung von Erklärungen herausgestellt. Zusätzlich hat das Modell geholfen, die Ideen zur Evaluation zu systematisieren und folglich Metriken festzulegen. Dabei konnte festgestellt werden, dass nicht alle Teile des Modells in jedem Kontext anwendbar sind.

Außerdem konnte der Leitfaden aus eigener Erfahrung bei der Konkretisierung der Ergebnisse des Workshops helfen. Die \textit{Objectives} im Modell boten dabei eine gute Hilfestellung zum Aufstellen eines Qualitätsmodells. Der Katalog der Zusammenhänge von bestimmten Eigenschaften, von Erklärungen und die Heuristiken für die Gestaltung haben die Umsetzung des Designs subjektiv gut unterstützt.

Aus der Evaluation der entstandenen Erklärungen kann des Weiteren gefolgert werden, dass durch die Anwendung des Leitfadens Erklärungen entwickelt werden konnten, welche positive Einflüsse auf die Softwarequalität eines Systems haben. Die Untersuchung des Einflusses der Erklärungen hat aber auch gezeigt, dass nicht alle entwickelten Erklärungen einen positiven Einfluss auf die Softwarequalität erreichen konnten. Hier hat allerdings die anhand des Leitfadens entwickelte zusätzliche direkte Evaluation der Erklärungen mögliche Probleme und Verbesserungsmöglichkeiten aufgedeckt. Diese können in einer weiteren Iteration der Erklärungen umgesetzt werden. Ferner deuten weder qualitative noch quantitative Ergebnisse der Evaluation auf negative Effekte durch die Erklärungen hin. Daher können diese ohne Bedenken über die Studie hinaus in \textit{NUNAV Navigation} integriert bleiben.

Aus den Ergebnissen des Quasi-Experiments ist außerdem hervorgegangen, dass alle vier verschiedenen Erklärungen unterschidlich von den Teilnehmern bewertet wurden. Da in der \textit{Case Study}, um eine möglichst hohe Zahl an \textit{End Usern} in jeder Studiengruppe innerhalb eines kurzen Zeitraums erreichen zu können, wurden die Erklärungen allerdings in zwei Gruppen zusammengefasst. In einer erneuten Studie sollte folglich die Dauer verlängert und die Erklärungen einzeln evaluiert werden. Auch könnte über einen längeren Zeitraum auch der Nutzerzuwachs von \textit{NUNAV Navigation} zwischen den Studiengruppen verglichen werden. Dies war während der durchgeführten Studie nicht möglich.

Folglich wird abgeleitet, dass in dieser Arbeit ein Leitfaden entstanden ist, welcher die geforderte Unterstützung bei der Integration von Erklärung bietet. Dabei wurden bei der Anwendung des Leitfadens alle Teile des Modells sowie der Katalog der Zusammenhänge von Aspekten von Erklärungen angewendet. Zudem haben die Heuristiken im Leitfaden das konkrete Design unterstützt.

Weitere positive Ergebnisse, welche die Anwendung des Leitfadens in der \textit{Graphmasters GmbH} mit sich gebracht hat, sind die entstandene Sensibilität für den Bedarf von Erklärungen bei \textit{End Usern} und das entwickelte \textit{Feature-Flag}-System, welches es ermöglicht in Zukunft einfacher Evaluationen für neu entwickelte Funktionen der mobilen Anwendungen durchzuführen.