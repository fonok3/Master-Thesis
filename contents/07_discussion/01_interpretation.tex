\section{Interpretation der Ergebnisse}

Für die Einordnung der Ergebnisse aus den beiden Teilen dieser Arbeit ist vor allem interessant, inwiefern der im ersten Teil entwickelte Leitfaden in der Praxis anwendbar ist. Für die Beurteilung des Nutzens des enthaltenen Modells, der Auswirkungen auf Qualitätsaspekte und der Heuristiken zur Integration von Erklärungen in ein bestehendes System wurde der Leitfaden bei der Firma Graphmasters eingesetzt. Um die Anwendbarkeit des entwickelten Leitfadens zu analysieren, werden im Folgenden zum einen die Erfahrungen, die während der Anwendung gemacht wurden, ausgewertet. Zum anderen wird die Qualität der im Rahmen der Anwendung entstandenen Erklärungen betrachtet.

Der Einstieg in die Anwendung des Leitfadens ist innerhalb eines Workshops erfolgt, welcher zusätzlich zur Vorstellung des Leitfadens mit einer Einführung in das Thema Erklärbarkeit begonnen hat (siehe \autoref{sec:explanation_demand_generation}). Anahnd der Rückfragen, zum Beispiel wo die genaue Abgrenzung zwischen Erklärbarkeit und gutem User-Interface-Design liegt, wurde deutlich, dass eine Anwendung des Leitfadens nicht ohne Vorwissen zum Thema \textit{Explainability} erfolgen kann. Der Leitfaden an sich enthält diese Einführung allerdings nicht.

Anhand von weiteren Rückfragen bei der Vorstellung des Leitfadens zur Integration von Erklärungen konnten außerdem Verständnisprobleme innerhalb des Leitfadens identifiziert werden. Insbesondere die Abgrenzung zwischen verschiedenen Inhaltstypen von Erklärungen haben nicht alle Teilnehmer des Workshops direkt verstanden. Die Typen waren zum Zeitpunkt der Durchführung als Fragewörter voneinander abgegrenzt. Für das finale Modell wurden daher einzelne Begriffe des Modells überarbeitet (siehe \autoref{fig:model_overview_complete}). Mit diesen integrierten Änderungen wird darauf geschlossen, dass Nutzer des Leitfadens mit Vorwissen über das Thema Erklärbarkeit diesen zu großen Teilen verstehen können. Eine direkte Evaluation der Verständlichkeit des Leitfadens ist allerdings im Rahmen dieser Arbeit nicht erfolgt.

Außerdem war während des Workshops zu beobachten, dass insbesondere die verschiedenen Möglichkeiten im Leitfaden, Erklärungen zu gestalten, die Diskussion zu Umsetzungsideen angeregt haben. Somit haben sich vor allem die verschiedenen \textit{Characteristics} zusammen mit den Einflüssen auf verschiedene Qualitätsaspekte als sehr hilfreich für die Entwicklung von Erklärungen herausgestellt. Zusätzlich hat das Modell geholfen, die Ideen zur Evaluation zu systematisieren und folglich Metriken festzulegen. Dabei konnte festgestellt werden, dass nicht alle Teile des Modells in jedem Kontext anwendbar sind.

Außerdem konnte der Leitfaden aus eigener Erfahrung bei der Konkretisierung der Ergebnisse des Workshops helfen. Die \textit{Objectives} im Modell boten dabei eine gute Hilfestellung zum Aufstellen eines Qualitätsmodells. Der Katalog der Zusammenhänge von bestimmten Eigenschaften von Erklärungen und die Heuristiken für die Gestaltung haben die Umsetzung des Designs subjektiv gut unterstützt.

Aus der Evaluation der entstandenen Erklärungen kann des Weiteren gefolgert werden, dass durch die Anwendung des Leitfadens Erklärungen entwickelt werden konnten, welche positive Einflüsse auf die Softwarequalität eines Systems haben. Die Untersuchung des Einflusses der Erklärungen hat aber auch gezeigt, dass nicht alle entwickelten Erklärungen einen positiven Einfluss auf die Softwarequalität erreichen konnten. Hier hat allerdings die anhand des Leitfadens entwickelte zusätzliche direkte Evaluation der Erklärungen mögliche Probleme und Verbesserungsmöglichkeiten aufgedeckt, welche in einer weiteren Iteration der Erklärungen umgesetzt werden können. Darüber hinaus deuten weder qualitative noch quantitative Ergebnisse der Evaluation auf negative Effekte durch die Erklärungen hin. Daher können diese ohne Bedenken über die Studie hinaus in \textit{NUNAV Navigation} integriert bleiben.

Folglich wird abgeleitet, dass in dieser Arbeit ein Leitfaden entstanden ist, welcher die geforderte Unterstützung bei der Integration von Erklärung bietet. Dabei wurden bei der Anwendung des Leitfadens alle Teile des Modells sowie der Katalog der Zusammenhänge von Aspekten von Erklärungen angewendet. Darüber hinaus haben die Heuristiken im Leitfaden das konkrete Design unterstützt.

Weitere positive Ergebnisse, welche die Anwendung des Leitfadens in der Graphmasters GmbH mit sich gebracht hat, sind die entstandene Sensibilität für den Bedarf von Erklärungen bei \textit{End Usern} und das entwickelte \textit{Feature-Flag}-System, welches es ermöglicht in Zukunft einfacher Evaluationen für neu entwickelte Funktionen der mobilen Anwendungen durchzuführen.